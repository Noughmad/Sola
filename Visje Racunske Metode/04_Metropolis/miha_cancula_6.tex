\documentclass[a4paper,10pt]{article}
\usepackage[utf8]{inputenc}
\usepackage[slovene]{babel}
\usepackage{graphicx}
\usepackage{hyperref}
\usepackage[left=2cm,right=2cm,top=2cm,bottom=3cm]{geometry}
\usepackage{amsmath}
\usepackage{float}
\usepackage{bbold}

\makeatletter
\renewcommand*\env@matrix[1][*\c@MaxMatrixCols c]{%
  \hskip -\arraycolsep
  \let\@ifnextchar\new@ifnextchar
  \array{#1}}
\makeatother

%opening
\title{Klasi\v cni in kvantni Monte Carlo}
\author{Miha \v Can\v cula}

\newcommand{\uvec}[1]{\ensuremath{\underline{#1}}}
\newcommand{\bb}{
  \ensuremath{b_1 b_2 \ldots b_n}
}
\newcommand{\psibb}{
  \ensuremath{\psi_{\bb}}
}

\begin{document}

\maketitle

\section{Isingov model}

Metropolisov algoritem sem uporabil na 2D Isingovem modelu brez zunanjega polja

\begin{align}
 H &= -\sum_{\langle r, r' \rangle} \sigma_r \sigma_{r'} 
\end{align}


Uporabil sem enostavno implementacijo, kjer ob vsaki potezi naklju"cno izberem en delec in mu z dolo"ceno verjetnostjo obrnem spin. 
Kot za"cetni pogoj sem vsakih uporabil povsem naklju"cno mre"zo. 

Kriti"cno temperaturo sem ocenil s pomo"cjo sprotnega merjenja energije, magnetizacije, specifi"cne toplote in magnetne susceptibilnosti. 
V termodinamskih sistemih v bli"zini faznega prehoda specifi"cna toplota in susceptibilnost divergirata, zaradi prostorske omejitve ra"cunskega modela pa imata pri prehodu le vrhove. 

\begin{figure}[H]
 \centering
 \input{g_plot_ising_cv}
\end{figure}

Toplotna kapaciteta sistema je sorazmerna z varianco energije, $C_v = \beta^2 ( \langle E^2 \rangle - \langle E \rangle^2)$. 
"Ce izraz delimo s "stevilom delcev, torej $N^2$, dobimo specifi"cno toploto, ki je prikazana na zgornji sliki. 
Opazimo, da polo"zaj in ostrina vrha nista odvisna od $N$, vsaj pri dovolj majhnih $N$, kjer "se dose"zemo termalizacijo. 

Z grafa lahko od"citamo temperaturo prehoda oz. $\beta_c$ kot najvi"sjo to"cko krivulje. 
Tudi pri velikem $N$ vrh ni oster, zato lahko maksimum le ocenimo na $\beta_c \approx 0.5$. 
Prava vrednosti je $\beta_c = \frac{1}{2} \log(1 + \sqrt{2}) \approx 0.44$, kar je precej blizu rezultatom simulacije. 

Vidne so tudi mo"cne oscilacije specifi"cne toplote pri velikih $\beta$. 
Pri manj"sem "stevilu korakov so oscilacije mo"cnej"se, zato domnevam, da se sistem tam "se ni termaliziral. 
Zaradi nizke temperature zelo te"zko pride iz lokalnih minimumov energije. 
Pri ve"cjem $N$, kjer za termalizacijo potrebujemo ustrezno ve"cje "stevilko korakov, so oscilacije izrazitej"se, 
vseeno pa razlo"cimo vrh pri $\beta \approx 0.5$. 

\begin{figure}[H]
 \centering
 \input{g_plot_ising_chi}
\end{figure}

Magnetna susceptibilnost sistema $\chi = \beta(\langle M^2 \rangle - \langle M \rangle^2)$ se obna"sa podobno kot specifi"cna toplota. 
Pri majhnih $N$ je vrh bolj izrazit, z ve"canjem velikosti sistema in konstantnem "stevilu potez pa postaja vse manj viden, pove"cujejo pa se tudi oscilacije pri velikih $\beta$. 
Polo"zaj vrha je spet v bli"zini analiti"cne vrednosti. 

Za gornja grafa sem naredil po 1000 meritev vsake spremenjivke, med zaporednimi meritvami pa sem izvedel 10000 korakov Metropolisovega algoritma. 
Pre za"cetkom meritev sem napravil $10^7$ korakov, da je sistem pri"sel vsaj pribli"zno v ravnovesje. 
Najve"cji sistem je imel $100 \time 100 = 10000$ spinov, tako da sem vsak spin posku"sal obrniti pribli"zno 1000-krat. 
Kljub temu pa sistem ne dose"ze termi"cnega ravnovesja, zlasti pri nizki temperaturi. 
Poleg tega ni dovolj korakov med posameznimi meritvami energije in magnetizacije, tako da te med seboj niso neodvisne. 
V primeru najve"cjega obravnavanega sistema med dvema meritvama vsak spin v povpre"cju poskusim obrniti le enkrat, kar ni dovolj za neodvisnost meritev. 
"Zal mi omejena ra"cunalni"ska zmogljivost ni dovoljevala obravnavanja sistemov z ve"cim "stevilom korakov. 

\section{Kvantni harmonski oscilator}

Simuliral sem tudi kvantni harmonskni oscilator. 
Harmonski oscilator ima le eno prostostno stopnjo, to je polo"zaj $q$. 
Ker pa operatorja kineti"cne in potencialne energije ne komutirata, sem izraz $e^{-\beta H}$ razcepil na produkt $M$ "clenov

\begin{align}
 e^{-\beta H} &= \left[e^{-\frac{\beta}{M}H}\right]^M = \exp\left(-{\frac{\beta}{M}V}\right) \exp\left(-{\frac{\beta}{M}T}\right) \exp\left(-{\frac{\beta}{M}V}\right) \exp\left(-{\frac{\beta}{M}T}\right) \cdots \\
 Z &= \sum_{q} \left\langle q \left|\exp\left(-{\frac{\beta}{M}V}\right) \exp\left(-{\frac{\beta}{M}T}\right) \exp\left(-{\frac{\beta}{M}V}\right) \exp\left(-{\frac{\beta}{M}T}\right) \cdots \right| q \right\rangle \\
 &= \sum_{q_1, q_2, \ldots} \left\langle q_1 \left|\exp\left(-{\frac{\beta}{M}V}\right) \right|q_1\right\rangle \left\langle q_1 \left| \exp\left(-{\frac{\beta}{M}T}\right) \right|q_2\right\rangle \cdots
\end{align}


Metropolisov algoritem je podoben kot v klasi"cnem primeru.
Trenutno stanje namesto mre"ze spinov predstavlja $M$ skalarjev $q_j$, "clen fazne vsote pa je enak

\begin{align}
 \exp\left(-\beta E(q_1, q_2, \ldots, q_M)\right) &= \exp(-\sum_{j=1}^M \left( \frac{M}{2\beta} (q_{j+1} - q_j)^2 + \frac{\beta}{M} V(q_j) \right)
\end{align}

Poteza je bila sprememba enega izmed $q_j$, $q_j \to q_j + \varepsilon \xi$, kjer je $\xi$ normalno Gaussovo porazdeljeno "stevilo. 
Parameter $\varepsilon$, ki dolo"ca povpre"cno velikost poteze, sem dinami"cno spreminjal v odvisnosti od $\beta$ in $M$, da je bil dele"z sprejetih potez vedno blizu 1/2.
Za primerno vrednost se je izkazal izraz $\varepsilon = 0.1 \min\{1, \sqrt{\beta}\}$

\subsection{Anharmonski oscilator}

Algoritem se ne spremeni, "ce uporabimo druga"cen potencial, ki je "se vedno diagonalen v bazi $|q\rangle$. 
Zato sem isti ra"cun ponovil "se z anharmonskim oscilatorjem, $V(q) = \frac{1}{2}q^2 + \lambda q^4$, za nekaj razli"cnih vrednosti $\lambda$. 

S pomo"cjo Metropolisovega vzor"cenja sem opazoval odvisnost povpre"cne energije $\langle H \rangle$ od inverzne temperature $\beta$. 
Rezultati so na spodnjem grafu. 

\begin{figure}[H]
 \centering
 \input{g_plot_energy}
\end{figure}

V harmonskem primeru povpre"cna energija pri"cakovano pada z inverzno temperaturo, v limiti $\beta \to \infty$ pa se ustali pri kon"cni vrednosti $\frac{1}{2}$. 

Nepri"cakovan in nefizikalen rezultat pa dobimo v nizkotemperaturni limiti z motenim potencialom, torej $\lambda > 0$. 
V tem primeru povpre"cna energija naraste do neke kon"cne vrednosti, ki je odvisna od $\lambda$ in tudi od $\varepsilon$. 
Padanje energije s temperaturo ni fizikalno, zato domnevam, da je posledica napak ra"cunanja. 
V tem primeru je $\beta/M \gg 1$, torej Trotterjev razcep ne dr"zi ve"c. 
Skok energije pri nizkih temperaturah je odvisen od vrednosti parametra $\lambda$, torej od oblike potenciala. 
Odvisna je tudi od izbire "stevila segmentov $M$. 
"Ce je izbrani $M$ premajhen, potem Trotter-Suzukijev razcep ni ve"c dober pribli"zek za $e^{-\beta H}$, saj $\Delta\beta = \beta/M$ ni ve"c majhnen parameter. 
Po drugi strani pa prevelik $M$ pomeni, da potrebujemo veliko "stevilo korakov, da sistem pride v ravnovesje. 

Za najbolj"se so se izkazale majhne vrednosti za $M$ nekje med 5 in 25. Za predstavitem na zgorjem grafu sem izbral $M=10$. 
V tem primeru dobimo dobro ujemanje med teoreti"cno napovedjo $\langle H \rangle = \frac{1}{2}\coth\frac{\beta}{2}$ za harmonski oscilator in rezultati z $\lambda = 0$. 

Za primerjavo sem posebej ra"cunal samo potencialni del energije, torej $\langle V \rangle$. 
Ta je diagonalen v $q$, zato je dovolj, da opazujemo le povpre"cje $V(q_1)$ po Metropolisovi porazdelitvi. 
Rezultati so na spodnjem grafu. 

\begin{figure}[H]
 \centering
 \input{g_plot_pot}
\end{figure}

Odvisnosti so precej podobna kot na prej"snji sliki, torej zlasti pri velikem $\beta$ k energiji prispeva predvsem potencialni "clen. 
Pri majhnem $\beta$ oz visoki temperaturi se pojavijo mo"cne oscilacije. 
Oscilacije lahko razlo"zimo, saj se pri visoki temperaturi energija hitro pretvarja iz potencialne v kineti"cno in obratno. 

\end{document}
