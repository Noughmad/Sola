\documentclass[a4paper,10pt]{article}
\usepackage[utf8]{inputenc}
\usepackage[slovene]{babel}
\usepackage{graphicx}
\usepackage{hyperref}
\usepackage[left=2cm,right=2cm,top=2cm,bottom=3cm]{geometry}
\usepackage{amsmath}
\usepackage{float}
\usepackage{bbold}

\makeatletter
\renewcommand*\env@matrix[1][*\c@MaxMatrixCols c]{%
  \hskip -\arraycolsep
  \let\@ifnextchar\new@ifnextchar
  \array{#1}}
\makeatother

%opening
\title{Klasi\v cni in kvantni Monte Carlo}
\author{Miha \v Can\v cula}

\newcommand{\uvec}[1]{\ensuremath{\underline{#1}}}
\newcommand{\bb}{
  \ensuremath{b_1 b_2 \ldots b_n}
}
\newcommand{\psibb}{
  \ensuremath{\psi_{\bb}}
}

\begin{document}

\maketitle

\section{Isingov model}

Metropolisov algoritem sem uporabil na 2D Isingovem modelu. 
Uporabil sem enostavno implementacijo, kjer ob vsaki potezi naklju"cno izberem en delec in mu z dolo"ceno verjetnostjo obrnem spin. 
Kot za"cetni pogoj sem vsakih uporabil povsem naklju"cno mre"zo. 

Kriti"cno temperaturo sem ocenil s pomo"cjo sprotnega merjenja energije, magnetizacije, specifi"cne toplote in magnetne susceptibilnosti. 
V termodinamskih sistemih v bli"zini faznega prehoda specifi"cna toplota in susceptibilnost divergirata, zaradi prostorske omejitve ra"cunskega modela pa imata pri prehodu le vrhove. 

\begin{figure}[H]
 \centering
 \input{g_plot_ising_cv}
\end{figure}

\begin{figure}[H]
 \centering
 \input{g_plot_ising_chi}
\end{figure}

\section{Kvantni harmoni"cni oscilator}

\begin{figure}[H]
 \centering
 \input{g_plot_energy}
\end{figure}

\begin{figure}[H]
 \centering
 \input{g_plot_pot}
\end{figure}





\end{document}
