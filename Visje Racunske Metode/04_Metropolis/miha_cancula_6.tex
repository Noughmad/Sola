\documentclass[a4paper,10pt]{article}
\usepackage[utf8]{inputenc}
\usepackage[slovene]{babel}
\usepackage{graphicx}
\usepackage{hyperref}
\usepackage[left=2cm,right=2cm,top=2cm,bottom=3cm]{geometry}
\usepackage{amsmath}
\usepackage{float}
\usepackage{bbold}

\makeatletter
\renewcommand*\env@matrix[1][*\c@MaxMatrixCols c]{%
  \hskip -\arraycolsep
  \let\@ifnextchar\new@ifnextchar
  \array{#1}}
\makeatother

%opening
\title{Klasi\v cni in kvantni Monte Carlo}
\author{Miha \v Can\v cula}

\newcommand{\uvec}[1]{\ensuremath{\underline{#1}}}
\newcommand{\bb}{
  \ensuremath{b_1 b_2 \ldots b_n}
}
\newcommand{\psibb}{
  \ensuremath{\psi_{\bb}}
}

\begin{document}

\maketitle

\section{Isingov model}

Metropolisov algoritem sem uporabil na 2D Isingovem modelu. 
Uporabil sem enostavno implementacijo, kjer ob vsaki potezi naklju"cno izberem en delec in mu z dolo"ceno verjetnostjo obrnem spin. 
Kot za"cetni pogoj sem vsakih uporabil povsem naklju"cno mre"zo. 

Kriti"cno temperaturo sem ocenil s pomo"cjo sprotnega merjenja energije, magnetizacije, specifi"cne toplote in magnetne susceptibilnosti. 
V termodinamskih sistemih v bli"zini faznega prehoda specifi"cna toplota in susceptibilnost divergirata, zaradi prostorske omejitve ra"cunskega modela pa imata pri prehodu le vrhove. 

\begin{figure}[H]
 \centering
 \input{g_plot_ising_cv}
\end{figure}

Toplotna kapaciteta sistema je sorazmerna z varianco energije, $C_v \propto \langle E^2 \rangle - \langle E \rangle^2$. 
"Ce izraz delimo s "stevilom delcev, torej $N^2$, dobimo do konstante natan"cno specifi"cno toploto, ki je prikazana na zgornji sliki. 
Opazimo, da polo"zaj in ostrina vrha nista odvisna od $N$. 

\begin{figure}[H]
 \centering
 \input{g_plot_ising_chi}
\end{figure}

Susceptibilnost sistema v odsotnosti zunanjega polja se s spreminjanjem $N$ obna"sa bolj nepredvidljivo. 
Namesto vrha s poten"cnim padanjem na obe strani opazimo hiter skok v bli"zini $\beta_c$. 
Dodatno opazimo precej"snje razlike v obna"sanju pri razli"cnih velikostih sistema. 
Pri $N=256$ se pojavita celo dva vrha. 
Pri zmanj"sanju "stevila korakov enak pojav vidimo tudi pri manj"sih $N$, tako da je to le posledica dejstva, da sistem "se ni v ravnovesju. 

Za gornja grafa sem naredil po 1000 meritev vsake spremenjivke, med zaporednimi meritvami pa sem izvedel 10000 korakov Metropolisovega algoritma. 
Najve"cji sistem je imel $256 \time 256 \approx 65000$ spinov, zato posamezne meritve niso ve"c korelirane, sistem pa tudi ne dose"ze ravnovesja. 

\section{Kvantni harmonski oscilator}

Simuliral sem tudi kvantni harmonskni oscilator. 
Harmonski oscilator ima le eno prostostno stopnjo, to je polo"zaj $q$. 
Ker pa operatorja kineti"cne in potencialne energije ne komutirata, sem izraz $e^{-\beta H}$ razcepil na produkt $M$ "clenov

\begin{align}
 e^{-\beta H} &= \left[e^{-\frac{\beta}{M}H}\right]^M = \exp\left(-{\frac{\beta}{M}V}\right) \exp\left(-{\frac{\beta}{M}T}\right) \exp\left(-{\frac{\beta}{M}V}\right) \exp\left(-{\frac{\beta}{M}T}\right) \cdots \\
 Z &= \sum_{q} \left\langle q \left|\exp\left(-{\frac{\beta}{M}V}\right) \exp\left(-{\frac{\beta}{M}T}\right) \exp\left(-{\frac{\beta}{M}V}\right) \exp\left(-{\frac{\beta}{M}T}\right) \cdots \right| q \right\rangle \\
 &= \sum_{q_1, q_2, \ldots} \left\langle q_1 \left|\exp\left(-{\frac{\beta}{M}V}\right) \right|q_1\right\rangle \left\langle q_1 \left| \exp\left(-{\frac{\beta}{M}T}\right) \right|q_2\right\rangle \cdots
\end{align}


Metropolisov algoritem je podoben kot v klasi"cnem primeru.
Trenutno stanje namesto mre"ze spinov predstavlja $M$ skalarjev $q_j$, "clen fazne vsote pa je enak

\begin{align}
 \exp\left(-\beta E(q_1, q_2, \ldots, q_M)\right) &= \exp(-\sum_{j=1}^M \left( \frac{M}{2\beta} (q_{j+1} - q_j)^2 + \frac{\beta}{M} V(q_j) \right)
\end{align}

Poteza je bila sprememba enega izmed $q_j$, $q_j \to q_j + \varepsilon \xi$, kjer je $\xi$ normalno Gaussovo porazdeljeno "stevilo. 
Parameter $\varepsilon$, ki dolo"ca povpre"cno velikost poteze, sem dinami"cno spreminjal v odvisnosti od $\beta$ in $M$, da je bil dele"z sprejetih potez vedno blizu 1/2. 

Algoritem se ne spremeni, "ce uporabimo druga"cen potencial, ki je "se vedno diagonalen v bazi $|q\rangle$. 
Zato sem isti ra"cun ponovil "se z anharmonskim oscilatorjem, $V(q) = \frac{1}{2}q^2 + \lambda q^4$, za nekaj razli"cnih vrednosti $\lambda$. 

S pomo"cjo Metropolisovega vzor"cenja sem opazoval odvisnost povpre"cne energije $\langle H \rangle$ od inverzne temperature $\beta$. 
Rezultati so na spodnjem grafu. 

\begin{figure}[H]
 \centering
 \input{g_plot_energy}
\end{figure}

Po pri"cakovanju povpre"cna energija pada z inverzno temperaturo, v limiti $\beta \to \infty$ pa se ustali pri kon"cni vrednosti. 
Ta vrednost je mo"cno odvisna od vrednosti parametra $\lambda$, torej od oblike potenciala. Odvisna je tudi od izbire "stevila segmentov $M$. 
"Ce je izbrani $M$ premajhen, potem Trotter-Suzukijev razcep ni ve"c dober pribli"zek za $e^{-\beta H}$, saj $\Delta\beta = \beta/M$ ni ve"c majhnen parameter. 
Po drugi strani pa prevelik $M$ pomeni, da potrebujemo veliko "stevilo korakov, da sistem pride v ravnovesje. 
Kon"cna limitna vrednost energije pri $\beta \to \infty$ je lahko posledica enega izmed teh dveh te"zav. 

Pri majhni $\beta$ oz. visoki temperaturi opazimo "se en nenavaden pojav: skupna energija je negativna. 
Na logaritemskem grafu je to vidno le kot manjkajo"ce to"cke, ki so bolje vidne, "ce graf primerjamo z naslednjim. 
Harmonski potencial $V=\frac{1}{2}q^2$ je vedno pozitiven, negativen je lahko le kineti"cni del energije

$$\langle H_k \rangle = \left\langle \frac{M}{2\beta} - \frac{M}{2\beta} \sum_{j=1}^{M}(q_{j+1} - q_j)^2 \right\rangle$$

Ta "clen je negativen, "ce se zaporedni $q_j$ preve"c razlikujejo med seboj. 
Za zgornji grafu sem uporabil $M=20$, ampak oba nepri"cakovana pojava, tako negativno energijo kot kon"cno limitno vrednost, opazimo pri vseh izbirah $M$ med 5 in 500. 

Izbira parametra $\varepsilon$ in za"cetnega pogoja (vsi $q_j = 0$, ali pa naklju"cne vrednosti) na odvisnost le malo vplivata. 
Da sprejmemo pribli"zno polovico potez, je dobra izbira $\varepsilon \approx \sqrt{\beta}$, pri ve"cjih $\beta$ pa deluje pa tudi konstanten $\varepsilon \approx 1$. 
"Ce izberemo $\varepsilon$, ki so mo"cno razlikuje od zgornjih ocen, sistem ne dose"ze ravnovesja. 
Z dosti manj"sim $\varepsilon$ sem dobil rezultate, kjer povpre"cna energija ne nara"s"ca s temperaturo, kar ni fizikalno. 

Za primerjavo sem posebej ra"cunal samo potencialni del energije, torej $\langle V \rangle$. 
Ta je diagonalen v $q$, zato je dovolj, da opazujemo le povpre"cje $V(q_1)$ po Metropolisovi porazdelitvi. 
Rezultati so na spodnjem grafu. 

\begin{figure}[H]
 \centering
 \input{g_plot_pot}
\end{figure}

Odvisnosti so precej podobna kot na prej"snji sliki, torej zlasti pri velikem $\beta$ k energiji prispeva predvsem potencialni "clen. 

\end{document}
