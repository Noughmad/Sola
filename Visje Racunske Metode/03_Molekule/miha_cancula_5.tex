\documentclass[a4paper,10pt]{article}
\usepackage[utf8]{inputenc}
\usepackage[slovene]{babel}
\usepackage{graphicx}
\usepackage{hyperref}
\usepackage[left=2cm,right=2cm,top=2cm,bottom=3cm]{geometry}
\usepackage{amsmath}
\usepackage{float}
\usepackage{bbold}

\makeatletter
\renewcommand*\env@matrix[1][*\c@MaxMatrixCols c]{%
  \hskip -\arraycolsep
  \let\@ifnextchar\new@ifnextchar
  \array{#1}}
\makeatother

%opening
\title{Modelekularna dinamika}
\author{Miha \v Can\v cula}

\newcommand{\uvec}[1]{\ensuremath{\underline{#1}}}
\newcommand{\bb}{
  \ensuremath{b_1 b_2 \ldots b_n}
}
\newcommand{\psibb}{
  \ensuremath{\psi_{\bb}}
}

\begin{document}

\maketitle

\section{Transport toplote}

Transport toplote sem preu"ceval z enodimenzionalno verigo oscilatorjev. 
Delca na konceh verige sem sklopil s toplotnima kopelima z brezdimenzijskima temperaturama $T_L=1$ in $T_R=2$. 

Zanimajo nas vrednosti v stacionarnem stanju, zato sem najprej napravil $N_I = 10^{8}$ korakov integracije dol"zine $h=10^{-2}$. 
Nato sem izvedel "se $N_A=10^{7}$ korakov, med katerimi sem opravil $10^{5}$ meritev temperature in toplotnega toka. 
Meritve sem na koncu popre"cil. 

\section{Maxwellske kopeli}

Najprej sem za modeliranje toplotnih kopeli uporabil Maxwellov algoritem, pri katerem gibalni koli"cini sklopljenih delcev resetiramo vsakih $R=100$ korakov. 
Preizku"sal sem razli"cne vrednosti za $R$, najbolje pa se je izkazala tak"sna, pri kateri se reset zgodi enkrat vsako "casovno enoto, torej $Rh \sim 1$. 
Med reseti sem stanje propagiral s simplekti"cnim integratorjem s simetri"cno shemo $S_2$. 

\input{g_odv_lambda_30}

\input{g_odv_lambda_100}

\input{g_odv_velikost}

\input{g_prehod}

\section{Nos\'e-Hooverjeve kopeli}

\end{document}
