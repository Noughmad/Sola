\documentclass[a4paper,10pt]{article}
\usepackage[utf8]{inputenc}
\usepackage[slovene]{babel}
\usepackage{graphicx}
\usepackage{hyperref}
\usepackage[left=2cm,right=2cm,top=2cm,bottom=3cm]{geometry}
\usepackage{amsmath}
\usepackage{float}
\usepackage{bbold}

\makeatletter
\renewcommand*\env@matrix[1][*\c@MaxMatrixCols c]{%
  \hskip -\arraycolsep
  \let\@ifnextchar\new@ifnextchar
  \array{#1}}
\makeatother

%opening
\title{Modelekularna dinamika}
\author{Miha \v Can\v cula}

\newcommand{\uvec}[1]{\ensuremath{\underline{#1}}}
\newcommand{\bb}{
  \ensuremath{b_1 b_2 \ldots b_n}
}
\newcommand{\psibb}{
  \ensuremath{\psi_{\bb}}
}

\begin{document}

\maketitle

\section{Transport toplote}

Transport toplote sem preu"ceval z enodimenzionalno verigo oscilatorjev. 
Delca na konceh verige sem sklopil s toplotnima kopelima z brezdimenzijskima temperaturama $T_L=1$ in $T_R=2$. 

Zanimajo nas vrednosti v stacionarnem stanju, zato sem najprej napravil $N_I = 10^{8}$ korakov integracije dol"zine $h=10^{-2}$. 
Nato sem izvedel "se $N_A=10^{7}$ korakov, med katerimi sem opravil $10^{5}$ meritev temperature in toplotnega toka. 
Meritve sem na koncu popre"cil. 

\section{Maxwellske kopeli}

Najprej sem za modeliranje toplotnih kopeli uporabil Maxwellov algoritem, pri katerem gibalni koli"cini sklopljenih delcev resetiramo vsakih $R=100$ korakov. 
Preizku"sal sem razli"cne vrednosti za $R$, najbolje pa se je izkazala tak"sna, pri kateri se reset zgodi enkrat vsako "casovno enoto, torej $Rh \sim 1$. 
Med reseti sem stanje propagiral s simplekti"cnim integratorjem s simetri"cno shemo $S_2$. 

\begin{figure}[H]
 \centering
 \input{g_odv_lambda_30}
  \caption{Temperaturni profil v odvisnosti od parametra $\lambda$ za sistem s 30 delci}
\end{figure}


"Ze pri srednji velikosti sistema ($N=30$) opazimo odvisnost temperaturne porazdelitve od motnje $\lambda$. 
Veriga povsem harmoni"cnih oscilatorjev ($\lambda=0$) ima povsod, razen na obeh konceh, povpre"cno temperaturo. 
Pri $\lambda$ nad 1 temperatura med obema koncema nara"s"ca linearno, brez skokov. 
Pri vmesnih vrednosti pa opazimo kombinacijo obeh pojavov: na konceh ima temperatura diskretne skoke, vmes pa nara"s"ca linearno. 

Zelo podoben pojav opazimo tudi z ve"cjim sistemom, na primer $N=100$ na spodnji sliki. 


\begin{figure}[H]
 \centering
 \input{g_odv_lambda_100}
  \caption{Temperaturni profil v odvisnosti od parametra $\lambda$ za sistem s 100 delci}
\end{figure}

Za oceno termodinamske limite je zanimiva tudi odvisnost od velikosti sistema. 
Pri konstantnem parametru $\lambda = 0.6$ pri ve"cjem sistemu opazimo vedno manj"se robne skoke. 

\begin{figure}[H]
 \centering
 \input{g_odv_velikost}
\caption{Temperaturni profil pri fiksnem $\lambda = 0.6$ in razli"cnih velikostih sistema}
\end{figure}


Na prej"snjih slikah smo videli, da je temperaturni profil odvisen predvsem od motnje $\lambda$. 
Opazujemo lahko prehod med stanjem s konstantno temperaturo transportom ($\lambda=1$) in stanjem z linearnim profilom ($\lambda=1$). 
Z ve"canjem velikosti sistema $N$ postanejo za"cetni skoki pri srednji $\lambda$ vedno manj izraziti, zato domnevam,
da v limiti $N\to\infty$ nastane oster prehod med konstantnim in linearnim profilom. 

Polo"zaj prehoda sem posku"sal najti tako, da sem opazoval temperaturo v to"cki na 1/5 verige v odvisnosti od $\lambda$ in $N$. 
Rezultati so na spodnji sliki. 

\begin{figure}[H]
 \centering
 \input{g_prehod}
  \caption{Iskanje faznega prehoda med konstantnim in linearnim profilom}
\end{figure}

Dokler je $\lambda$ manj"sa od neke kriti"cne vrednosti, ki je nekje okrog $0.01$, temperatura ni odvisna niti od $\lambda$ niti od $N$. 
Nad to mejo pa temperatura pada z $\lambda$, in to hitreje pri ve"cjih $N$. 
To se sklada z domnevo, da v termodinamski limiti dobimo oster prehod. 
Na podlagi mojih izra"cunov mi ni uspelo dolo"citi natan"cnega polo"zaja prehoda, niti ne morem z gotovostjo trditi ali se ta zgodi pri $\lambda=0$ ali pri neki kon"cni vrednosti. 

Pri velikostih sistema $N \leq 200$ je odvisnost od $\lambda$ zvezna. 
Glede na to, da se z ve"canjem $N$ strmina pove"cuje, bi lahko pri"cakovali, da pri termodinamskem $N$ opazimo oster in nezvezen fazni prehod. 

\section{Nos\'e-Hooverjeve kopeli}

Pri uporabi Nos\'e-Hooverjevih kopeli sem imel te"zave predvsem pri iskanju pravega "casa $\tau$. 
Pri enakem "stevilu in dol"zini korakov je bil ra"cun z NH kopelmi bolj ob"cutljiv glede na relaksacijski "cas $\tau$ kot Maxwellov glede na interval resetiranja.
Posku"sal sem z razli"cnimi vrednostmi, najbolj"se rezultate pa sem dobil s $\tau \approx 1$. 
"Casovni razvoj sem simuliral z integratorjem \texttt{RK4} knji"znice \texttt{GSL}. 

\begin{figure}[H]
\centering
\input{g_hoover_lambda_10}
\caption{Temperaturni profil v odvisnosti od parametra $\lambda$ za sistem z 10 delci in $\tau = 1$}
\end{figure}

Opazimo podoben vzorec kot z uporabo Maxwellskih kopeli. 
Razlika je najve"cja pri majhnem $\lambda$, kjer sicer najdemo temperaturni plato, ki pa ima temperaturo ni"zjo od povpre"cja kopeli. 
Pri dovolj velike $\lambda$ se vzpostavi pri"cakovan linearen profil temperature. 

Rezultati za ve"cjih sistem ($N=50$) so zelo podobni, le da je temperaturni plato pri majhnem $\lambda$ "se hladnej"si. 

\begin{figure}[H]
\centering
\input{g_hoover_lambda_50}
\caption{Temperaturni profil v odvisnosti od parametra $\lambda$ za sistem s 50 delci in $\tau = 1$}
\end{figure}

Pri pri majhnem $\lambda$ opazimo tudi mo"cno odvisnost od izbire $\tau$. 
Pri vrednosti $\tau = 3$, na primer, je plato dvignjen, medtem ko je pri $\tau = 1$ spu"s"cen. 
Glede na te rezultate bi dosti bolj zaupal algoritmu z naklju"cnimi Maxwellskimi kopelmi. 

\section{Toplotni tok}

V vseh primerih sem poleg lokalne temperature ra"cunal tudi toplotni tok. 
Uporabil sem le Maxwellske kopeli, ker so pri ra"cunanju temperature dale bolj"se rezultate. 

\begin{figure}[H]
 \centering
 \input{g_odv_lambda_100_tok}
  \caption{Toplotni tok v odvisnosti od parametra $\lambda$ za sistem s 100 delci}
\end{figure}

Pri vseh vrednosti $\lambda$ opazimo konstanten tok po celotni verigi. 
To je seveda pri"cakovano, saj se v stacionarnem stanju toplota nikjer ne zadr"zuje. 
Je pa tok skozi sistem mo"cno odvisen od $\lambda$. 

Ker je lokalni tok neodvisen od polo"zaja v verigi, so bolj nazorni grafi odvisnosti povpre"cnega toka od $\lambda$ in od $N$. 

\begin{figure}[H]
 \centering
 \input{g_tok_lambda}
\caption{Odvisnost povpre"cnega toplotnega toka od $\lambda$ pri razli"cnih velikostih sistema}
\end{figure}

Po pri"cakovanju opazimo upadanje skupnega toka z ve"canjem $\lambda$. 
Pri nelinearnem sistemu pride do sipanja med fononi, zaradi "cesar je transport manj u"cinkovit. 

Bolj pomembna je verjetno odvisnost toka od velikosti sistema $\overline{J}(N)$, zlasti "ce je ta poten"cna $\overline{J} \propto N^{-p}$. 

\begin{figure}[H]
 \centering
 \input{g_tok_N}
\caption{Odvisnost povpre"cnega toplotnega toka od velikosti sistema pri razli"cnih $\lambda$}
\end{figure}

Pri $\lambda = 0$ je tok neodvisen od velikosti, torej imamo balisti"cni transport z eksponentom $p=0$. 
Z ve"canjem $\lambda$ pa za"cne tok padati z $N$, kar nakazuje difuzijski transport. 
Vrednost eksponenta pa la"zje dolo"cimo z logaritemskim grafom. 

\begin{figure}[H]
 \centering
 \input{g_tok_N_log}
\caption{Odvisnost povpre"cnega toplotnega toka od velikosti sistema pri razli"cnih $\lambda$}
\end{figure}

Prilagajanje premice da rezultat $p\approx 0.7$ za $\lambda=1$.
To je "se vedno manj od vrednosti 1 pri normalnem difuzijskem transportu. 
Pri nadaljnjem pove"cevanju $\lambda$, na primer do 2, pa eksponent $p$ dose"ze pri"cakovano vrednost 1. 

\end{document}
