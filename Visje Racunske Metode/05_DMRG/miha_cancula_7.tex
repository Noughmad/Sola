\documentclass[a4paper,10pt]{article}
\usepackage[utf8]{inputenc}
\usepackage[slovene]{babel}
\usepackage{graphicx}
\usepackage{hyperref}
\usepackage[left=2cm,right=2cm,top=2cm,bottom=3cm]{geometry}
\usepackage{amsmath}
\usepackage{float}
\usepackage{bbold}

\makeatletter
\renewcommand*\env@matrix[1][*\c@MaxMatrixCols c]{%
  \hskip -\arraycolsep
  \let\@ifnextchar\new@ifnextchar
  \array{#1}}
\makeatother

%opening
\title{Metode \textsc{DMRG}}
\author{Miha \v Can\v cula}

\newcommand{\uvec}[1]{\ensuremath{\underline{#1}}}
\newcommand{\bb}{
  \ensuremath{b_1 b_2 \ldots b_n}
}
\newcommand{\psibb}{
  \ensuremath{\psi_{\bb}}
}

\begin{document}

\maketitle

\section{Entropija prepletenosti}

Entropijo prepletenosti stanja $\Psi$ izra"cunamo tako, da vektor $\Psi$ pretvorimo v matriko, kjer stanje spinov v obmo"cju $A$ indeksira stolpec matrike, stanje spinov v obmo"cju $B$ pa vrstico. Za tak"sno matriko lahko z razcepom \textsc{SVD} izra"cunamo singularne vrednosti oz. Schmidtove koeficiente $\lambda_\mu$. Entropija prepletenosti je tedaj enaka

\begin{align}
 S = -\sum_\mu \lambda_\mu^2 \log \lambda_\mu^2
\end{align}

"Ce je v obmo"cju $A$ prvih $n_A$ spinov, ostali pa v obmo"cju $B$, je pretvorba vektorja $\Psi$ v matriko enostavna. Vektor razdelimo na segmente dol"zine $N_A = 2^{n_A}$, nato pa ti segmenti postanejo stolpci matrike. V programu $\texttt{Octave}$ temu ustreza funkcija \texttt{reshape}. Operacija je te"zavnej"sa, "ce je obmo"cje $A$ bolj zapleteno, na primer "ce izberemo vsak drugi spin. 

\section{Razcep na produkt matrik}

\end{document}
