\documentclass[a4paper,10pt]{article}
\usepackage[utf8]{inputenc}
\usepackage[slovene]{babel}
\usepackage{graphicx}
\usepackage{hyperref}
\usepackage[left=2cm,right=2cm,top=2cm,bottom=3cm]{geometry}
\usepackage{amsmath}
\usepackage{float}
\usepackage{bbold}

\makeatletter
\renewcommand*\env@matrix[1][*\c@MaxMatrixCols c]{%
  \hskip -\arraycolsep
  \let\@ifnextchar\new@ifnextchar
  \array{#1}}
\makeatother

%opening
\title{Metode \textsc{DMRG}}
\author{Miha \v Can\v cula}

\newcommand{\uvec}[1]{\ensuremath{\underline{#1}}}
\newcommand{\bb}{
  \ensuremath{b_1 b_2 \ldots b_n}
}
\newcommand{\psibb}{
  \ensuremath{\psi_{\bb}}
}

\begin{document}

\maketitle

\section{Entropija prepletenosti}

Entropijo prepletenosti stanja $\Psi$ izra"cunamo tako, da vektor $\Psi$ pretvorimo v matriko, kjer stanje spinov v obmo"cju $A$ indeksira stolpec matrike, stanje spinov v obmo"cju $B$ pa vrstico. Za tak"sno matriko lahko z razcepom \textsc{SVD} izra"cunamo singularne vrednosti oz. Schmidtove koeficiente $\lambda_\mu$. Entropija prepletenosti je tedaj enaka

\begin{align}
 E = -\sum_\mu \lambda_\mu^2 \log \lambda_\mu^2
\end{align}

\subsection{Kompaktni obmo"cji}

"Ce je v obmo"cju $A$ prvih $n_A$ spinov, ostali pa v obmo"cju $B$, je pretvorba vektorja $\Psi$ v matriko enostavna. Vektor razdelimo na segmente dol"zine $N_A = 2^{n_A}$, nato pa ti segmenti postanejo stolpci matrike. V programu $\texttt{Octave}$ temu ustreza funkcija \texttt{reshape}. Operacija je te"zavnej"sa, "ce je obmo"cje $A$ bolj zapleteno, na primer "ce izberemo vsak drugi spin. Entropijo prepletenosti sem ra"cunal za razli"cne particije, izbiral sem razli"cne velikosti za $n$ in $n_A$, nato pa izra"cunal entropijo prepletenosti za osnovno stanje in nekaj naklju"cnih stanj. Za vsako izbiro $n_A$ sem generiral 30 naklju"cnih stanj. 

\begin{figure}[H]
 \centering
 \input{g_entropija}
 \caption{Entropija prepletenosti v odvisnosti od velikosti obmo"cja $A$. Obe obmo"cji sta kompaktni. }
 \label{fig:entropija}
\end{figure}

Ne glede na izbiro stanja je entropija prepletenosti najvi"sja, "ce sta obmo"cji $A$ in $B$ enako veliki, in je simetri"cna na zamenjavo obmo"cij. 

\subsection{Odvisnost od velikosti sistema}

Za oceno termodinamske limite je uporabna zlasti odvisnost od velikosti sistem $n$. 

\subsection{Nekompaktni obmo"cji}


\section{Razcep na produkt matrik}

Implementiral sem tudi algoritem za razcep poljubnega stanja $\Psi$ na produkt matrik. 

Pravilnost razcepa na matri"cni produkt je enostavno preverjati. 
Treba je le zmno"ziti matrike med seboj in rezultat primerjati z ustreznim koeficientom stanja $\Psi$. 
Za to"cen razcep, kjer velikost matrik ni omejena, mno"zenje matrik vrne rezultat, ki je pravilen do strojne natan"cnosti ra"cunalnika. 
"Zal pa to"cen razcep porabi veliko ra"cunalni"skega spomina, zato je bolj u"cinkovito odstraniti najmanj"se Schmidtove koeficiente in s tem zmanj"sati matrike. 
To sem naredil na dva na"cina: takoj sem zanemaril vse singularne vrednosti, manj"se $\varepsilon$, "ce je velikost matrik preve"c narasla pa sem upo"steval le prvih $M$ vrednosti. 
Bele"zil sem vsoto kvadratov vseh ``odrezanih'' singularne vrednosti, saj je ta merilo za napako. 

\end{document}
