\documentclass[a4paper,10pt]{article}
\usepackage[utf8]{inputenc}
\usepackage[slovene]{babel}
\usepackage{graphicx}
\usepackage{hyperref}
\usepackage[left=2cm,right=2cm,top=2cm,bottom=3cm]{geometry}
\usepackage{amsmath}
\usepackage{float}
\usepackage{bbold}

\makeatletter
\renewcommand*\env@matrix[1][*\c@MaxMatrixCols c]{%
  \hskip -\arraycolsep
  \let\@ifnextchar\new@ifnextchar
  \array{#1}}
\makeatother

%opening
\title{Metode \textsc{DMRG}}
\author{Miha \v Can\v cula}

\newcommand{\uvec}[1]{\ensuremath{\underline{#1}}}
\newcommand{\bb}{
  \ensuremath{b_1 b_2 \ldots b_n}
}
\newcommand{\psibb}{
  \ensuremath{\psi_{\bb}}
}

\begin{document}

\maketitle

\section{Entropija prepletenosti}

Entropijo prepletenosti stanja $\Psi$ izra"cunamo tako, da vektor $\Psi$ pretvorimo v matriko, kjer stanje spinov v obmo"cju $A$ indeksira stolpec matrike, stanje spinov v obmo"cju $B$ pa vrstico. Za tak"sno matriko lahko z razcepom \textsc{SVD} izra"cunamo singularne vrednosti oz. Schmidtove koeficiente $\lambda_\mu$. Entropija prepletenosti je tedaj enaka

\begin{align}
 E = -\sum_\mu \lambda_\mu^2 \log \lambda_\mu^2
\end{align}

\subsection{Kompaktni obmo"cji}

"Ce je v obmo"cju $A$ prvih $n_A$ spinov, ostali pa v obmo"cju $B$, je pretvorba vektorja $\Psi$ v matriko enostavna. Vektor razdelimo na segmente dol"zine $N_A = 2^{n_A}$, nato pa ti segmenti postanejo stolpci matrike. V programu $\texttt{Octave}$ temu ustreza funkcija \texttt{reshape}. Operacija je te"zavnej"sa, "ce je obmo"cje $A$ bolj zapleteno, na primer "ce izberemo vsak drugi spin. Entropijo prepletenosti sem ra"cunal za razli"cne particije, izbiral sem razli"cne velikosti za $n$ in $n_A$, nato pa izra"cunal entropijo prepletenosti za osnovno stanje in nekaj naklju"cnih stanj. Za vsako izbiro $n_A$ sem generiral 30 naklju"cnih stanj. 

\begin{figure}[H]
 \centering
 \input{g_entropija}
 \caption{Entropija prepletenosti v odvisnosti od velikosti obmo"cja $A$. Obe obmo"cji sta kompaktni. }
 \label{fig:entropija}
\end{figure}

Ne glede na izbiro stanja je entropija prepletenosti najvi"sja, "ce sta obmo"cji $A$ in $B$ enako veliki, in je simetri"cna na zamenjavo obmo"cij. 

\subsection{Odvisnost od velikosti sistema}

Za oceno termodinamske limite je uporabna zlasti odvisnost od velikosti sistem $n$. Pri nekaj razli"cnih velikostih sem izra"cunal entropije delitve na dve enaki polovici in delitve, kjer je v obmo"cju $A$ le en spin. 

\begin{figure}[H]
 \centering
 \input{g_velikost}
 \caption{Entropija prepletenosti v odvisnosti od velikosti sistema $n$. Obe obmo"cji sta kompaktni in enako veliki. }
 \label{fig:velikost}
\end{figure}

Entropija delitve z $n_A = 1$ ni odvisna od velikosti sistema, medtem ko entropije delitve na enaki obmo"cji opazno nara"s"ca z velikostjo. 
V skladu s prej"snjim grafom lahko sklepamo, da je entropija prepletenosti odvisna predvsem od velikosti manj"sega izmed obmo"cij. 

\subsection{Nekompaktni obmo"cji}

Opazoval sem tudi obna"sanje entropije, ko ne vzamemo kompaktnih obmo"cji, ampak k $A$ "stejemo vsak $k$-ti spin. 
S tem seveda spremenimo "stevilo vezi na meji, tak"snih vezi je to"cno $2/k$, "ce je $k \geq 2$. 

\begin{figure}[H]
 \centering
 \input{g_nekompaktna}
 \caption{Entropija prepletenosti med dvema nekompaktnima obmo"cjema.  }
 \label{fig:nekompaktna}
\end{figure}

Uporabil sem sistem velikosti $n=12$, ki je precej deljivo "stevilo, zato da sem lahko za obmo"cje $A$ lahko "stel vsak drugi, vsak tretji, vsak "cetrti ali vsak "sesti spin. 
Entropija prepletenosti ka"ze lepo odvisnost od "stevila vezi, $E \propto k^{-3/4}$. 

\section{Razcep na produkt matrik}

Implementiral sem tudi algoritem za razcep poljubnega stanja $\Psi$ na produkt matrik. 
Uporabil sem programsko okolje \texttt{Octave}, ki je primerno za delo z matrikami. 
Na vsakem koraku sem za preoblikovanje ``gosenice'' uporabil funkcijo \texttt{reshape}, tako da se mi ni bilo treba ukvarjati s preve"c zankami in indeksi. 

Pravilnost razcepa na matri"cni produkt je enostavno preverjati. 
Treba je le zmno"ziti matrike med seboj in rezultat primerjati z ustreznim koeficientom stanja $\Psi$. 
Za to"cen razcep, kjer velikost matrik ni omejena, mno"zenje matrik vrne rezultat, ki je pravilen do strojne natan"cnosti ra"cunalnika. 

Implementacija v \texttt{Octave} je dostopna na internetu na naslovu \url{https://github.com/Noughmad/Sola/blob/master/Visje%20Racunske%20Metode/05_DMRG/mpa.m}. 
Prilo"zena je funkcija, ki preverja pravilno delovanje za naklju"cno stanje $\Psi$. 

\end{document}
