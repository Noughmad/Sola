\documentclass[a4paper,10pt]{article}
\usepackage[utf8]{inputenc}
\usepackage[slovene]{babel}
\usepackage{graphicx}
\usepackage{hyperref}
\usepackage[left=2cm,right=2cm,top=2cm,bottom=3cm]{geometry}
\usepackage{amsmath}
\usepackage{float}
\usepackage{bbold}

\makeatletter
\renewcommand*\env@matrix[1][*\c@MaxMatrixCols c]{%
  \hskip -\arraycolsep
  \let\@ifnextchar\new@ifnextchar
  \array{#1}}
\makeatother

%opening
\title{Algoritem \textsc{TEBD}}
\author{Miha \v Can\v cula}

\newcommand{\uvec}[1]{\ensuremath{\underline{#1}}}
\newcommand{\bb}{
  \ensuremath{b_1 b_2 \ldots b_n}
}
\newcommand{\psibb}{
  \ensuremath{\psi_{\bb}}
}

\begin{document}

\maketitle

\section{Algoritem}

Algoritem \textsc{TEBD} sem implementiral v programskem jeziku \texttt{Octave}.
Pri tem sem uporabil funkcijo za razcep na matri"cni produkt iz prej"snje naloge. 
Uporabil sem tudi dvodel"cni propagator za Heisenbergove verige, ki smo ga uporabljali "ze pri nalogi o Trotter-Suzukijevem razcepu. 

Algoritem je podrobno opisan v navodilih, kot implementacijski detajl lahko omenim le, da sem za razcep matrik namesto vgrajene metode \texttt{svg} uporabil metodo \texttt{svgs}, ki deluje iterativno in vrne le zahtevano "stevilo singularnih vrednosti. Dimenzijo vezi $M_j$ sem omejil tako, da sem uporabil najve"c $M = 200$ singularnih vrednosti, dodatno pa sem zavrgel "se vse tisto, ki so po absolutni vrednosti vsaj $10^6$-krat manj"se od najve"cje. 

Pri propagaciji oz. iskanju termi"cnih stanj norma stanja nara"s"ca eksponentno, zato sem stanje po nekaj korakih ponovno normaliziral. 
Za normalizacijo sem izra"cunal vse "clene vektorja stanja $\psi$, vektor normaliziral, nato pa zopet izvedel razcep na produkt matrik. 
Tak"sen postopek je zamuden, zato ga nisem izvajal na vsakem koraku propagacije. 

Vedno sem uporablja antiferomagneten Heisenbergov Hamiltonian. 

\section{Osnovno stanje Heisenbergove verige}

"Ce sem naklju"cno stanje propagiral z realnim $z = \beta$, kar ustreza iskanju termi"cnih stanj, pri dovolj velikem $\beta$ program vrne dober pribli"zek za osnovno stanje, v obliki matri"cnega produkta. Iz matri"cnega produkta bi lahko izra"cunali stanje $\psi$, ampak za predstavitev je bolj koristno ra"cunanje energije osnovnega stanja in korelacije v tem stanju. 

Energijo stanja bi lahko izra"cunal z delovanjem Hamiltoniana na stanje, kot $E = \langle \psi |H|\psi\rangle$. 
Ker pa je matrika $H$ lahko zelo velika, sem prihranil na ra"cunalni"skem spominu z uporabo enakosti 
\begin{align}
 E = -\lim_{\beta\to\infty} \frac{\|\exp(-\beta H)|\psi_i\rangle\|}{\beta}
\end{align}
ki velja, "ce je na za"cetno stanje $\psi_i$ normirano. Obstoj oz. konvergenca te limite je dobro viden na naslednji sliki. 

\begin{figure}[H]
\centering
\input{g_osn_norma}
\caption{Odvisnost norme $\langle \psi | \psi \rangle$ od inverzne temperature $\beta$}
\label{fig:norma}
\end{figure}

Na grafu vidimo lepo eksponentno nara"s"canje norme, torej so prispevki vi"sjih stanj res zanemarljivi. 
Naklon premice na logaritemskem grafu je odvisen od velikosti sistema. 

\section{Spinske korelacije}

Opazoval sem tudi korelacije med spini posameznih delcev. 
Najbolj zanimiva je verjetno odvisnost korelacije $\langle \sigma_j^z \sigma_k^z \rangle$ od razdalje med $j$ in $k$. 
To odvisnost za $j=1$ prikazuje slika \ref{fig:spin-1k}. 

\begin{figure}[H]
\centering
\input{g_korelacija}
\caption{Spinska korelacija delcev na razli"cnih razdaljah v osnovnem stanju}
\label{fig:spin-1k}
\end{figure}

Ker sem uporabil odprte robne pogoje, stanje ni nujno odvisno le od razdalje med $j$ in $k$, ampak je lahko odvisno tudi od njunega absolutnega polo"zaja. 
Zato sem dodatno ra"cunal odvisnost korelacije $\langle \sigma_j^z \sigma_{j+1}^z$ od $j$. 

\begin{figure}[H]
\centering
\input{g_korelacija_soseda}
\caption{Spinska korelacija dveh sosednjih delcev v osnovnem stanju}
\label{fig:spin-jp}
\end{figure}
\end{document}
