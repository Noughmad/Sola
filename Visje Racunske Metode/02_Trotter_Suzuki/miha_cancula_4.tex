\documentclass[a4paper,10pt]{article}
\usepackage[utf8]{inputenc}
\usepackage[slovene]{babel}
\usepackage{graphicx}
\usepackage{hyperref}
\usepackage[left=2cm,right=2cm,top=2cm,bottom=3cm]{geometry}
\usepackage{amsmath}
\usepackage{float}
\usepackage{bbold}

\makeatletter
\renewcommand*\env@matrix[1][*\c@MaxMatrixCols c]{%
  \hskip -\arraycolsep
  \let\@ifnextchar\new@ifnextchar
  \array{#1}}
\makeatother

%opening
\title{Trotter-Suzukijev razcep, kubitne verige}
\author{Miha \v Can\v cula}

\newcommand{\uvec}[1]{\ensuremath{\underline{#1}}}
\newcommand{\bb}{
  \ensuremath{b_1 b_2 \ldots b_n}
}
\newcommand{\psibb}{
  \ensuremath{\psi_{\bb}}
}

\begin{document}

\maketitle

\section{Ozna"cevanje stanj}

Ra"cune sem opravljal v binarni bazi, katere lastna stanja so $|\uvec{b}\rangle = |b_1 b_2 \ldots b_n\rangle$, $b_i \in {0,1}$. 
Ta baza je $2^n$-dimenzionalna, torej vsako stanje opi"semo z $N = 2^n$ koeficienti $\psi_{\uvec{b}}$. 

Za numeri"cno ra"cunanje stanja predstavimo z vektorji, operatorje pa z matrikami. 
Bazne vektorje je koristno urediti tako, da bo ra"cunanje z njimi "cim la"zje in "cim bolj u"cinkovito. 
To velja predvsem za konstrukcijo matrik operatorjev. 
Ker so matrike velike ($2^n \times 2^n$), je "ze sama njihova konstrukcija zamudna, "ce vsak element nastavimo posebej. 
V tej nalogi bomo uporabljali predvsem dvodel"cne operatorje, ki delujejo na dva sosednja delca $j$ in $j+1$. 
Propagator $U^{2}$ je "ze tak"sen, Hamiltonov operator pa lahko zapi"semo kot vsoto dvodel"cnih operatorjev. 

Izbral sem ``binarno'' ureditev stanj, torej tak"sno, kjer so bazna stanja urejena po vrednosti $\overline{b_1 b_2 \ldots b_n}$, "ce jih zapi"semo v obratnem vrstnem redu in interpretiramo kot dvoji"ska "stevila. 
Prekomernim zankam sem se izognil tako, da sem dvodel"cni operator prevedel na matriko $N\times N$. 
Indeks delca $j$ je dolo"cen tako, da ima stanje, kjer ima le prvi delec spin 1, indeks stanja $1$.

\begin{table}[H]
\centering
 \begin{tabular}{r|l}
  $J$ & $b_4 b_3 b_2 b_1$ \\
  \hline
  0 & 0000 \\
  1 & 0001 \\
  2 & 0010 \\
  3 & 0011 \\
  4 & 0100 \\
  $\vdots$ & $\vdots$ \\
  14 & 1110 \\
  15 & 1111
 \end{tabular}
\end{table}



Dvodel"cni operator $U^{(j)}$ me"sa le koeficiente $\psibb$, katerih indeksi $b_i$ se razlikujejo kve"cjemu na mestih $j$ ali $j+1$.
Ker $b_j$ in $b_{j+1}$ lahko zavzameta "stiri razli"cne vrednosti, so tak"sni koeficienti vedno v skupinah po 4. 
V primeru $j=1$ in gornjega o"stevil"cenja je prevod na matriko enostaven: 
V isti skupini so vedno "stirje zaporedni koeficienti, zato je celotna matrika blo"cno diagonalna, kjer je vsak blok $4\times 4$ matrika $U$. 

\begin{equation}
 \begin{bmatrix}
  U_{11} & U_{12} & U_{13} & U_{14} \\
  U_{21} & U_{22} & U_{23} & U_{24} \\
  U_{31} & U_{32} & U_{33} & U_{34} \\
  U_{41} & U_{42} & U_{43} & U_{44}
  \end{bmatrix} \Rightarrow \begin{bmatrix}
  U_{11} & U_{12} & U_{13} & U_{14} & & & \\
  U_{21} & U_{22} & U_{23} & U_{24} & & &\\
  U_{31} & U_{32} & U_{33} & U_{34} & & &  \\
  U_{41} & U_{42} & U_{43} & U_{44} & & & \\
  & & & & U_{11} & U_{12} & U_{13} & U_{14} \\
  & & & & U_{21} & U_{22} & U_{23} & U_{24} \\
  & & & & U_{31} & U_{32} & U_{33} & U_{34} \\
  & & & & U_{41} & U_{42} & U_{43} & U_{44}
  \end{bmatrix}
\end{equation}

Za $j>1$ je razmislek malo bolj zapleten. Operator $U^{(2)}$ sklaplja stanja, kjer so enaki vsi indeksi razen $b_2$ in $b_3$. Ena izmed tak"snih skupin je $(0, 2, 4, 6)$.
Osnovno matriko za $U$ moramo torej ``napihniti'' na dimenzijo $8\times 8$, brez da bi delovala na ostale koeficiente. 
"Ce upo"stevamo, da na enak na"cin sklaplja tudi stanja $(1, 3, 5, 7)$, vidimo, da mora biti ``napihnjena'' matrika enaka matriki za $U$, kjer vsak element $U_{ab}$ nadomestimo z matriko $U_{ab}\cdot \mathbb{1}_2$. 
To je ekvivalentno Kroneckerjevemu direktnemu produktu $U \otimes \mathbb{1}_2$. 
Stanja z vi"sjimi indeksi $J$ upo"stevamo enako kot prej, tako da bloke $U \otimes \mathbb{1}_2$ zlo"zimo po diagonali matrike. 

\begin{equation}
 \begin{bmatrix}
  U_{11} & U_{12} & U_{13} & U_{14} \\
  U_{21} & U_{22} & U_{23} & U_{24} \\
  U_{31} & U_{32} & U_{33} & U_{34} \\
  U_{41} & U_{42} & U_{43} & U_{44}
  \end{bmatrix} \Rightarrow \begin{bmatrix}
  U_{11} & & U_{12} & & U_{13} & & U_{14} & \\
  & U_{11} & & U_{12} & & U_{13} & & U_{14} \\
  U_{21} & & U_{22} & & U_{23} & & U_{24} & \\
  & U_{21} & & U_{22} & & U_{23} & & U_{24} \\
  U_{31} & & U_{32} & & U_{33} & & U_{34} &\\
  & U_{31} & & U_{32} & & U_{33} & & U_{34} \\
  U_{41} & & U_{42} & & U_{43} & & U_{44} & \\
  & U_{41} & & U_{42} & & U_{43} & & U_{44}
  \end{bmatrix}
\end{equation}

Dvodel"cni operator na poljubnem delcu $j$ lahko predstavimo z matriko $N\times N$, kjer najprej ustvarimo blok $U \otimes \mathbb{1}_{2^{j-1}}$ dimenzije $4\cdot 2^{j-1}\times 4\cdot 2^{j-1}$. Iz $2^{n-j-1}$ tak"snih blokov nato zlo"zimo blo"cno diagonalno matriko. 

Na te"zave naletimo le pri upo"stevanju periodi"cnih robnih pogojev. Operator $U^{(n)}$ namre"c sklaplja koeficiente, kjer se lahko razlikujeta le $b_n$ in $b_1$. Primera tak"snih skupin sta $(0, 1, 8, 9)$ in $(6, 7, 14, 15)$. Torakt matrike na ``napihujemo'' enakomerno, ampak jo razdelimo na "stiri bloke $2\times 2$ in te "stiri bloke postavimo na ustrezna mesta, da sklapljajo koeficiente $(0, 1, 2^{j-1}, 2^{j-1}+1)$. Ostale skupine "stirih koeficientov dobimo tako, da vsem "stirim indeksom pri"stejemo poljuben ve"ckratnik 2. Celotno matriko $N\times N$ dobimo tako, da ``napihnjeno'' matriko direktno mno"zimo z diagonalno matriko $2^{j-2}\times 2^{j-2}$, ki ima na diagonali enke na lihih mestih in ni"cle na sodih. 

\begin{equation}
 \begin{bmatrix}[cc|cc]
  U_{11} & U_{12} & U_{13} & U_{14} \\
  U_{21} & U_{22} & U_{23} & U_{24} \\
  \hline
  U_{31} & U_{32} & U_{33} & U_{34} \\
  U_{41} & U_{42} & U_{43} & U_{44}
  \end{bmatrix} = \begin{bmatrix}[c|c]
  A & B \\
  \hline
  C & D
 \end{bmatrix} \Rightarrow \begin{bmatrix}
  A & \cdots & B \\
  \vdots & \ddots & \vdots \\
  C & \cdots & D
  \end{bmatrix} \Rightarrow \begin{bmatrix}
  A & & & B & & \\
  & A & & & B & \\
  & & \ddots & & & \ddots \\
  C & & & D & & \\
  & C & & & D & \\
  & & \ddots & & & \ddots \\
  \end{bmatrix}
\end{equation}

Celotni propagator je produkt vseh dvodel"cnih propagatorjev. Ker ti med seboj ne komutirajo, smo produkt razcepili na produkta sodih in lihih "clenov. 

\begin{align}
\label{eq:produkti}
 U_N^{(lih)} = \prod_{j=1}^{n-1} U_N^{(j)}, \qquad U_N^{(sod)} = \prod_{j=2}^{n} U_N^{(j)} 
\end{align}

"Cleni znotraj vsakega izmed zgornjih produktov med seboj komutirajo, zato vrstni red mno"zenja ni pomemben. 

Na enak na"cin skonstruiramo tudi matriko Hamiltonovega operatorja, tako da namesto $U$ za za"cetno matriko uporabimo $h = \vec{\sigma} \stackrel{\cdot}{\otimes} \vec{\sigma}$. 
Namesto produkta pa tam nastopa vsota, zato ne potrebujemo razcepa na sode in lihe "clene. 

\begin{align}
 H_N &= \sum_{j=1}^{n} h_N^{(j)}
\end{align}

Ko enkrat izberemo velikost koraka Trotter-Suzukijev razcep, potrebujemo le majhno "stevilo razli"cnih vrednosti $z$. 
Matrike propagatorja lahko zato izra"cunamo vnaprej, "casovni razvoj posameznega stanja pa izra"cunamo kar z matri"cnim mno"zenjem

\begin{align}
 |\psi'\rangle &= U_N(z) |\psi\rangle
\end{align}

Matrike $U_N$ so redke, zato celoten postopek vzame manj kot $\mathcal{O}(N^2)$ "casa. 
V praksi se produktov iz ena"cbe (\ref{eq:produkti}) ne spla"ca ra"cunati vnaprej, saj s tem izgubimo redkost matrik. 
"Ce namesto tega $|\psi\rangle$ mno"zimo z vsako matriko posebej, porabimo za to ravno $\mathcal{O}(n2^n) = \mathcal{O}(N\log N)$ operacij. 
Vsaka matrika ima najve"c $4N$ neni"celnih elementov, vseh matrik pa je $n$. 
V na"sem primeru, ko je tudi sama matrika $U$ redka, je neni"celnih elementov "se manj. 

\section{"Casovna zahtevnost}

Ra"cunanje sem implementiral v programu \texttt{GNU Octave}, ki je posebej primeren za enostavno operiranje z redkimi matrikami. 
Matrike $N \times N$ sem skonstruiral po zgornjem postopku, tako da sem ve"ckrat uporabil direktni produkt matrik namesto zank v programu. 
Pri velikosti problema $n=16$ in $n=18$ je program za konstrukcijo matrik porabil pribli"zno toliko "casa, kot za dva koraka propagacije s shemo $S_4$, torej pribli"zno 15 mno"zenj s celotnim propagatorjem. Ve"cjih $n$ nisem mogel dose"ci, ker je ra"cunalniku zmanjkalo pomnilnika. 

\begin{figure}[h]
\centering
\input{g_casovna_zahtevnost}
 \caption{Trajanje ra"cunanja v odvisnosti od velikosti problema $n$. Podatki se dobro prilegajo krivulji $A + B n 2^n$. }
\end{figure}

Trajanje ra"cunanja se ujema s pri"cakovano odvisnostjo $\mathcal{O}(N \log N)$. 
Odstopanje je le pri majhnih $n$, kjer moramo upo"stevati "se majhnen konstantni "clen za ``overhead''.

Na grafu vidimo tudi razliko med pristopom, ko sode in lihe napihnjene dvodel"cne propagatorje vnaprej zmo"zimo in tistim, ko stanje mno"zimo z vsakim posebej. 
Velike razlike v trajanju ra"cuna ne opazimo, je pa predhodno mno"zenje ugodno predvsem pri majhnem "stevilu delcev, saj imamo manj knjigovodstva. 
Pri ve"cjem "stevilu delcev pa se ve"ckratno mno"zenje izka"ze za hitrej"se, saj pri tem bolje izkoristimo redkost matrik. 

\section{Termalno povpre"cje}

Najprej sem z zgornjo metodo ra"cunal pri"cakovane vrednosti proste energije in energije. 
Za to uporabimo realen $z$, ki je povezan s temperaturo prek zveze

\begin{align}
 z &= \frac{\beta}{2} = \frac{1}{2k_B T}
\end{align}

Fazna vsota $Z$ je enaka pri"cakovani normi vektorja $|\psi\rangle$, po tem ko smo ga propagirali z $U_N(z)$. Pri"cakovana energija $\langle H \rangle_\beta$ pa je enaka povpre"cju matri"cnega elementa $\langle \psi | H | \psi\rangle$, spet po propagaciji z $U_N(z)$. 

Obe povpre"cji sem izra"cunal tako, da sem vzel $N_\psi$ naklju"cnih za"cetnih vektorjev, jih propagiral in nato izra"cunal "zeljeno vrednost. 
S tem dobimo tako pri"cakovano vrednost kot tudi statistiko napak. 
Vsaki"c sem $z$ razdelil na ve"cje "stevilo manj"sih korakov, ki so bili v brezdimenzijskih enotah veliki najve"c $10^{-3}$. 
Velikosti koraka nisem prilagajal glede na vrednosti $z$, saj naj zanimajo "cimbolj natan"cne pri"cakovane vrednosti v nizkotemperaturni limiti, torej pri velikih $\beta$. 

\begin{figure}[H]
 \centering
 \input{g_energija_obe}
 \caption{Odvisnost proste energije $F$ in povpre"cne energije $\langle H\rangle$ od inverzne temperature $\beta$}
\end{figure}

Obe termodinamski spremenljivki se s temperaturo spreminjata kot

\begin{align}
 \{F,\langle H \rangle \} &= -A + \frac{B}{1 + C\beta}
\end{align}

Konstanti $B$ in $C$ sta razli"cni za vsako izmed spremenljivk, konstanta $A$ pa je v obeh primerih enaka "stevilu delcev $A = n$. 
To je razumljivo, saj je v nizkotemperaturni limiti, ko gre $\beta \to \infty$, sistem vedno v osnovnem stanju. 
Osnovno stanje antiferomagnetnega sistema s sodim "stevilom delcev je tak"sno, da imata vsaka dva soseda nasprotna spina, pri izbrani normalizaciji pa ima tak"sno stanje energijo $-n$. 
Pri nizkih temperaturi gre tudi razlika $E-F = TS$, ki je sorazmerna s temperaturo, proti ni"c, kar vidimo tudi na zgornjem grafu. 

\section{"Casovna korelacija}

Za simularanje "casovnega razvoja pri neskon"cni temperaturi uporabimo imaginaren $z$, ki je po kvantni mehaniki enak $z = it$. 

\subsection{Magnetizacija}
Matrika enodel"cnega operatorja $\sigma_1^z$ je posebej enostavna v binarni bazi, saj bazna stanja s sodim $J$ mno"zi z 1, stanja z lihim $J$ pa z $-1$. 
Matrika $X$ je torej diagonalna, po diagonali pa se izmenjujejo $1$ in $-1$. 

\begin{figure}[H]
\centering
 \input{g_korelacija_mag}
 \caption{"Casovna korelacija enodel"cne magnetizacije $\sigma_1^z$}
\end{figure}

Opazimo, da korelacija mo"cno niha. Model je antiferomagneten, torej pri"cakujemo tudi negativno korelacijo. 
Nihanje o"citno ni "cisto sinusno, vseeno pa izgleda kot vsota ve"cih sinusov. 
Da bi to preveril sem naredil "se Fourierovo transformacijo avtokorelacijske funkcije, ki ji enaka kvadratu absolutne vrednosti Fourierove transformiranke magnetizacije. 

\begin{figure}[H]
\centering
 \input{g_fft_mag}
 \caption{Fourierova transformiranka "casovne korelacije enodel"cne magnetizacije $\sigma_1^z$}
\end{figure}

Vidimo "stiri jasno vrhove pri pribli"zno enakomerno razmaknjenih frekvencah $\omega = k \omega_0$. 
Ostrina vrhov nam pove, da je korelacijska funkcija v dobrem pribli"zku periodi"cna in le po"casi pada proti 0. 

\subsection{Spinski tok}

Pri obravnavi spinskega toka $J$ si pomagamo z dejstvom, da je operator $J$ vsota dvodel"cnih operatorjev $J_j = \sigma_j^x \sigma_{j+1}^y - \sigma_j^y \sigma_{j+1}^x$. 
Za"cnemo torej z matriko $J_0 = \sigma^x \otimes \sigma^y - \sigma^y \otimes \sigma^x$ dimenzije $4\times 4$, ki jo po zgoraj opisanem postopku ``napihnemo'' v matriko $N\times N$. 

\begin{figure}[H]
\centering
 \input{g_korelacija_tok}
 \caption{"Casovna korelacija spinskega toka $J$}
\end{figure}

V primerjavi z magnetizacijo pa avtokorelacijska funkcija spinskega toka pada vidno hitreje. 
Za majhne "case jo lahko pribli"zamo z izrazom z du"senim nihanjem, $C(t) \propto \cos(\omega t) \exp(-t/\tau)$. 

\begin{figure}[H]
\centering
 \input{g_fft_tok}
 \caption{Fourierova transformiranka "casovne korelacije spinskega toka $J$}
\end{figure}

Na Fourierovi transformiranki je viden en sam vrh, ki je manj oster. Po dalj"sem "casu ($t\geq 10$) pa nihanje ne zamre popolnoma, ampak utripa, kar je bolje vidno na logaritemskem grafu spodaj. 

\begin{figure}[H]
 \centering
\input{g_tok_log}
\caption{Logaritemski graf korelacije spinskega toka v dalj"sem "casovnem intervalu}
\end{figure}

Tudi na logaritemskem grafu vidimo najprej hitro padanje ($\exp(-t/\tau)$ s $\tau\approx 5/3$), ki pa se ustavi pri amplitudi okrog $0.01$. 
"Ce ta del zanemarimo, lahko difuzijsko konstanto izra"cunamo kot

\begin{align}
 D &= \int_0^{\infty} Ae^{-t/\tau} \; \mathrm{d}t = A\tau \approx \frac{3}{2} \cdot \frac{5}{3} = \frac{5}{2}
\end{align}

Z upo"stevanjem, da korelacija ne pade "cisto na ni"c, ampak se tudi po dolgem "casu ustali pri neki majhni a kon"cni vrednosti, zgornji integral divergira in je difuzijska konstanta neskon"cna. 

\end{document}
