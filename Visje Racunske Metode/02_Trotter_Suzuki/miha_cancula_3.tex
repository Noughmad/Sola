\documentclass[a4paper,10pt]{article}
\usepackage[utf8]{inputenc}
\usepackage[slovene]{babel}
\usepackage{graphicx}
\usepackage{hyperref}
\usepackage[left=3cm,right=3cm,top=3cm,bottom=3cm]{geometry}
\usepackage{amsmath}
\usepackage{float}

%opening
\title{Trotter-Suzukijev razcep, simplekti\v cna integracija}
\author{Miha \v Can\v cula}

\begin{document}

\maketitle

\section{Ohranitev energije}

Implementiral sem integrator, ki za "casovni korak uporablja Trotter-Suzukijev razcep. 
Izbral sem dve shemi, in sicer simetri"cni $S_2$ in $S_4$. 
Za primerjavo sem dodal "se znano metodo \texttt{RK4}. 
V preizkusu sem za vse tri metode nastavil enako in nespremenljivo dol"zino koraka. 

\input{g_energija_0.tex} \\
\input{g_energija_01.tex} \\
\input{g_energija_1.tex}

\section{Ekviparticijski izrek}



\end{document}
