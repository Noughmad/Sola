\documentclass[a4paper,10pt]{article}
\usepackage[utf8]{inputenc}
\usepackage[slovene]{babel}
\usepackage{graphicx}
\usepackage{hyperref}
\usepackage[left=2cm,right=2cm,top=2cm,bottom=3cm]{geometry}
\usepackage{amsmath}
\usepackage{float}

%opening
\title{Trotter-Suzukijev razcep, simplekti\v cna integracija}
\author{Miha \v Can\v cula}

\begin{document}

\maketitle

\section{Ohranitev energije}

Implementiral sem integrator, ki za "casovni korak uporablja Trotter-Suzukijev razcep. 
Izbral sem dve shemi, in sicer simetri"cni $S_2$ in $S_4$. 
Za primerjavo sem dodal "se znano metodo \texttt{RK4}. 
V preizkusu sem za vse tri metode nastavil enako in nespremenljivo dol"zino koraka. 

\begin{figure}[H]
\input{g_energija_0.tex}
\caption{Ohranitev energije v sistemu z $\lambda = 0$}
\label{fig:ohranitev-lambda-0}
\end{figure}

\begin{figure}[H]
\input{g_energija_01.tex}
\caption{Ohranitev energije v sistemu z $\lambda = 0.1$}
\label{fig:ohranitev-lambda-01}
\end{figure}

\begin{figure}[H]
\input{g_energija_1.tex}
\caption{Ohranitev energije v sistemu z $\lambda = 1$}
\label{fig:ohranitev-lambda-1}
\end{figure}

Energijo najbolje ohranja metoda s simplekti"cnim integratorjem s shemo $S_4$. 
Pri uporabljeni ra"cunalni"ski natan"cnosti sploh ne opazimo odstopanj od za"cetne vrednosti. 
Pri ostalih dveh metodah se energija spreminja s "casom, je pa med njima pomembna razlika. 
S simplekti"cno metodo s shemo $S_2$ energija le niha, medtem ko ne njena povpre"cna vrednost ne spreminja. 
Pri metodi Runge-Kutta pa vidimo jasen trend, ki je "se izrazitej"si pri ve"cjih vrednostih $\lambda$. 

Simplekti"cne metode poka"zejo veliko prednost pred metodo Runge-Kutta pri dolgih "casih integriranja in mo"cnim odstopanjem od harmonskega potenciala. 

\section{Ekviparticijski izrek}

Preverjal sem tudi veljavnost ekviparticijskega izreka. Do ekviparticije naj bi pri"slo, "ce je sistem ergodi"cen oz. kaoti"cen. 
V primeru dvodimenzionalnega anharmonskega oscilatorja ekviparticija pomeni, da je pribli"zno enako energije v nihanji v $x$ smeri kot v $y$ smeri, seveda le v "casovnem povpre"cju po dolgem "casu. 
Izbral sem za"cetni pogoj, kjer to ne dr"zi, in opazoval, po kolik"snem "casu se bo vzpostavila enakost energij.

Kot kvantitativno merilo za ekviparticijo sem uporabil izraz v navodilih
\begin{align}
 \langle p_i^2\rangle &= \frac{1}{T} \int_0^T p_i^2(t) \; \mathrm{d}t
\end{align}

V primeru, da je sistem kaoti"cen, se morata izraza $\langle p_x^2\rangle$ in $\langle p_y^2\rangle$ po dovolj dolgem "casu izena"citi. 

\begin{figure}[H]
\input{g_ekviparticija_1.tex}
\caption{Ekviparticija v sistemu z $\lambda = 1$}
\label{fig:eq-lambda-1}
\end{figure}

\begin{figure}[H]
\input{g_ekviparticija_low.tex}
\caption{Ekviparticija v sistemu z $\lambda = 1.16$}
\label{fig:eq-lambda-low}
\end{figure}

\begin{figure}[H]
\input{g_ekviparticija_mid.tex}
\caption{Ekviparticija v sistemu z $\lambda = 1.18$}
\label{fig:eq-lambda-mid}
\end{figure}

\begin{figure}[H]
\input{g_ekviparticija_high.tex}
\caption{Ekviparticija v sistemu z $\lambda = 1.20$}
\label{fig:eq-lambda-high}
\end{figure}

\begin{figure}[H]
\input{g_ekviparticija_2.tex}
\caption{Ekviparticija v sistemu z $\lambda = 2$}
\label{fig:eq-lambda-2}
\end{figure}

\begin{figure}[H]
\input{g_ekviparticija_10.tex}
\caption{Ekviparticija v sistemu z $\lambda = 10$}
\label{fig:eq-lambda-10}
\end{figure}


Za la"zjo primerjavo sem za"cetni pogoj vedno postavil na $(x, y, p_x, p_y) = (0, 1/2, 1, 0)$. S tak"sno izbiro dobimo ostro spremembo dinamike v bli"zini $\lambda_c = 1.18$. Pod to mejo (sliki \ref{fig:eq-lambda-1} in \ref{fig:eq-lambda-low}) vrednosti $\langle p_i^2 \rangle$ sicer oscilirata, ampak njuni dolgoro"cni povpre"cji konvergirata k razli"cnima vrednostma. Ko vrednost $\lambda$ prese"ze mejo, postane dinamika kaoti"cna, obe povpre"cji "se mo"cneje oscilirata, vendar sedaj obe povpre"cji oscilirata okrog iste vrednosti (sliki \ref{fig:eq-lambda-2} in \ref{fig:eq-lambda-10}). V tem primeru lahko re"cemo, da imamo ekviparticijo, saj je v vsakem izmed 1D nihanj enako energije. V bli"zini meje, torej okrog $\lambda = 1.2$ (sliki \ref{fig:eq-lambda-mid} in \ref{fig:eq-lambda-high}), opazimo nekako za"casno ekviparticijo. Nekaj "casa se sistem obna"sa integrabilno, tako da sta povpre"cji mo"cno razli"cni, nekaj "casa pa kaoti"cno. Z nadaljnjim zvi"sevanjem $\lambda$ je sistem vse bolj kaoti"cen. 

S spreminjanjem za"cetnega pogoja se spremeni tudi mejna vrednost $\lambda_c$. Pri pove"cevanju $p_x$ ta meja pada, tako da z za"cetnim pogojem $p_x=2$ pade pod 1. Iz ostrine prehoda med nekaoti"cno in kaoti"cno dinamiko, ter iz odvisnosti prehoda od za"cetnega pogoja sklepam, da je v vsakem primeru del faznega prostora regularen, del pa kaoti"cen. Ko sistem enkrat zaide v kaoti"cni del, pride do ekviparticije. V bli"zine meje se lahko zgodi, da sistem za"casno preide iz kaoti"cnega dela nazaj v integrabilni del, kjer se povpre"cji razlikujeta. Po pri"cakovanje je ob ve"canju parametra $\lambda$ dele"z kaoti"cnega faznega prostora vedno ve"cji, in na neki to"cki ta dele"z zajame na"s za"cetni pogoj.  

\end{document}
