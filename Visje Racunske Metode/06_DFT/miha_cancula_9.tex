\documentclass[a4paper,10pt]{article}
\usepackage[utf8]{inputenc}
\usepackage[slovene]{babel}
\usepackage{graphicx}
\usepackage{hyperref}
\usepackage[left=2cm,right=2cm,top=2cm,bottom=3cm]{geometry}
\usepackage{amsmath}
\usepackage{float}
\usepackage{bbold}
\usepackage{amsmath}
\usepackage{amssymb}

\makeatletter
\renewcommand*\env@matrix[1][*\c@MaxMatrixCols c]{%
  \hskip -\arraycolsep
  \let\@ifnextchar\new@ifnextchar
  \array{#1}}
\makeatother

%opening
  
\title{Problem elektronske strukture: metode povprečnega polja in gostotnih funkcionalov}
\author{Miha \v Can\v cula}

\newcommand{\uvec}[1]{\ensuremath{\underline{#1}}}
\newcommand{\bb}{
  \ensuremath{b_1 b_2 \ldots b_n}
}
\newcommand{\psibb}{
  \ensuremath{\psi_{\bb}}
}

\begin{document}

\maketitle

\section{Algoritem}

Metoda \textsc{dft} je sestavljena iz hkratnega re"sevanje dveh ena"cb. 
Prva je stacionarna Schr\"odingerjeva ena"cba, ki podaja valovno funkcijo delca v dolo"cene potencialu. 
Druga ena"cba je Poissonova, ki ob dani gostoti naboja poda elektri"cni potencial.
"Ce so obravnavani delci nabiti, kot na primer elektroni, sta ena"cbi sklopljeni. 

Ena"cbi re"sujemo iterativno, tako od ob vsakem koraku najprej najdemo valovno funkcijo, ki ustreza trenutnemu pribli"zku za potencial. 
Iz valovne funkcijo z re"sevanjem Poissonove ena"cbe izra"cunamo elektri"cni potencial. 

Potencial jedra je sferi"cno simetri"cna, zato so tak"sna tudi osnovna elektronska stanja. 
Valovna funkcija $\psi(\vec r)$ in elektri"cni potencial $V(\vec r)$ sta tako le funkciji oddaljenosti od jedra $r$. 
Za la"zje ra"cunanje uporabimo "se substituciji
\begin{align}
 \psi(r) &= u(r)/r \\
 V(r) &= NU(r)/r
\end{align}
kjer je $N$ naboj jedra in je enak "stevilu elektronov.
S tak"sno substitucijo poenostavimo ena"cbe in se znebimo divergence potenciala v izhodi"s"cu. 

\section{Vodikov atom}

Vodikov atom lahko slu"zi za preverjanje posameznega re"sevanja vsake izmed obeh ena"cbe. 
Edini elektron v vodikovem atomu namre"c ne interagira sam s sabo, ampak se giblje le v potencialu jedra, ki je znan in konstanten. 

Za re"sevanje Schr\"odingerjeve ena"cbe sem izbral strelsko metodo z bisekcijo, za numeri"cno integracijo pa metodo Numerova. 
Ta kombinacija se je izkazala za primerno v prvi nalogo pri tem predmetu. 
Metoda posku"sa razli"cne vrednosti za energijo elektronskega stanja, dokler ne zadosti robnemu pogoju, da je elektron vezan, torej da gre $u(r)$ za velike $r$ proti 0. 

Za re"sitev problema vodikovega atoma moramo metodo pognati le enkrat, saj se potencial ne spreminja. 
Na ta na"cin lahko preverimo pravilnost re"sevanja Schr\"odingerjeve ena"cbe. 

\begin{figure}[H]
\input{g_test_vodik}
 \caption{Osnovno stanje elektrona v vodikovem atomu}
 \label{fig:vodik-sch}
\end{figure}

Ko imamo podano osnovno stanje elektrona v vodikovem atomu lahko izra"cunamo tudi elektri"cni potencial elektrona. 
Tak"sen potencial bi "cutil testni naboj, "ce bi ga pribli"zali atomu. 
Ker je valovna funkcija in s tem gostota naboja sferi"cno simetri"cna, je tak"sen tudi potencial, zato lahko opazujemo le odvisnost $V(r)$. 

\begin{figure}[H]
\input{g_test_vodik_pot}
 \caption{Elektri"cni potencial elektrona v vodikovem atomu}
 \label{fig:vodik-potencial}
\end{figure}

Tako valovna funkcija kot potencial elektrona v vodikovem atomu se odli"cno ujemata s teoreti"cno napovedjo. 
Metoda izra"cuna tudi pravilno vrednost za energijo osnovnega stanja, ki je enaka $-1/2$. 

\section{Helijev atom}

V helijevem atomu sta dva elektrona, ki interagirata med seboj. 
Za poenostavitev lahko predpostavimo, da sta krajevna dela valovne funkcije enaka za oba elektrona. 
Schr\"odingerjevo ena"cbo moramo tako re"sevati le enkrat na vsakem koraku. 

Vsak elektron "cuti elektri"cni potencial jedra in potencial drugega elektrona. 
S privzetkom povpre"cnega polja, da sta valovni funkciji in s tem gostoti naboja obeh elektronov enaki, sta enaka tudi potenciala obeh elektronov. 
Zato lahko tudi Poissonovo ena"cbe re"sujemo le enkrat na vsakem koraku. 

Metoda povpre"cnega polja zanemari korelacije med elektronoma. 
Pri izra"cunu potenciala, ki ga "cuti vsak izmed elektronov, zato dodamo dodaten "clen, ki opi"se te korelacije po Kohn-Shamovem modelu. 
Celoten potencial sedaj sestavljajo trije prispevki:

\begin{enumerate}
 \item Potencial jedra, $V_j = -\frac{2}{r}$
 \item Potencial drugega elektron, $V_{HF} = \frac{2U(r)}{r}$
 \item Izmenjalno-korelacijski potencial, ki opi"se kvante korelacije med elektronoma, $V_{xc} = -\left( \frac{3u^2(r)}{2\pi^2r^2} \right)^{1/3}$. 
\end{enumerate}

Prvi prispevek je konstanten, drugi prispevek izra"cunamo z re"sevanjem Poissonove ena"cbe in poznavanjem valovne funkcije $u(r)$, tretjega pa neposredno iz valovne funkcije. 
Vsaka iteracija metode tako vsebuje tri dele:

\begin{enumerate}
 \item Re"sevanje Poissonove ena"cbe, dobimo potencialu $V_{HF}$. 
 \item Potencialu pri"stejemo prispevek jedra in korelacij.
 \item Re"sevanje Schr\"odingerjeve ena"cbe, dobimo nov pribli"zek za valovno funkcijo elektronov. 
\end{enumerate}

Iteracijo nadaljujemo, dokler se energija lastnega stanja ne neha spreminjati. 
Kon"cno stanje je prikazano na spodnjih slikah. 

\begin{figure}[H]
\input{g_helij_e}
 \caption{Osnovno stanje elektrona v helijevem atomu}
 \label{fig:helij-sch}
\end{figure}

Osnovno stanje elektrona v helijevem atomu je bolj lokalizirano okrog jedra kot v vodikovem atomu. 
To je konsistentno s podatki\cite{wiki:radius}. 
Iz podane valovne funkcije lahko izra"cunamo energijo osnovnega stanja, ki pa ni kar dvakratnik energije posameznega elektrona. 
V izbranih brezdimenzijskih enotah je izra"cunana energija posameznega elektrona enaka $\varepsilon = -0,\!516809$, skupna energija pa $E = -2,\!72289$. 

\begin{figure}[H]
\input{g_helij_pot}
 \caption{Elektri"cni potencial elektrona v helijevem atomu}
 \label{fig:vodik-potencial}
\end{figure}

Podobno opazimo tudi pri elektri"cnem potencilu, ki se zasiti pri manj"si oddaljenosti od jedra kot pri vodiku. 

\begin{thebibliography}{6}
 \bibitem{wiki:radius} Atomic radius -- Wikipedia \\ \url{http://en.wikipedia.org/wiki/Atomic_radius#Calculated_atomic_radii}
\end{thebibliography}


\end{document}

