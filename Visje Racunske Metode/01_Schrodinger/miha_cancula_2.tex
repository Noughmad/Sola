\documentclass[a4paper,10pt]{article}
\usepackage[utf8]{inputenc}
\usepackage[slovene]{babel}
\usepackage{graphicx}
\usepackage{hyperref}
\usepackage[left=3cm,right=3cm,top=3cm,bottom=3cm]{geometry}
\usepackage{amsmath}
\usepackage{float}

%opening
\title{Schr\"odingerjeva ena\v cba, stacionaren problem}
\author{Miha \v Can\v cula}

\begin{document}

\maketitle

\section{Matrika hamiltoniana}

Za izra"cun lastnih energij sistema s hamiltonovim operatorjem $\hat H$ potrebujemo matri"cne elemente $H_{ij} = \langle\psi_i|H|\psi_j\rangle$ v neki ortonormirani bazi $|\psi_i\rangle$. Logi"cna izbira za na"s problem so lastna stanja harmonskega oscilatorja $|n\rangle$. Matri"cne elemente bi lahko izra"cunali neposredno iz lastnih stanja in hamiltonove matrike, dobljene z diskretizacijo prostora. Te"zava se pojavi, ko nas zanimajo vi"sja vzbujena stanja. Pri rekurzivnem ali eksplicitnem ra"cunanju Hermitovih polinomov $H_n$ za velike $n$ (ve"cje od 50) namre"c hitro pridemo do odstevanja velikih "stevil, kar zaradi kon"cne ra"cunalni"ske natan"cnosti privede do velikih napak. Dodatno ta stanja zelo hitro nihajo v prostoru, zato moramo za dober opis uporabiti zelo fino diskretizacjo. 

Obema te"zavama se lahko v na"sem primeru izognemo, "ce matri"cne elemente izrazimo analiti"cno. Vemo, da se hamiltonov operator za harmonski oscilator v bazi lastnih stanj $|n\rangle$ glasi
\begin{align}
  \hat H_0 &= \frac{p^2}{2} + \frac{x^2}{2} = a^\dagger a + \frac{1}{2}
\end{align}
Drugi del potenciala, $x^4$, tudi lahko izrazimo s kreacijskim in anihilacijskim operatorjem
\begin{align}
  x = & {} \frac{1}{\sqrt{2}} (a^\dagger + a) \\
  x^4 = & {} \frac{1}{4} (a^\dagger + a)^4 = \frac{1}{4} \left({a^\dagger}^2 + 1 + 2n + a^2\right)^2 \\
      = & {} \frac{1}{4} \Big( {a^\dagger}^4 + 1 + 4n^2 + a^4 + 2{a^\dagger}^2 + 4n + 2a^2 + \\
      &+ 2{a^\dagger}^2n + 2n{a^\dagger}^2 + 2{a}^2n + 2n{a}^2 + {a^\dagger}^2a^2 + a^2{a^\dagger}^2\Big)
\end{align}
kjer je $n=a^\dagger a$, oznake za operatorje pa sem spu"s"cal. Matri"cne elemente kreacijskega in anihilacijskega operatorja poznamo
\begin{align}
  \langle m|\hat a^\dagger|n\rangle &= \sqrt{n+1}\,\delta_{m,n+1}\\
  \langle m|\hat a|n\rangle &= \sqrt{n}\,\delta_{m,n-1} \\ 
  \langle m|\hat n|n\rangle &= n\,\delta_{m,n} 
\end{align}
"Ce to vstavimo v izraz za $x$ in zdru"zimo "clene z enakimi $\delta_{ij}$, dobimo rezultat
\begin{align}
  4\langle m|\hat x^4|n\rangle = & {} \sqrt{n(n-1)(n-2)(n-3)} \; \delta_{m,n-4} \\
  & + 2(n-2 + n + 1)\sqrt{n(n-1)} \; \delta_{m,n-2} \\
  & + (1 + 4n + 4n^2 + (n+1)(n+2) + n(n-1)) \; \delta_{m,n} \\
  & + 2(n+2 + n + 1)\sqrt{(n+1)(n+2)} \; \delta_{m,n+2} \\
  & + \sqrt{(n+1)(n+2)(n+3)(n+4)} \; \delta_{m,n+4}
\end{align}
in kon"cen izraz za matri"cni element $H$ v bazi $|n\rangle$
\begin{align}
  H_{mn} &= \left(n+\frac{1}{2}\right) \delta_{m,n} + \lambda\langle m|\hat x^4|n\rangle
\end{align}

Matrika $H$ ima pet neni"celnih diagonal. 

\section{Lastne energije}

Izra"cunal sem prvih 100 lastnih energij oscilatorja, ki so enake najni"zjih lastnim vrednostim zgoraj opisane matrike. Uporabil sem matriko velikosti $1000\times1000$, nato pa z uporabo rutin iz paketa \texttt{ARPACK} izra"cunal 100 najni"zjih lastnih vrednosti. Uporaba 1000 baznih stanj se mi je zdela primerna, ker pri najve"cjih obravnavani motnji $\lambda=1$ energija stanja ravno dose"ze energijo tiso"cega lastnega stanja. 

\begin{figure}[H]
 \centering
\input{g_energije}
\caption{Odvisnost prvih 100 lastnih energij anharmonskega oscilatorja od motnje $\lambda$}
\label{fig:lastne-energije}
\end{figure}

V primeru brez motnje energija je $n$-tega stanja linearno nara"s"ca z $n$, kar seveda pri"cakujemo po ena"cbi $E_n^{(0)} = n + 1/2$. Pri pove"cevanju motnje odvisnost postaja superlinearna, pove"cajo pa se tudi lastne energije najni"zjih stanj. Odvisnost pri $\lambda=1$ lahko pribli"zno opi"semo kot $E_n^{(1)}(n) = 1.9 + 1.47 \cdot n^{1.32}$. 

\section{Konvergenca lastnih energij}

Opazoval sem tudi, kako se energije lastnih stanj pribli"zujejo konstanti vrednosti, ko pove"cujemo velikost matrike. V ta namen se "studiral le prvih 20 lastni stanj. "Se vedno sem uporabil matriko, skonstruirano po postopku iz prvega poglavja. 

\begin{figure}[H]
\centering
\input{g_konvergenca_ho_01}
\caption{Konvergenca stanj s kontruirano matriko in $\lambda=0.1$}
\end{figure}

\begin{figure}[H]
\centering
\input{g_konvergenca_ho_1}
\caption{Konvergenca stanj s kontruirano matriko in $\lambda=1$}
\end{figure}

Opazimo, da so pri uporabi premajhnih matrik lastne energije vedno prevelike. Pri pove"cevanju $N$, torej pri dodatku ve"cjega "stevila osnovnih stanj harmonskega oscilatorja, lastne vrednosti padejo in se po"casi pribli"zujejo kon"cnim vrednosti. 

Enak postopek sem ponovil tudi z baznimi stanji, dobljenimi po Lanczosevem algoritmu. Najprej sem poskusil za za"cetno stanje vzeti osnovno stanje HO. Pri tem sem za"sel v te"zave, saj je pri majhni motnji to stanje blizu lastnega stanja hamiltoniana. Rezultati so dosti bolj"si, "ca za za"cetno stanje uporabim produkt po elementih med osnovnim stanjem in naklju"cnim vektorjem. Tak"sno stanje ima "se vedno Gaussovo envelopo in gre na obeh koncih proti ni"c, zato nimamo te"zav zaradi robnih efektov. Hkrati pa zaradi naklju"cnosti ni blizu nobenemu lastnemu stanju hamiltoniana. 

\begin{figure}[H]
\centering
\input{g_konvergenca_L_01}
\caption{Konvergenca stanj z Lanczosevim algoritmom in $\lambda=0.1$}
\end{figure}

\begin{figure}[H]
\centering
\input{g_konvergenca_L_1}
\caption{Konvergenca stanj z Lanczosevim algoritmom in $\lambda=1$}
\end{figure}

Lanczosev algoritem se v tem primeru izka"ze za slab"sega, saj tudi pri zelo velikih matrikah ($250\times 250$) energije ne skonvergirajo. Napake lahko pripi"semo numeri"cni nenatan"cnosti, saj pri Lanczosevem postopku pogosto od"stevamo vektorje med seboj. Poleg tega je natan"cnost omejena z diskretizacijo. "Ce "zelimo $N$ ortogonalnih baznih stanj, mora biti prostor diskretiziran na vsaj $N$ enot, "se bolj"sa pa je bolj fina diskretizacija. V na"sem primeru ima konstrukcija matrike iz prvega poglavja odlo"cilno prednost, saj se izognemo tako diskretizaciji prostora kot numeri"cnemu od"stevanju vektorjev. 

Nazadnje sem preverjal "se "stevilo stanj, ki so dovolj blizu njihove kon"cne vrednosti, v odvisnosti od velikosti matrike $N$. Mera za dovolj blizu je bila, da se je energija razlikovala od kon"cne vrednosti za najvec $\varepsilon = 0.001$. Ket kon"cno vrednost sem uporabil vrednosti pri $N=1000$. S kontruirano matriko opazimo lepo linearno nara"s"canje, naklon premice pa je kar enak iskanemu "stevilu $r$. Lanczoseva metoda pri velikosti matrik, ki sem si jih lahko privo"s"cil s svojim ra"cunalnikom, "se ni skonvergirala, tako da nisem mogel dolo"citi vrednosti $r$. 

\begin{figure}
\centering
 \input{g_konvergenca}
 \caption{Konvergenca in dele"z natan"cno izra"cunanih lastnih energij}
 \label{fig:konvergenca}
\end{figure}

Za $\lambda=1$ je vrednost $r$ skoraj 1/5. Iz tega lahko sklepamo, da so lastne energije na sliki \ref{fig:lastne-energije}, kjer sem za izra"cun 100 energij uporabil 1000 lastnih stanj, izra"cunane z natan"cnostjo, bolj"so od $\varepsilon$ 1/1000. 

\section{"Casovni razvoj}

Za ra"cun "casovnega razvoja sem koherentno stanje $|z\rangle$ razvil po lastnih stanjih anharmonskega oscilatorja, $|z\rangle = \sum c_k |\psi_k\rangle$, $c_k = \langle z|\psi_k\rangle$. Ker poznamo energijo vsakega izmed lastnih stanj, lahko "casovni razvoj poljubnega stanja zapi"semo kot

\begin{align}
|z(t)\rangle = \sum c_k |\psi_k(t) \rangle = \sum c_k e^{-iE_kt} |\psi_k\rangle
\end{align}

Animacije, ki so rezultat izra"cuna tako s "casovnim korakom iz prej"snji naloge kot z razvojem po lastnih funkcija, so prilo"zene poro"cilu. V tem primeru je direktna metoda bolj natan"cna. Domnevam, da je vir napake predvsem dejstvo, da je zgornji razvoj kon"cen, torej koherentnega stanja ne moremo natan"cno opisati. 

\end{document}
