\documentclass[a4paper,10pt]{article}
\usepackage[utf8]{inputenc}
\usepackage[slovene]{babel}
\usepackage{graphicx}
\usepackage{hyperref}
\usepackage[left=3cm,right=3cm,top=3cm,bottom=3cm]{geometry}
\usepackage{amsmath}

%opening
\title{Schr\"odingerjeva ena\v cba, stacionaren problem}
\author{Miha \v Can\v cula}

\begin{document}

\maketitle

\section{Matrika hamiltoniana}

Za izra"cun lastnih energij sistema s hamiltonovim operatorjem $\hat H$ potrebujemo matri"cne elemente $H_{ij} = \langle\psi_i|H|\psi_j\rangle$ v neki ortonormirani bazi $|\psi_i\rangle$. Logi"cna izbira za na"s problem so lastna stanja harmonskega oscilatorja $|n\rangle$. Matri"cne elemente bi lahko izra"cunali neposredno iz lastnih stanja in hamiltonove matrike, dobljene z diskretizacijo prostora. Te"zava se pojavi, ko nas zanimajo vi"sja vzbujena stanja. Pri rekurzivnem ali eksplicitnem ra"cunanju Hermitovih polinomov $H_n$ za velike $n$ (ve"cje od 50) namre"c hitro pridemo do odstevanja velikih "stevil, kar zaradi kon"cne ra"cunalni"ske natan"cnosti privede do velikih napak. Dodatno ta stanja zelo hitro nihajo v prostoru, zato moramo za dober opis uporabiti zelo fino diskretizacjo. 

Obema te"zavama se lahko v na"sem primeru izognemo, "ce matri"cne elemente izrazimo analiti"cno. Vemo, da se hamiltonov operator za harmonski oscilator v bazi lastnih stanj $|n\rangle$ glasi
\begin{align}
  \hat H_0 &= \frac{p^2}{2} + \frac{x^2}{2} = a^\dagger a + \frac{1}{2}
\end{align}
Drugi del potenciala, $x^4$, tudi lahko izrazimo s kreacijskim in anihilacijskim operatorjem
\begin{align}
  x = & {} \frac{1}{\sqrt{2}} (a^\dagger + a) \\
  x^4 = & {} \frac{1}{4} (a^\dagger + a)^4 = \frac{1}{4} \left({a^\dagger}^2 + 1 + 2n + a^2\right)^2 \\
      = & {} \frac{1}{4} \Big( {a^\dagger}^4 + 1 + 4n^2 + a^4 + 2{a^\dagger}^2 + 4n + 2a^2 + \\
      &+ 2{a^\dagger}^2n + 2n{a^\dagger}^2 + 2{a}^2n + 2n{a}^2 + {a^\dagger}^2a^2 + a^2{a^\dagger}^2\Big)
\end{align}
kjer je $n=a^\dagger a$, oznake za operatorje pa sem spu"s"cal. Matri"cne elemente kreacijskega in anihilacijskega operatorja poznamo
\begin{align}
  \langle m|\hat a^\dagger|n\rangle &= \sqrt{n+1}\,\delta_{m,n+1}\\
  \langle m|\hat a|n\rangle &= \sqrt{n}\,\delta_{m,n-1} \\ 
  \langle m|\hat n|n\rangle &= n\,\delta_{m,n} 
\end{align}
"Ce to vstavimo v izraz za $x$ in zdru"zimo "clene z enakimi $\delta_{ij}$, dobimo rezultat
\begin{align}
  4\langle m|\hat x^4|n\rangle = & {} \sqrt{n(n-1)(n-2)(n-3)} \; \delta_{m,n-4} \\
  & + 2(n-2 + n + 1)\sqrt{n(n-1)} \; \delta_{m,n-2} \\
  & + (1 + 4n + 4n^2 + (n+1)(n+2) + n(n-1)) \; \delta_{m,n} \\
  & + 2(n+2 + n + 1)\sqrt{(n+1)(n+2)} \; \delta_{m,n+2} \\
  & + \sqrt{(n+1)(n+2)(n+3)(n+4)} \; \delta_{m,n+4}
\end{align}
in kon"cen izraz za matri"cni element $H$ v bazi $|n\rangle$
\begin{align}
  H_{mn} &= \left(n+\frac{1}{2}\right) \delta_{m,n} + \lambda\langle m|\hat x^4|n\rangle
\end{align}

Matrika $H$ ima pet neni"celnih diagonal. 


\section{Konvergenca lastnih energij}

\input{g_konvergenca_ho_01} \\
\input{g_konvergenca_L_01} \\
\input{g_konvergenca_ho_1} \\
\input{g_konvergenca_L_1} \\
\input{g_konvergenca} \\

Pri ve"cjem "stevilu uporabljenih baznih stanj so se za"cele lastne vrednosti podvojevati. 

\end{document}
