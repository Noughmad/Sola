\documentclass[a4paper,10pt]{article}
\usepackage[utf8]{inputenc}
\usepackage[slovene]{babel}
\usepackage{graphicx}
\usepackage{hyperref}
\usepackage[left=3cm,right=3cm,top=3cm,bottom=3cm]{geometry}

%opening
\title{Schr\"odingerjeva ena\v cba}
\author{Miha \v Can\v cula}

\begin{document}

\maketitle

\section{Stabilnost}

Za reševanje časovnega razvoja Schr\"odingerjeve enačbe sem uporabil vse tri obravnavane metode: eksplicitno Eulerjevo, eksplicitno s končnimi propagatorji in implicitno. 
Najprej sem primerjal njehovo stabilnost. V ta namen sem po vsakem časovnem koraku izračunal normalizacijo $\langle\psi|\psi\rangle$. 
Za velikost časovnega koraka sem postavil $\tau=h^{2}$, kjer je $h$ krajevni korak diskretizacije. 

Izkazalo se je, da direktna Eulerjeva metoda ne ohranja normalizacije. 
Vrednost $\langle\psi|\psi\rangle$ močno naraste že po nekaj korakih. 
Ker metoda propade tako hitro je nisem niti upošteval na grafih. 

Ostali dve metodi se obneseta dosti bolje, saj se normalizacija le malo spremeni tudi po nekaj 1000 korakih. 
Primerjava med metodo s končnimi propagatorji in implicitno metodo je na sliki \ref{fig:stabilnost}.

\begin{figure}[h]
\input{g_stabilnost}
\caption{Primerjava stabilnosti eksplicitne in implicitne metode}
\label{fig:stabilnost} 
\end{figure}

Očitno je tako z motnjo kot brez nje implicitna metoda najstabilnejša. 
Opazoval pa sem tudi obnašanje njene stabilnosti, če sistemu dodamu motnjo $\lambda x^{4}$.
V nekaterih primerih normalizacija $\langle\psi|\psi\rangle$ pada, v drugih pa narašča, vedno pa je ta sprememba zelo majhna tudi po 10000 korakih. 
Odvisnosti za nekaj vredosti $\lambda$ so na sliki \ref{fig:stabilnost-imp}

\begin{figure}[h]
\input{g_stabilnost_imp}
\caption{Stabilnost implicitne metode pri različnih vrednosti motnje $\lambda$}
\label{fig:stabilnost-imp} 
\end{figure}

V vseh primerih je skupna verjetnost malo manjša od 1. 
Razlog za to je omejeno območje integracije, sam sem integriral le med $-L$ in $L$, nisem pa upošteval majhne verjetnosti, da se delec nahaja izven škatle. 

\section{Koherentna stanja}

Pravilnost metod sem preverjal tudi z opazovanjem koherentnih stanj. 

Na spodnjih slikah sta prikazani časovni odvisnosti pričakovanega položaja porazdelitve $\langle\psi|x|\psi\rangle$ (slika \ref{fig:trajektorija}) in 
širine porazdelitve $\sqrt{\langle\psi|x^{2}|\psi\rangle - \langle\psi|x|\psi\rangle^{2}}$ (slika \ref{fig:sirina}). 
Vidi se, da v primeru brez motnje ($\lambda=0$) pričakovana vrednost $x$ sledi sinusni krivulji, širina porazdelitve pa ostaja konstantna.
Ko pa vključimo motnjo, postane gibanje kaotično. 
Dodatno lahko vidimo, da se amplituda nihanja pričakovane vrednosti položaja zmanjša. 

\begin{figure}[h]
\input{g_trajektorija}
\caption{Trajektorija pri"cakovane vrednosti polo"zaja delca pri razli"cnih $\lambda$} 
\label{fig:trajektorija}
\end{figure}

\begin{figure}[h]
\input{g_sirina}
\caption{"Sirina porazdelitve verjetnostne gostote delca pri razli"cnih $\lambda$} 
\label{fig:sirina}
\end{figure}

Pri motenem potencialu lahko opazimo, da "sirina porazdelitve hitro naraste, nato pa se ustali pri neki vrednosti in le "se niha. 
"Sirina porazdelitve v harmonskem potencialu, ki bi pri to"cnem ra"cunu morala ostati konstantna, pa za"cne nihati brez izrazitega trenda. 
Le po velikem "stevilu nihajev (nekaj tiso"c korakov) lahko opazimo, da "sirina v povpre"cju nara"s"ca. 

\section{Dvodimenzionalni problem}

Obravnaval sem tudi problem v dveh dimenzijah. 
Poročilu so priložene animacije, ki prikazujejo časovni razvoj koherentnega stanja pri različnih vrednostih $\lambda$. 
Jasno je vidno, da brez motnje dobimo kroženje, ki je malo popačeno zaradi numeričnih napak (uporabil sem območje 100x100). 
Pri dodatku že šibke motnje ($\lambda=0.1$) gibanje postane kaotično, oblika paketa pa se hitro izgubi. 
Pri močnejši motnji je pojav še hitrejši, dodatni potencial postane močnejši od harmonskega in kroženja ne opazimo več. 

\end{document}
