\documentclass[a4paper,10pt]{article}

\usepackage[utf8]{inputenc}
\usepackage[slovene]{babel}
\usepackage{amsmath}
\usepackage{amsfonts}
\usepackage{relsize}
\usepackage[smaller]{acronym}
\usepackage{graphicx}
\usepackage{subfigure}
\usepackage{cite}
\usepackage{url}
\usepackage[unicode=true]{hyperref}
\usepackage{color}
\usepackage[version=3]{mhchem}
\usepackage{wrapfig}
\usepackage{comment}
\usepackage{float}
\usepackage[top=3cm,bottom=3cm,left=3cm,right=3cm]{geometry}

\renewcommand{\vec}{\mathbf}
\newcommand{\eps}{\varepsilon}
\renewcommand{\phi}{\varphi}
\renewcommand{\theta}{\vartheta}
\newcommand{\dd}{\mathrm{d}}

\newcommand{\parcialno}[2]{
  \frac{\partial #1}{\partial #2}
}
\newcommand{\parcdva}[2]{
  \frac{\partial^2 #1}{\partial #2 ^2}
}
\newcommand{\lag}{\mathcal{L}}

\title{Gibanje gravitacijske vrtavke}
\author{Miha \v Can\v cula}
\begin{document}

\maketitle

\begin{abstract}
Razi"s"ci gibanje splo"sne gravitacijske vrtavke s Hamiltonovo funkcijo

\begin{align}
  H = \sum_{s=1}^{3}\left(\frac{1}{2J_s}l_s^2 + mga_s n_s\right)
\end{align}

kjer so $l_1$,$l_2$,$l_3$ komponente vrtilne koli"cine vzdol"z lastnih osi teznorja vztrajnostnega momenta $J$, $\vec e_s$, in $J\vec e_s = J_s\vec e_s$, $n_1$,$n_2$,$n_3$ pa so vertikalne komponente lastnih osi, $n_s = (0,0,1)\cdot \vec e_s$. $\vec l$ in $\vec e$ predstavljajo kanoni"cen set dinami"cnih spremenljivk s slede"co (Liejevo) algebro Poissonovih oklepajev

\begin{align}
  \{l_r, l_s\} &= \sum_t \eps_{rst}l_t \\
  \{l_r, n_s\} &= \sum_t \eps_{rst}n_t \\
  \{n_r, n_s\} &= 0
\end{align}

kjer je $\eps_{rst}$ popolnoma antisimetri"cen Levi-Civitajev tenzor. Vzemi npr. perturbirano Langrangeovo vrtavko, s parametri $(\Delta, a, \lambda)$, $J_1=J_2 = 1$ in $J_3=\Delta$, ter $a_1 = \lambda$, $a_2=0$, $a_3=a$, tako da je za $\lambda=0$ vrtavka integrabilna. Napravi smiselno predstavitev dinamike s Poincarejevo preslikavo in poskusi dolo"citi relativen dele"z kaoti"cnega faznega prostora (energijske lupine, pri neki energiji $E$) v odvisnosti od perturbacijskega parametra $\lambda$. 
\end{abstract}

\section{Ena"cbe gibanja}

Ena"cbe gibanja vrtavke v Hamiltonovi formulaciji se glasijo

\begin{align}
  \dot l_s &= \{H,l_s\} \\
  \dot n_s &= \{H,n_s\}
\end{align}

V Hamiltonianu nastopajo tako $l_s$ kot $n_s$, njihove medsebojne Poissonove oklepaje pa poznamo. Netrivialni so le "cleni z $l_s^2$, zato moramo najprej izra"cunati Poissonove oklepaje oblike

\begin{align}
  \{l_s^2, l_r\} &= \sum_i\left( \parcialno{l_s^2}{q_i}\parcialno{l_r}{p_i} - \parcialno{l_s^2}{p_i}\parcialno{l_r}{q_i} \right) = 2l_s \sum_i\left[ \parcialno{l_s}{q_i}\parcialno{l_r}{p_i} - \parcialno{l_s}{p_i}\parcialno{l_r}{q_i} \right] = 2l_s \{l_s, l_r\}
\end{align}

Uporabili smo kanoni"cne spremenljivke $q_i$ in $p_i$, ki nastopajo v definiciji Poissonovega oklepaja. Fizika ni odvisna od izbire teh spremenjivk, in tudi v nadaljnjem ra"cunanju jih ne bomo potrebovali. 

Sedaj lahko zapi"semo tudi ena"cbe gibanja v ekplicitni obliki

\begin{align}
  \dot l_s = \{H,l_s\} &= \frac{l_{s+1}}{J_{s+1}}l_{s+2} - \frac{l_{s+2}}{J_{s+2}}l_{s+1} - mga_{s+1}n_{s+2} + mga_{s+2}n_{s+1} \\
  \dot n_s = \{H,n_s\} &= \frac{l_{s+1}}{J_{s+1}}n_{s+2} - \frac{l_{s+2}}{J_{s+2}}n_{s+1}
\end{align}

kjer upo"stevamo cikli"cne indekse, $l_{s+3} = l_s$. 

Za na"s poseben primer vrtavke, z dolo"cinim vrednostmi za $J_s$ in $a_s$, se ena"cbe glasijo

\begin{align}
  \dot l_1 &= l_2l_3 \left( 1 - \frac{1}{\Delta} \right) + mgan_2 \\
  \dot l_2 &= l_3l_1 \left( \frac{1}{\Delta} - 1 \right) + mg\left( \lambda n_3 - a n_1 \right) \\
  \dot l_3 &= -mg \lambda n_2 \\
  \dot n_1 &= l_2n_3 - \frac{l_3n_2}{\Delta} \\
  \dot n_2 &= \frac{l_3n_1}{\Delta} - l_1n_3 \\
  \dot n_3 &= l_1 n_2 - l_2 n_1
\end{align}

\section{Poincarejeva preslikava}

Vrtavka nima nobene lastne "casovne enote ali od zunaj dolo"cene frekvence. Kljub temu pa se sistem vrti, zato je najbolj primerna Poincarejeva preslikava. Preostane nam le "se izbira koli"cine, ki jo bomo dolo"cili s se"cno ploskvijo. 

Element faznega prostora je podan s tremi komponentami vrtilnime koli"ciname in tremi komponentami orientacije. Vrtilna koli"cina v osnovi ne periodi"cna, zato nobene izmed njenih komponent nisem upo"steval pri preslikavi. Ostanejo nam tri koli"cine, ki dolo"cajo orientacijo vrtavke. 

Lagrangeva vrtavka ima dve enaki vrednosti vztrajnostnega momenta, tretja ($J_3$) pa je od njiju razli"cna. Tak"sne so tudi obi"cajne vrtavke, ki jih vrtimo na ravni povr"sini. Ta asimetrija jim da preferen"cno smer, iz izku"senj vemo, da se ve"cina vrtavk vrti predvsem v smeri osi $z$. V tem na"cinu gibanja se komponenta $n_3$ le po"casi spreminja, medtem ko $n_1$ in $n_2$ nihata s frekvenco vrtenja vrtavke. Naravni izbiri za se"cno ploskev sta torej tak"sni, da je $n_1$ ali $n_2$ enaka ni"c, izbrati moramo pa "se predznak njenega "casovnega odvoda. Od predznaka odvoda ni ni"c odvisno, zato sem lahko izbral, naj bo pozitiven. 

Pri nezmoteni vrtavki sma smeri $x$ in $y$ povsem enakovredni, zato je vseeno, katero izmed komponent $n_1$ ali $n_2$ izberemo. Motnja $\lambda$ pa to simetrijo zlomi. "Ce izhodi"s"ce koordinatnega sistema postavimo v te"zi"s"ce vrtavke, je pri nemoteni prijemali"s"ce na osi $z$, motnja pa ga premakne nekam na ravnino $xz$. Edini lastni vektor, ki je "se vedno pravokoten na prijemali"s"ce, je $e_y$ oz. $e_2$. Za se"cno ploskev sem zato izbral pogoj $n_2 = 0$ in $\dot n_2 > 0$. 

\section{Ljapunovi eksponenti}

Kaoti"cnost sistema sem dolo"cal z najve"cjim Ljapunovim eksponentom. Ta eksponent sem izra"cunal numeri"cno, s simulacijo dveh bli"znjih orbit. 

Velikost eksponenta nas v resnici ne zanima, pomembno je le, ali je ve"cji od 0. Ravno s tak"snimi kriteriji imajo numeri"cne metode najve"cje te"zave. Zato sem se potrudil dose"ci "cim ve"cjo natan"cnost, tako da sem izra"cunal relativno razhajanje orbit po vsaki izmed 200 iteracij Poincarejeve preslikave. Logaritmom izra"cunanih razhajanj sem prilagodil linearno funkcijo, nato pa ugotavljal, ali je dobljeno nara"s"canje dovolj mo"cno, da lahko trdimo, da je Ljapunov eksponent pozitiven. 

Na vsakem koraku sem orbiti reskaliral, tako da je razlika med njima vedno ostala majhna. Velikosti relativnih skokov na vsakem koraku sem prilagodil funkcijo $y(t) = a+b/t$, saj v primeru kaosa pri"cakujemo konstanto, druga"ce pa padanje kot $\sim 1/t$. Funkcijo sem prilagajal s knji"znico \texttt{GSL}, ki vrne tudi kovarian"cno matriko, iz katere lahko preberemo napako posameznega koeficienta. Orbito sem "stel kot kaoti"cno, "ce je bil koeficient $a$ ve"cji od absolutne vrednosti svoje napake. 

\section{Dele"z kaoti"cnega faznega prostora}

Fazni prostor sem omejil z najve"cno energijo $E$, nato pa naklju"cno izbral po 200 za"cetnih to"ck znotraj energijske lupine in izra"cunal najve"cji Ljapunov eksponent za vsak za"cetni pogoj. Pre"stel sem, v koliko primerih je bila orbita kaoti"cna, v odvisnosti od parametra motnje $\lambda$. Rezultati so na sliki \ref{fig:kaos}. 

\begin{figure}[H]
 \label{fig:kaos}
 \input{g_kaos}
\end{figure}

Po pri"cakovanju kaoti"cnost nara"s"ca z velikostjo motnje $\lambda$. Kljub temu, da je Lagrangeva vrtavka z $\lambda=0$ integrabilna, sem vseeno dobil dolo"cen dele"z (pribli"zno 10\%) kaoti"cnosti. To pripisujem predvsem napakam numeri"cnega ra"cunanja. 

Nepri"cakovan rezultat pa je, da kaoti"cnost pada z nara"s"canjem energije, torej je kaoti"cen predvsem del faznega prostora z nizkimi energijami oz. po"casnim vrtenjem vrtavke. 

\end{document}
