\documentclass[a4paper,10pt]{article}
\usepackage[utf8]{inputenc}
\usepackage[slovene]{babel}
\usepackage{amsmath}
%opening
\title{Derjaguinov pribli\v zek}
\author{Miha \v Can\v cula}

\newcommand{\dd}{\;\mathrm{d}}

\begin{document}

\maketitle

\section{Izpeljava}

Vzemimo dve veliki krogli s polmeroma $R_1$ in $R_2$ na majhni medsebojni razdalji $D$. 
"Ce sta oba polmera mnogo ve"cja od razdalje med sferama, k sili med njima prispeva le interakcija med tankimi obro"ci na obeh sferah, ki so od zveznice med sferama oddaljeni $x$ in imajo $2\pi x \dd x$. 
Skupna sila med dvema sferama je torej enaka

\begin{align}
\label{eq:integral}
 F(D) &= \int_{r=0}^{r=\infty} 2\pi r \dd r f(Z)
\end{align}

kjer je $Z$ razdalja med tankima obro"cema na oddaljenosti $x$ od zveznice, $f(Z)$ pa sila na enoto povr"sine med dvema ravnima povr"sinama. 
Za majhne $r$ lahko kroglo pribli"zamo s parabolo in dobimo zvezo med $Z$ in $r$ kot

\begin{align}
 Z &= D + z_1 + z_2 = D + \frac{r^2}{2} \left(\frac{1}{R_1} + \frac{1}{R_2}\right) \\
 \dd Z &= \frac{r^2}{2} \left(\frac{1}{R_1} + \frac{1}{R_2}\right)r \dd r
\end{align}

Izraza lahko vstavimo v ena"cbo (\ref{eq:integral}) in dobimo

\begin{align}
 F(D) &\approx \int_D^\infty 2\pi \left(\frac{1}{R_1} + \frac{1}{R_2}\right)^{-1}f(Z) \dd Z = 2\pi \left(\frac{R_1R_2}{R_1 + R_2}\right) W(D)
\end{align}

kjer je $W(D)$ interakcijska energija na enoto povr"sine med dvema ravnima povr"sinama na razdalji $D$. Ker je $f(Z)$ sila na enoto povr"sine, je njen integral
\begin{align}
  W(D) = \int_D^{\infty} f(Z) \dd Z 
\end{align}
enak delu, ki ga moramo opraviti, da plo"s"ci z razdalje $D$ razmaknemo neskon"cno dale"c. 
To delo pa je po definiciji enako interakcijski energiji med plo"s"cama. 
Tu sta tako $f(Z)$ kot $W(D)$ sta definirani na enoto povr"sine. 

\newpage
\section{Druga"cne geometrije}

V gornji izpeljavi smo dejstvo, da imamo opravka ravno s sferami in ne s kak"snimi drugimi zaobljenimi povr"sinami, upo"stevali le pri zvezi med $Z$ in $r$. 
Zelo podobno izpeljavo lahko ponovimo tudi za druga"cne geometrije, na primer za interakcijo med sfero in ravno plo"s"co, ali pa med prekri"zanima valjema. 

\begin{table}[h]
\begin{tabular}{c|c|c|c}
 Geometrija & $Z - D$ & $\dd Z$ & $F(D) / W(D)$ \\
 \hline
 Dve sferi & $\frac{r^2}{2} \left(\frac{1}{R_1} + \frac{1}{R_2}\right)$ & $\left(\frac{1}{R_1} + \frac{1}{R_2}\right)r \dd r$ & $2\pi \left(\frac{R_1R_2}{R_1 + R_2}\right)$ \\
 \hline
 Enaki sferi & $\frac{r^2}{R}$ & $\frac{2}{R}r \dd r$ & $\pi R$ \\
 \hline
 Sfera in plo"s"ca & $\frac{r^2}{2R}$ & $\frac{1}{R} r \dd r$ & $2\pi R$ \\
 \hline
 Prekri"zana valja & $\frac{x^2}{2R_1} + \frac{y^2}{2R_2}$ & $\frac{x \dd x}{R_1} + \frac{y \dd y}{R_2}$ & $2\pi \sqrt{R_1R_2}$
\end{tabular}
\label{tab:geometrije}
\caption{Razmerje med silo med ukrivljenima povr"sinama $F(D)$ in energijo interakcije med ravnima povr"sinama $W(D)$ za nekaj zna"cilnih geometrij}
\end{table}

\subsection{Sfera in plo"s"ca}

Silo med sfero in ravno plo"s"co lahko izra"cunamo kar kot limito, ko se polmer ene izmed sfer pribli"zuje neskon"cno. 

\begin{align}
 F(D) &\approx 2\pi R\cdot W(D)
\end{align}


\subsection{Prekri"zana valja}

V tem primeru nimamo osne simetrije, zato integral $\int 2\pi r \dd r$ nadomestimo z $\int \dd x \dd y$. 
Razdalja med to"ckama na valjih $Z$ je odvisna od $x$ in $y$ posebej. 

\begin{align}
 F(D) &= \iint_{-\infty}^{\infty} f(Z) \dd x \dd y \\
 Z &= D + \frac{x^2}{2R_1} + \frac{y^2}{2R_2}
\end{align}

Primer, ko imata oba valja enak polmer, je enostaven, saj je $Z = \frac{x^2 + y^2}{2R}$ odvisen le od oddaljenosti od zveznice, enako kot pri sferah. 
Integral pa lahko izra"cunamo tudi za dva razli"cna valja s spretno zamenjavo spremenljivk, nampre"c $\tilde y = y\sqrt{R_1/R_2}$. 
Sila med valjema lahko izra"cunamo kot prej
\begin{align}
 Z &= D + \frac{x^2 + \tilde y^2}{2R_1} = D + \frac{r^2}{2R_1} \\
 \dd Z &= \frac{x \dd x}{R_1} + \frac{\tilde y \dd \tilde y}{R_1} = \frac{r \dd r}{R_1} \\
 F(D) &\approx \int_D^{\infty} \sqrt{\frac{R_2}{R_1}} 2\pi R_1 f(Z) \dd Z = 2\pi \sqrt{R_1 R_2} W(D) 
\end{align}
kjer je faktor $\sqrt{R_2/R_1}$ Jacobijeva determinanta prehoda iz koordinate $(x, y)$ na $(x, \tilde y)$. 

\section{Primer: Deplecijska interakcija}

Deplecijsko silo med kroglicami smo pri tem predmetu "ze obravnavali, enaka je
\begin{align}
\label{eq:depl-sila}
 \mathcal F(h) &= -\pi \frac{N}{\beta V} \left(R + \frac{2\sigma-h}{2}\right) \left(R + \frac{2\sigma + h}{2}\right)
\end{align}
kjer je $N/V$ "stevilska gostota kroglic, $R$ polmer sfer in $\sigma$ polmer kroglic. 
Razdalja med sferama $h$ je definirana kot razdalja med sredi"s"ci sfer. 
Izraz velja za $2R \leq h \leq 2R+2\sigma$, povsod drugje je sila enaka nic. 
Za konsistenco z zapisom v prej"snjem poglavju sem raje uporabljal spremenljivko $D = h - 2R$, ki predstavlja razdaljo med povr"sinama sfer. 
Sila je nine"celna, "ce je $D \leq 2\sigma$. 

Deplecijsko interakcijo med ravnima izra"cunamo podobno, le da je izra"cun preseka prepovedanih prostornin enostavnej"si. 
Ta je enak kar
\begin{align}
 V' = A (2\sigma - D)
\end{align}
iz "cesar dobimo izraz za prosto energijo (spet pi"sem le izraz za primer, ko je $V' > 0$)
\begin{align}
 F &= -\frac{1}{\beta} \ln C(T) + \frac{N}{\beta}(\ln V - \frac{V'}{V})
\end{align}
Tu je $V$ celotna prostornina posode, ki ni odvisna od $h$. 
Odvisnost od razdalje med plo"scami se skriva le v $V'$. 
Interakcijsko energijo med plo"s"cama lahko ena"cimo s "clenom, ki je odvisen od $h$, zanima nas pa le energija na enoto povr"sine. 

\begin{align}
\label{eq:depl-energija}
 W(h) &= \frac{N}{\beta V} (D - 2\sigma)
\end{align}

Derjaguinov pribli"zek trdi, da se izra"za (\ref{eq:depl-sila}) in (\ref{eq:depl-energija}) razlikujeta le za multiplikativno konstanto $\pi R$. 
Da to preverimo najprej zapi"semo silo med sferama z medsebojno razdaljo $D$

\begin{align}
 \mathcal F(D) &= -\pi \frac{N}{\beta V} \left(\sigma - \frac{D}{2} \right) \left(\sigma + \frac{D}{2} + 2R\right) \\
  &\approx -\pi\frac{N}{\beta V} \left(\sigma - \frac{D}{2} \right) \cdot 2R \\
  &= \pi R \frac{N}{\beta V}(D - 2\sigma) = \pi R\; W(D)
\end{align}
V drugi vrstici smo upo"stevali, da sta razdalja med sferama $D$ in polmer kroglic $\sigma$ mnogo manj"sa od polmera sfer $R$. 
Zgornja zveza je enaka tisti za enaki sferi v Tabeli \ref{tab:geometrije}, kar potrjuje veljavnost Derjaguinovega pribli"zka. 

\begin{thebibliography}{10}

\bibitem{israelachvili}
J.~N. Israelachvili.
\newblock {\em Intermolecular and Surface Forces}.
\newblock Academic Press (1992).

\bibitem{evans}
D.~F. Evans in H. Wennerstr\"om. 
\newblock {\em The Colloidal Domain}.
\newblock Wiley (1999).



\end{thebibliography}


\end{document}
