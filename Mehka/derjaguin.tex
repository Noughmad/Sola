\documentclass[a4paper,10pt]{article}
\usepackage[utf8]{inputenc}
\usepackage[slovene]{babel}
\usepackage{amsmath}
%opening
\title{Derjaguinov pribli\v zek}
\author{Miha \v Can\v cula}

\newcommand{\dd}{\;\mathrm{d}}

\begin{document}

\maketitle

\section{Izpeljava}

Vzemimo dve veliki krogli s polmeroma $R_1$ in $R_2$ na majhni medsebojni razdalji $D$. 
"Ce sta oba polmera mnogo ve"cja od razdalje med sferama, k sili med njima prispeva le interakcija med tankimi obro"ci na obeh sferah, ki so od zveznice med sferama oddaljeni $x$ in imajo $2\pi x \dd x$. 
Skupna sila med dvema sferama je torej enaka

\begin{align}
\label{eq:integral}
 F(D) &= \int_{r=0}^{r=\infty} 2\pi r \dd r f(Z)
\end{align}

kjer je $Z$ razdalja med tankima obro"cema na oddaljenosti $x$ od zveznice, $f(Z)$ pa sila na enoto povr"sine med dvema ravnima povr"sinama. 
Za majhne $r$ lahko kroglo pribli"zamo s parabolo in dobimo zvezo med $Z$ in $r$ kot

\begin{align}
 Z &= D + z_1 + z_2 = D + \frac{r^2}{2} \left(\frac{1}{R_1} + \frac{1}{R_2}\right) \\
 \dd Z &= \frac{r^2}{2} \left(\frac{1}{R_1} + \frac{1}{R_2}\right)r \dd r
\end{align}

Izraza lahko vstavimo v ena"cbo (\ref{eq:integral}) in dobimo

\begin{align}
 F(D) &\approx \int_D^\infty 2\pi \left(\frac{1}{R_1} + \frac{1}{R_2}\right)^{-1}f(Z) \dd Z = 2\pi \left(\frac{R_1R_2}{R_1 + R_2}\right) W(D)
\end{align}

kjer je $W(D)$ interakcijska energija med dvema ravnima povr"sinama na razdalji $D$. Iz definicije $f(Z)$ sledi, da je $\int 2\pi x \dd x f(Z) = W(D)$. 

\section{Druga"cne geometrije}

V gornji izpeljavi smo dejstvo, da imamo opravka ravno s sferami in ne s kak"snimi drugimi zaobljenimi povr"sinami, upo"stevali le pri zvezi med $Z$ in $r$. 
Zelo podobno izpeljavo lahko ponovimo tudi za druga"cne geometrije, na primer za interakcijo med sfero in ravno plo"s"co, ali pa med prekri"zanima valjema. 

\begin{table}[h]
\begin{tabular}{c|c|c|c}
 Geometrija & $Z - D$ & $\dd Z$ & $F(D) / W(D)$ \\
 \hline
 Dve sferi & $\frac{r^2}{2} \left(\frac{1}{R_1} + \frac{1}{R_2}\right)$ & $\left(\frac{1}{R_1} + \frac{1}{R_2}\right)r \dd r$ & $2\pi \left(\frac{R_1R_2}{R_1 + R_2}\right)$ \\
 \hline
 Sfera in plo"s"ca & $\frac{r^2}{2R}$ & $\frac{1}{R} r \dd r$ & $2\pi R$ \\
 \hline
 Prekri"zana valja & $\frac{x^2}{2R_1} + \frac{y^2}{2R_2}$ & $\frac{x \dd x}{R_1} + \frac{y \dd y}{R_2}$ & $2\pi \sqrt{R_1R_2}$
\end{tabular}
\end{table}

\subsection{Sfera in plo"s"ca}

Silo med sfero in ravno plo"s"co lahko izra"cunamo kar kot limito, ko se polmer ene izmed sfer pribli"zuje neskon"cno. 

\begin{align}
 F(D) &\approx 2\pi R\cdot W(D)
\end{align}


\subsection{Prekri"zana valja}

V tem primeru nimamo osne simetrije, zato integral $\int 2\pi r \dd r$ nadomestimo z $\int \dd x \dd y$. 
Razdalja med to"ckama na valjih $Z$ je odvisna od $x$ in $y$ posebej. 

\begin{align}
 F(D) &= \iint_{-\infty}^{\infty} f(Z) \dd x \dd y \\
 Z &= D + \frac{x^2}{2R_1} + \frac{y^2}{2R_2}
\end{align}

Primer, ko imata oba valja enak polmer, je enostaven, saj je $Z = \frac{x^2 + y^2}{2R}$ odvisen le od oddaljenosti od zveznice, enako kot pri sferah. 
Integral pa lahko izra"cunamo tudi za dva razli"cna valja s spretno zamenjavo spremenljivk, nampre"c $\tilde y = y\sqrt{R_1/R_2}$. 
Sila med valjema lahko izra"cunamo kot prej
\begin{align}
 Z &= D + \frac{x^2 + \tilde y^2}{2R_1} = D + \frac{r^2}{2R_1} \\
 \dd Z &= \frac{x \dd x}{R_1} + \frac{\tilde y \dd \tilde y}{R_1} = \frac{r \dd r}{R_1} \\
 F(D) &\approx \int_D^{\infty} \sqrt{\frac{R_2}{R_1}} 2\pi R_1 f(Z) \dd Z = 2\pi \sqrt{R_1 R_2} W(D) 
\end{align}
kjer je faktor $\sqrt{R_2/R_1}$ Jacobijeva determinanta prehoda iz koordinate $(x, y)$ na $(x, \tilde y)$. 

\begin{thebibliography}{10}

\bibitem{israelachvili}
J.~N. Israelachvili.
\newblock {\em Intermolecular and Surface Forces}.
\newblock Academic Press (1992).



\end{thebibliography}


\end{document}
