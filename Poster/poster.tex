\documentclass{beamer}

\usetheme{Madrid}

\usepackage[orientation=portrait,size=a1,scale=1]{beamerposter}
\usepackage[overlay]{textpos}

\usepackage[utf8]{inputenc}
\usepackage{default}

\setbeamertemplate{background canvas}[vertical shading][top=orange,bottom=yellow]
\beamertemplatenavigationsymbolsempty
\setbeamertemplate{blocks}[rounded][shadow=false]
\setbeamercolor{block body}{bg=yellow!30}

\title{Modeling of light propagation through smectic waveguides}

\author{\underline{Miha \v Can\v cula\inst{1}}\and Miha Ravnik\inst{1}}
\institute{\inst{1}Faculty of Mathematics and Physics, University of Ljubljana}

\begin{document}

\begin{textblock}{18}(-1,0.25)
\begin{block}{}
\centering
{\Huge \inserttitle} \\
\vspace{1cm}
{\LARGE \insertauthor} \\
\vspace{1cm}
\insertinstitute
\end{block}
\end{textblock}

\begin{textblock}{5.5}(0.5,2.5)
\begin{block}{Motivation}
Liquid crystal waveguides are useful for Bla bla bla
Slikca od Musevica s smekticnim vlaknom
\end{block}
\end{textblock}

\begin{textblock}{5.5}(0.5,4)
 \begin{block}{Methods}
  We used \textsc{FDTD}, prilozi slikce (Yee lattice vs moj)
 \end{block}

\end{textblock}

\begin{textblock}{8.5}(6.5,2.5)
\begin{block}{Eigenmodes}
We showed that a Gaussian beam entering a fibre quickly turns into a Laguerre-Gaussian beam, and then into something else. 
\end{block}
\end{textblock}

\end{document}
