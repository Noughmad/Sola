\documentclass[a4paper,10pt]{article}

\usepackage[utf8x]{inputenc}
\usepackage[slovene]{babel}
\usepackage{amsmath}
\usepackage{amsfonts}
\usepackage{relsize}
\usepackage[smaller]{acronym}
\usepackage{graphicx}
\usepackage{subfigure}
\usepackage{cite}
\usepackage{url}
\usepackage{hyperref}
\usepackage{color}
\usepackage[version=3]{mhchem}
\usepackage{wrapfig}

%opening
\title{Hidrodinamske nestabilnosti}

\renewcommand{\vec}{\mathbf}

\begin{document}
\begin{center}
\includegraphics[width=6cm]{../logo_fmf_uni-lj_sl}\\[0.5cm]
Oddelek za fiziko \\[2cm]
{ \large Seminar -- 1. letnik, II. stopnja } \\[1cm]
{ \huge \bf Hidrodinamske Nestabilnosti}\\[2cm]
{\large Avtor: Miha \v Can\v cula}\\[0.6cm]
{\large Mentor: prof. dr. Alojz Kodre} \\[0.6cm]
{\large Ljubljana, marec 2012}
\end{center}
\vfill

\begin{abstract}

\end{abstract}

\section{Uvod}

\section{Stabilnost}

\subsection{Definicija}

O nestabilnosti govorimo, ko infinitezimalno majhna sprememba trenutnega stanja lahko povzro"ci ve"cjo, merljivo razliko po nekem kon"cnem "casu~\cite{drazin}. 

Tak"sna definicija je precej splo"sna, zato jo za potrebe seminarja raje definiramo o"zje in bolj eksaktno: Imejmo sistem, katerega "casovno spreminjanje lahko z eno ali ve"c diferencialnimi ena"cbami, ki imajo dolo"ceno simetrijo. Z nastavkom, ki upo"steva to simetrijo, lahko dobimo re"sitev ena"cb. Stabilnost se poka"ze, ko temu nastavku dodamo majhno motnjo, ki ne upo"steva simetrije. Stabilni sistem se bo vrnil v simetri"cno stanje, medtem ko pri nestabilnem pride do zloma simetrije. 

\section{Tanki filmi}

Hidrodinamsko nestabilnost lahko opazujemo pri polzenju teko"cine po klan"cini~\cite{kondic}. 

\begin{figure}[h]
\centering
 \includegraphics[width=.8\textwidth]{./Slike/film-skica}
\caption{Skica teko"cine v dveh dimenzijah. Viden je greben tik za fronto teko"cine in pa zo"zitev dale"c za fronto, ki je pri ra"cunih ne bomo upo"stevali}
\label{fig:film-skica}
\end{figure}


\subsection{Ena"cbe}

"Ce privzamemo nestisljivost teko"cine $\nabla \cdot \vec u = 0$, se Navier-Stokesova ena"cba poenostavi v 

\begin{align}
 \label{eq:ns-film}
 \frac{\partial \vec u}{\partial t} + (\nabla \cdot \vec u) &= -\frac{1}{\rho}\nabla p + \frac{\mu}{\rho}\nabla^2 \vec u + g (\sin \alpha \vec i - \cos \alpha \vec k)
\end{align}

kjer je $\vec u$ hitrost teko"cine, $\rho$ njena gostota in $\mu$ viskoznost. "Clena z $g$ sta dinami"cna in stati"cna komponenta sile te"ze. Pomembni so tudi robni pogoji, obi"cajno se izbere slede"ce:

\begin{itemize}
 \item Na meji med teko"cino in klancem teko"cina ne drsi, torej je tam $\vec u = 0$. 
 \item Na meji med teko"cino in zrakom ima tlak nezveznost, ki jo sorazmerja povr"sinski napetosti in ukrivljenosti meje $\kappa$. 
\end{itemize}

Ker obravnavamo tanke filme, lahko privzamemo, da je debelina $h$ manj"sa od katerekoli dol"zinske skale v ravnini. S tem privzetkom lahko ena"cbo (\ref{eq:ns-film}) poenostavimo v ena"cbo za $h$. 

\begin{align}
 \frac{\partial h}{\partial t} = -\frac{1}{3\mu}\nabla \cdot \left[ \gamma h^3 \nabla \nabla^2 h - \rho g h^3 \nabla h \cos \alpha + \rho g h^2 \sin \alpha \vec i \right]
\end{align}

\subsection{Brezdimenzijska oblika}

\section{Milni mehur"cki}

Milni mehur"cki so stabilni na majhne motnje zaradi povr"sinske napetosti teko"cine. "Ce pa mehur"cek predremo v eni to"cki, ustvarimo rob, kjer povr"sinska napetost ni uravnote"zena, zato se rob za"cne umikati. Ker je opna obi"cajno zelo tanka, 
je ukrivljenost na robu velika, zato fronta napreduje zelo hitro~\cite{diploma}. To napredovanje je pri mehur"ckih tako hitro, da s prostim o"cesom fronte sploh ne opazimo, ampak se nam zdi, da celoten mehur"cek razpade naenkrat. 
 
\begin{figure}[h]
 \centering
\includegraphics[width=.8\textwidth]{./Slike/bubble-3}
\caption{Razpad milnega mehir"cka~\cite{slike-mehurcek}. Kapljice na desni strani niso posejane enakomerno, ampak so vidni ve"cje in manj"se gostote. }
\end{figure}


\section{Razpad toka v kapljice}

\section{Zaklju"cek}

\begin{thebibliography}{3}
  \bibitem{diploma} S. "Copar, Numeri"cna analiza nestabilnosti na robu teko"cinske opne, Diplomsko delo (2009)
  \bibitem{kondic} L. Kondic, SIAM Review \textbf{45}, 95 (2003)
  \bibitem{eggers} J. Eggers in E. Villermaux, Rep. Prog. Phys. \textbf{71}, 036601 (2008)
  \bibitem{drazin} P. G. Drazin, \textit{Introduction to hydrodynamic stability}, Cambridge University Press (2002)
  \bibitem{slike-mehurcek} \url{http://www.dailymail.co.uk/sciencetech/article-1199149/Super-slow-motion-pictures-soap-bubble-bursting-stunning-detail.html} (23. 1. 2012)
\end{thebibliography}


\end{document}
