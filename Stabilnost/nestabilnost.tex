\documentclass[a4paper,10pt]{article}

\usepackage[utf8x]{inputenc}
\usepackage[slovene]{babel}
\usepackage{amsmath}
\usepackage{amsfonts}
\usepackage{relsize}
\usepackage[smaller]{acronym}
\usepackage{graphicx}
\usepackage{subfigure}
\usepackage{cite}
\usepackage{url}
\usepackage{hyperref}
\usepackage{color}
\usepackage[version=3]{mhchem}
\usepackage{wrapfig}

%opening
\title{Hidrodinamske nestabilnosti}

\begin{document}
\begin{center}
\includegraphics[width=6cm]{../logo_fmf_uni-lj_sl}\\[0.5cm]
Oddelek za fiziko \\[2cm]
{ \large Seminar -- 1. letnik, II. stopnja } \\[1cm]
{ \huge \bf Hidrodinamske Nestabilnosti}\\[2cm]
{\large Avtor: Miha \v Can\v cula}\\[0.6cm]
{\large Mentor: prof. dr. Alojz Kodre} \\[0.6cm]
{\large Ljubljana, marec 2012}
\end{center}
\vfill

\begin{abstract}

\end{abstract}

\section{Uvod}

\section{Stabilnost}

\subsection{Definicija}

O nestabilnosti govorimo, ko infinitezimalno majhna sprememba trenutnega stanja lahko povzro"ci ve"cjo, merljivo razliko po nekem kon"cnem "casu~\cite{drazin}. 

Tak"sna definicija ima smisel, ko sistem opi"semo z eno ali ve"c diferencialnimi ena"cbami, tako da pri primernih robnih pogojih dobimo re"sitev z dolo"ceno simetrijo. 

\subsection{Dolo"citev}

Stabilnost sistema, ki se podreja eni ali ve"cim diferencialnim ena"cbam, dolo"cimo tako, 

\section{Tanki filmi}

\section{Milni mehur"cki}

\section{Razpad toka v kapljice}

\section{Zaklju"cek}

\begin{thebibliography}{3}
  \bibitem{diploma} S. "Copar, Numeri"cna analiza nestabilnosti na robu teko"cinske opne, Diplomsko delo (2009)
  \bibitem{kondic} L. Kondic, SIAM Review \textbf{45}, 95 (2003)
  \bibitem{eggers} J. Eggers in E. Villermaux, Rep. Prog. Phys. \textbf{71}, 036601 (2008)
  \bibitem{drazin} P. G. Drazin, \textit{Introduction to hydrodynamic stability}, Cambridge University Press (2002)
\end{thebibliography}


\end{document}
