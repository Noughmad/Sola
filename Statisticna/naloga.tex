\documentclass[a4paper,10pt]{article}
\usepackage[utf8]{inputenc}
\usepackage[slovene]{babel}
\usepackage{amsmath}
\usepackage{graphicx}
\usepackage{float}
\usepackage[left=2cm,right=2cm,top=2cm,bottom=2cm]{geometry}

%opening
\title{Neravnovesni pojavi -- enakost Jarzynskega}
\author{Miha \v Can\v cula}

\begin{document}

\maketitle

\section{Enakost Jarzynskega}

Vzemimo sistem z vsaj enim zunanjim parametrom. V primeru Isingovega modela, ki se mu bomo posvetili v drugem poglavju, je ta parameter zunanje magnetno polje $h$. "Ce je sistem na za"cetku v ravnovesju pri vrednosti parametra $A$, nato pa v kon"cnem "casu spremenimo zunanji parameter na vrednost $B$, s to spremembo opravimo delo $W$. To delo je odvisno od mikrostanja sistema. "Ce pa poskus ponovimo mnogokrat in vsaki"c zapi"semo koli"cino opravljenega dela, enakost Jarzynskega trdi

\begin{align}
 \langle e^{-\beta W}\rangle \equiv \int \mathrm{d}W \rho(W) e^{-\beta W} = e^{-\beta \Delta F}
\end{align}

kjer je $\rho(W)$ verjetnostna porazdelitev dela, $\Delta F = F_B - F_A$ pa razlika prostih energij med ravnovesnima stanjema pri temperaturi $\beta$ in vrednosti zunanjega parametra $A$ oz. $B$. 

Na levi strani enakosti nastopa delo, ki ga opravimo, ko sistem spravimo iz ravnovesja. "Ce je sprememba zunanjega parametra velika in hitra, je lahko sistem dale"c od ravnovesja. Na desni strani enakosti pa nastopajo le ravnovesne vrednosti proste energije. Enakost Jarzynskega torej povezuje neravnovesno obna"sanje sistema z ravnovesnimi koli"cinami. 

Pomembna lastnost zgornje enakosti je, da ni odvisna od na"cina, kako se zunanji parameter spreminja s "casom. Sprememba je lahko poljubno hitra ali pa zelo po"casna. 
\section{Izpeljava}

Hamiltonian sistema s faznim prostorom $\Gamma$ in zunanjim parametrom $\lambda$ lahko zapi"semo kot

\begin{align}
 \mathcal{H} = \mathcal{H}(\Gamma; \lambda)
\end{align}

Na za"cetku je sistem pri $\lambda=A$ v stanju mikrostanju $\Gamma_0$. Ko pa parameter v "casu $\tau$ spremenimo na vrednosti $B$, sistem preide v mikrostanje $\Gamma_\tau$. V hamiltonskem sistemu je opravljeno delo $W$ enako razliki energij na koncu in na za"cetku. 
\begin{align}
 W = \mathcal{H}(\Gamma_\tau; B) - \mathcal{H}(\Gamma_0; A)
\end{align}

"Ce zapi"semo izraz za povpre"cje $\langle e^{-\beta W}\rangle$ po za"cetnem stanju kot
\begin{align}
 \langle e^{-\beta W}\rangle = \int \mathrm{d}\Gamma_0 p(\Gamma_0) e^{\beta [\mathcal{H}(\Gamma_\tau; B) - \mathcal{H}(\Gamma_0; A)]}
\end{align}
in uporabimo verjetnostno porazdelitev za"cetnega stanja
\begin{align}
 p(\Gamma) = \frac{1}{Z(A)} \exp\left[-\beta \mathcal{H}(\Gamma; A)\right]
\end{align}
dobimo povezavo
\begin{align}
 \langle e^{-\beta W}\rangle = \frac{1}{Z(A)}\int \mathrm{d}\Gamma_0 e^{-\beta\mathcal{H}(\Gamma_\tau; B)} 
\end{align}

Sedaj lahko uvedemo kanoni"cno substitucijo spremenljivk $\Gamma_0 \to \Gamma_\tau$ in upo"stevamo Liouvillov izrek o invariantnosti mere pri kanoni"cnih transformacijah. Pridemo do enakosti

\begin{align}
  \langle e^{-\beta W}\rangle = \frac{1}{Z(A)}\int \mathrm{d}\Gamma_\tau e^{-\beta\mathcal{H}(\Gamma_\tau; B)} = \frac{Z(B)}{Z(A)} = e^{-\beta \Delta F}
\end{align}

\section{Simulacija}

Enakosti sem preveril tudi numeri"cno, in sicer na dvodimenzionalnem Isingovem modelu. Zunanje magnetno polje $h$ sem linearno pove"ceval od vrednosti 0 do kon"cne vrednosti $H$ po "casu $\tau$. Pri tak"sni spremembi sistem prejme delo

\begin{align}
 W = -\int_0^H M \;\mathrm{d}H = -\frac{H}{\tau} \int_0^\tau M(t) \; \mathrm{d}t
\end{align}
kjer je $M(t) = \sum_i S_i(t)$ magnetizacija sistema. 

Za simulacijo sem uporabil Metropolisov algoritem, za vsako kombinacijo parametrov ($\beta$, $H$ in $\tau$) pa sem simulacijo ponovil 100000-krat. Vsaki"c sem uporabil kvadratno mre"zo velikosti 64x64. "Casovno enoto sem izbral tako, da je program ob vsakem koraku dol"zine 1 naredil 64x64 korakov Metropolisovega algoritma. Vsak spin je tako imel v povpre"cju vsak korak eno mo"znost, da se obrne. 

\subsection{Paramagnetna faza}

\begin{figure}[H]
\centering
 \input{para}
 \caption{Statistika opravljenega dela v paramagnetni fazi ($\beta = 0.2 < \beta_c$, $\beta h = 0.3$)}
\end{figure}

V paramagnetni fazi je porazdelitev dela pribli"zno Gaussova, ne glede na izbiro "casa $\tau$. S tem ko podalj"sujemo "cas, process postaja vedno bolj obrnljiv, vrh porazdelitve pa o"zji in vi"sji.  

\subsection{Feromagnetna faza}

Pri $\beta$ nad kriti"cno vrednostjo snov preide v feromagnetno fazo. V tej fazi "ze pred vklopom zunanjega polja skoraj vsi spini ka"zejo v isto smer, opravljeno delo pa je odvisno od te smeri. Na voljo imamo le dve smeri (spin $\pm 1$), zato sta na grafu jasno vidna dva vrhova. 

\begin{figure}[H]
\centering
 \input{fero}
 \caption{Statistika opravljenega dela v feromagnetni fazi ($\beta = 0.7 > \beta_c$, $\beta h = 0.1$)}
\end{figure}

Pove"cavi obeh vrhov sta na naslednjih dveh slikah. Vrhova nista Gaussova, podobno kot prej pa s podalj"sevanje "casa $\tau$ postajata ostrej"sa. 

\begin{figure}[H]
\centering
 \input{fero_left}
 \caption{Statistika opravljenega dela v feromagnetni fazi ($\beta = 0.7 > \beta_c$, $\beta h = 0.1$), levi vrh}
\end{figure}

\begin{figure}[H]
\centering
 \input{fero_right}
 \caption{Statistika opravljenega dela v feromagnetni fazi ($\beta = 0.7 > \beta_c$, $\beta h = 0.1$), desni vrh}
\end{figure}

\subsection{Kriti"cna temperatura}

Zanimivo je tudi obna"sanje to"cno pri temperaturi prehoda, torej $\beta = \beta_c = \frac{\ln 1+\sqrt{2}}{2}$. 

\begin{figure}[H]
\centering
 \input{crit_1}
 \caption{Statistika opravljenega dela pri temperaturi prehoda ($\beta = \beta_c$, $\beta h = 0.1$)}
\end{figure}

\begin{figure}[H]
\centering
 \input{crit_3}
 \caption{Statistika opravljenega dela pri temperaturi prehoda ($\beta = \beta_c$, $\beta h = 0.3$)}
\end{figure}

Tudi v tem primeru opazimo dva vrhova, ki pa nista jasno lo"cena med seboj. Poleg tega vrhova nista simetri"cna, polo"zaj desnega je odvisen od "casa $\tau$, medtem ko je levi pribli"zno konstanten. 

\subsection{Preverjanje enakosti}

Opravljeno delo je po absolutni vrednosti preveliko, da bi lahko neposredno izra"cunal pri"cakovano vrednosti $\langle e^{-\beta W}\rangle$, saj pridemo do mejo ra"cunalni"ske natan"cnosti. Namesto tega sem uporabil zvezo

\begin{align}
 \Delta F = \langle W \rangle - \frac{\beta}{2}\theta
\end{align}
kjer sta $\langle W \rangle$ in $\theta$ povpre"cna vrednost in varianca porazdelitve. Rezultati v spodnji tabeli ka"zejo, da $\Delta F$ ni odvisna od izbire "casa $\tau$, ampak le od kon"cne vrednosti $H$. 

\begin{table}[H]
\begin{center}
 \begin{tabular}{|c|c|c|c|}
 \hline
  $\beta$ & $\beta h$ & $\tau$ & $\langle W\rangle - \frac{\beta}{2}\theta$ \\
  \hline
  0.2 & 0.3 & 25 & -2556\\
  0.2 & 0.3 & 50 & -2413\\
  0.2 & 0.3 & 100 & -2305\\
  \hline
  0.7 & 0.1 & 25 & -1416\\
  0.7 & 0.1 & 50 & -1436\\
  0.7 & 0.1 & 100 & -1415\\
  \hline
  0.44 & 0.1 & 25 & -1492\\
  0.44 & 0.1 & 50 & -1509\\
  0.44 & 0.1 & 100 & -1519\\
  \hline
  0.44 & 0.3 & 25 & -13302\\
  0.44 & 0.3 & 50 & -13280\\
  0.44 & 0.3 & 100 & -12800\\
  \hline
 \end{tabular}
 \caption{Pri"cakovane vrednosti $\langle e^{-\beta W}\rangle$}
\end{center}
\end{table}

Vrednosti v zgornji tabeli ka"zejo nekak"sno odvisnosti od $\tau$, ki pa nikoli ne prese"ze pribli"zno 10\%. Tolik"sno odstopanje lahko pojasnimo z izbiro kon"cnega sistema, nenatan"cnosti ra"cunanja. Zaradi "casovne omejitve morda sistem na za"cetku ni bil v ravnovesnem stanju. Odstopanja so kljub temu dovolj majhna, da lahko potrdimo enakost Jarzynskega. 

\end{document}
