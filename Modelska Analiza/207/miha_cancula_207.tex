\documentclass[a4paper,10pt]{article}

\usepackage[utf8]{inputenc}
\usepackage[slovene]{babel}
\usepackage{amsmath}
\usepackage{amsfonts}
\usepackage{relsize}
\usepackage[smaller]{acronym}
\usepackage{graphicx}
\usepackage{subfigure}
\usepackage{cite}
\usepackage{url}
\usepackage[unicode=true]{hyperref}
\usepackage{color}
\usepackage[version=3]{mhchem}
\usepackage{wrapfig}
\usepackage{comment}
\usepackage{float}
\usepackage[top=3cm,bottom=3cm,left=3cm,right=3cm]{geometry}

\renewcommand{\vec}{\mathbf}
\newcommand{\eps}{\varepsilon}
\renewcommand{\phi}{\varphi}
\renewcommand{\theta}{\vartheta}
\newcommand{\dd}{\mathrm{d}}

\newcommand{\parcialno}[2]{
  \frac{\partial #1}{\partial #2}
}
\newcommand{\parcdva}[2]{
  \frac{\partial^2 #1}{\partial #2 ^2}
}
\newcommand{\lag}{\mathcal{L}}

\title{Metoda kon\v cnih elementov: \\ Poissonova ena\v cba}
\author{Miha \v Can\v cula}
\begin{document}

\maketitle

\section{Postopek re"sevanja}

\subsection{Diskretizacija}

Prednost metode kon"cnih elementov je v tem, da lahko prostor razdelimo na trikotnike s poljubnim tlakovanjem. Za "cim bolj"so natan"cnost metode je koristno, da so trikotniki pribli"zno enakostrani"cni in enake velikosti. 

\subsubsection{Polkrog}

Tlakovanje, pri katerem polkrog razdelimo na kolobarje, nato pa vsak kolobar posebej tlakujemo s trikotniki, se je izkalazo za najbolj"se. V tem primeru so vsi trikotniki pribli"zno enako veliki, poleg tega pa je ta metoda najenostavnej"sa za programiranje. 

To"cke in povezave med njimi sem razporedil po vzorcu, prikazanem na sliki \ref{fig:povezave-polkrog}. Z zeleno barvo so ozna"cene robne to"cke, z rde"co pa te"zi"s"ca trikotnikov. 

\begin{figure}[H]
\subfigure[\label{fig:povezave-polkrog}]{\includegraphics[height=.3\textwidth]{g_slika_srediscna}}
\subfigure[\label{fig:povezave-batman}]{\includegraphics[height=.3\textwidth]{g_slika_batman}}
\caption{Diskretizaciji za obe obravnavani geometriji cevi}
\label{fig:povezave}
\end{figure}

\subsubsection{Izsekan kvadrat}

Tlakovanje cevi iz 5. naloge je bilo la"zje, saj tu nimamo okroglih oblik. Delo sem si dodatno poenostavil s tem, da nisem uporabil enakostrani"cnih trikotnikov, ampak enakokrake pravokotne (torej polovico kvadrata). Na ta na"cin sem "zrtvoval nekaj natan"cnosti, saj je hipotenuza dalj"sa od stranice v enakostrani"cnem trikotniku z enako plo"s"cino. Tak"sno tlakovanje pa omogo"ca enostavno pokritje ostrih izsekanih vogalov cevi. Orientacijo trikotnikov sem obrnil tako, da je bila hipotenuza v smeri, kjer sem pri"cakoval najmanj"so ukrivljenost. 

Primer tak"snega tlakovanja je na sliki \ref{fig:povezave-batman}. 

\subsection{Re"sevanje matri"cnega sistema}

Naslednji korak po dolo"citvi polo"zaja to"ck in povezav med njimi je priprava matri"cnega sistema za re"sevanje. Pri tem nastopajo skalarni produkti konstantih funkcij in pa plo"s"cine trikotnikov. To lahko izra"cunamo s preprostimi formulami, ki so napisane tudi v navodilih. Geometrijo moramo upo"stevati le "se za dolo"citev, katere to"cke so na robu, nato pa jo lahko pozabimo. 

Pozornost pa je treba posvetiti "se robnim pogojem. Ra"cunamo vrednost funkcije $u$ le v to"ckah, ki niso na robu obmo"cja, saj imamo robni pogoj "ze podan. Notranjim to"ckam sem priredil vrstico, podano z elementi in desno stranjo kot so opisani v navodilih. 

Matrika $A$ je redka, zato sem za re"sevanje matri"cnega sistema uporabil knji"znico za redke matrike \texttt{cholmod}. Zaradi te lastnosti je matri"cni sistem mo"zno re"siti v linearnem "casu $\mathcal{O}(n)$. Da ne bi po nepotrebnem uvajal kak"sne vi"sje "casovne odvisnosti sem med pripravo matrike vedno se"steval po vseh trikotnikih, katerih "stevilo nara"s"ca linearno s "stevilom to"ck. 

V "casu okrog ene minute mi je uspelo re"siti primere do okrog 100000 to"ck. 

\section{Re"sitve}

Profile hitrosti po cevi sem prikazal z uporabo knji"znice \texttt{MathGL}. Rezultati za oba preseka cevi so na slikah \ref{fig:profil-srediscna} in \ref{fig:pretok-polkrog}. 
 
\begin{figure}[H]
\centering
 \includegraphics[width=\textwidth]{g_contour_srediscna_0}
 \caption{Profil hitrosti v polkro"zni cevi, sredi"s"cna diskretizacija}
 \label{fig:profil-srediscna}
\end{figure}

\begin{figure}[H]
\centering
  \includegraphics[width=.8\textwidth]{g_contour_batman_0}
 \caption{Profil hitrosti v cevi z obliko izrezanega kvadrata}
 \label{fig:profil-kvadrat}
\end{figure}

\section{Pretok po cevi}

Kon"cno sem izra"cunal "se pretok teko"cine po cevi z izbranim profilom. Ker trikotniki nimajo enakih plo"s"cin, je tudi te"za vsakega izra"cunanega koeficienta razli"cna. Pretok, ki pripada to"cki $i$, sem izrazil kot produkt vrednosti $w_i$ in tretjine plo"s"cin vseh trikotnikov, ki mejijo na to to"cko. Tretjina nastopa z enakim razlogom kot pri vektorju desnih strani, ker je 1/3 integral linearne funkcije po trikotniku. Zaradi te lastnosti lahko pretok po cevi hitro izra"cunamo kot skalarni produkt $\vec x \cdot \vec b$.  

\begin{figure}[H]
 \input{g_pretok_polkrog}
 \caption{Pretok po polkro"zni cevi}
 \label{fig:pretok-polkrog}
\end{figure}

\begin{figure}[H]
 \input{g_pretok_kvadrat}
 \caption{Pretok po cevi v obliki izrezanega kvadrata}
 \label{fig:pretok-kvadrat}
\end{figure}

Prave vrednost, ki je limita pri $n\to\infty$, seveda ne moremo izra"cunati na ta na"cin, lahko pa jo ocenimo. Pri tem nam pomaga dejstvo, da ocena za $\Phi$ monotono nara"s"ca s "stevilom to"ck $n$. Privzel sem poten"cno konvergenco in izra"cunanim vrednostim priredil funkcijo

\begin{align}
\label{eq:napoved-napake}
 \Phi(n) = \Phi_0 - \alpha \cdot n^{-\beta}
\end{align}

Optimalno vrednost parametra $\Phi_0$ je ocena za preto"cnost cevi, potenca $\beta$ pa stopnja konvergence. Pri obeh izra"cunih je optimanla vrednost $\beta$ med 0.85 in 0.95, torej nenatan"cnost pada skoraj linearno s "stevilom to"ck. Na slikah \ref{fig:pretok-polkrog} in \ref{fig:pretok-kvadrat} vidimo, da se izra"cunana odvisnost zelo dobro ujema s predpostavko (\ref{eq:napoved-napake}). 

\subsection{Primerjava z znanimi vrednostmi}

Pri re"sevanju pete naloge s pospe"seno relaksacijo sem dobil rezultat $\Phi_{SOR} \approx 0.0163137$. Ta vrednost se na 4 mesta natan"cno ujema z rezultatom te naloge, $\Phi_{FEM} \approx 0.0163169$. 

Iz znanega pretoka po cevi lahko izra"cunamo Poiseuvillov koeficient (v izbranih brezdimenzijskih enotah) kot

\begin{align}
 \kappa = \frac{8\pi}{S^2} \Phi
\end{align}

Za polkro"zno cev je vrednost enaka

\begin{align}
 \kappa_p = \frac{8\pi}{S^2} \Phi_p = \frac{8\pi \cdot 4}{\pi^2} \Phi_p = \frac{32}{\pi}\Phi_p \approx 0.75774
\end{align}

za izrezano kvadratno pa

\begin{align}
 \kappa_k = \frac{8\pi}{S^2} \Phi_k = \frac{8\pi}{(7/9)^2} \Phi_k \approx 0.67790
\end{align}

\section{Hitrost re"sevanja}

Meril sem tudi "cas, ki ga algoritem porabi za re"sevanje problema

\begin{figure}[H]
 \input{g_hitrost}
 \caption{"Cas izvajanja programa}
 \label{fig:hitrost}
\end{figure}

Z grafa lahko vidimo, da "cas izvajanja ni odvisen od geometrije problema ali izbire diskretizacije, le od "stevila to"ck. V obeh geometrija "cas ra"cunanja nara"s"ca linearno s "stevilom to"ck $n$, torej je algoritem reda $\mathcal{O}(n)$.


\end{document}
