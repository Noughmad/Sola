\documentclass[a4paper,10pt]{article}

\usepackage[utf8]{inputenc}
\usepackage[slovene]{babel}
\usepackage{amsmath}
\usepackage{amsfonts}
\usepackage{relsize}
\usepackage[smaller]{acronym}
\usepackage{graphicx}
\usepackage{subfigure}
\usepackage{cite}
\usepackage{url}
\usepackage[unicode=true]{hyperref}
\usepackage{color}
\usepackage[version=3]{mhchem}
\usepackage{wrapfig}
\usepackage{comment}
\usepackage{float}
\usepackage[top=3cm,bottom=3cm,left=3cm,right=3cm]{geometry}

\renewcommand{\vec}{\mathbf}
\newcommand{\eps}{\varepsilon}
\renewcommand{\phi}{\varphi}
\renewcommand{\theta}{\vartheta}
\newcommand{\dd}{\mathrm{d}}

\newcommand{\parcialno}[2]{
  \frac{\partial #1}{\partial #2}
}
\newcommand{\parcdva}[2]{
  \frac{\partial^2 #1}{\partial #2 ^2}
}
\newcommand{\lag}{\mathcal{L}}

\title{Direktno re\v sevanje Poissonove ena\v cbe}
\author{Miha \v Can\v cula}
\begin{document}

\maketitle

\section{Algoritem}

Gibanje vrvice sem ra"cunal v korakih. Vsak korak je sestavljen iz dveh delov:
\begin{enumerate}
 \item Ob poznavanju naklona $\phi(s,t)$ in $\phi(s,t-\Delta t)$ izra"cunomo silo $F(s,t)$
 \item Ob poznavanju naklona $\phi(s,t)$ in $\phi(s,t-\Delta t)$ ter sile $F(s,t)$ izra"cunomo naklon $\phi(s,t+\Delta t)$
\end{enumerate}

"Casovni korak vsebuje le druga ena"cba, medtem ko je prva le pomo"zna. Za"cetni pogoj nam dolo"ca pogoje prvega dela, zato za"cnemo s tem, nato pa izmeni"cno izvajamo oba. 

V ena"cbah nastopa drugi "casovni odvod $\parcdva{\phi}{t}$, zato sem v spominu vedno shranjeval trenutno in prej"snjo vrednosti $\phi$. Po drugi strani pa nikjer ne nastopa "casovni odvod sile, zato mi je ni bilo treba shranjevati med koraki, ampak sem jo potreboval le na prehodu od 1. do 2. dela koraka. 

\subsection{Za"cetni pogoj}

Najenostavnej"sa izbiri za za"cetni pogoj je oblike $\phi(s,0) = \phi_0$, kar predstavlja ravno vrv. Za simulacijo pa to ni dober za"cetni pogoj, saj tak"sno obliko vrvi dobimo le, "ce je vrv napeta z neskon"cno silo. Zelo velika sila v vrvi pa povzro"ci napake v numeri"cnem ra"cunu. Zato sem raje izbral bolj fizikalne robne pogoje, torej vrv v obliki veri"znice. Ena"cba veri"znice v na"sih koordinatah se glasi

\begin{align}
  y &= y_0 + C \cdot \cosh \left(\frac{x-a}{C}\right) \\
  \phi &= \arctan \left(\parcialno{y}{x}\right) = \arctan \sinh\left(\frac{x-a}{C}\right) \\
  \phi &= \arctan \sinh\left(\frac{x-a}{C}\right)
\end{align}

Konstante $y_0$, $C$ in $a$ si izberemo tako, da zadostimo pogoju $y(x=0) = 0$. Druga dva pogoj si lahko izberemo poljubno, dolo"cata pa za"cetni naklon in napetost vrvi. Ravni vrvi se pribli"zamo, 



\subsection{Izra"cun sile v vrvi}

Za"cetni pogoj nam dolo"ca, naklon vrvi po celotni dol"zini, o sili v njej pa ne vemo ni"c. Zato najprej izra"cunamo silo s pomo"cjo ena"cbe (\ref{eq:sila}). 

\begin{align}
 \label{eq:sila}
 \left[\parcdva{}{s} - \left(\parcialno{\phi}{s}\right)^2 \right]F = -\left(\parcialno{\phi}{t}\right)^2
\end{align}

Na desni strani ena"cbe nastopa "casovni odvod kota $\phi$. Tega lahko izra"cunamo kot razliko med kotom ob trenutnem in prej"snjim "casu, na za"cetku pa uporabimo pogoj, da je vrv pri miru $\dot \phi(s,0) = 0$. 

Najbolj primerno se mi je zdelo $F$ in $\phi$ predstaviti kot vektorja, tako da vrv razdelimo na kon"cno "stevilo enako dolgih odsekov. V tak"sni predstavitvi moramo diskretizirati tudi operator odvoda po $s$, se ena"cba (\ref{eq:sila}) prevede na matri"cni sistem

\begin{align}
 \label{eq:sila-diskretno}
 \frac{F_{i-1} - 2F_i + F_{i+1}}{h^2} - \frac{(\phi_{i+1} - \phi_{i-1})^2}{(2h)^2}F_i &= -\left(\dot\phi_i\right)^2 \\
 F_{i-1} + F_{i+1} + \left(-2 - \frac{(\phi_{i+1} - \phi_{i-1})^2}{4}\right) F_i &= -h^2 \left(\dot\phi_i\right)^2
\end{align}

kjer je $h=1/N$ korak diskretizacije. Ker je ena"cba drugega reda, sem za pribli"zek prvega odvoda $\parcialno{\phi}{s}$ uporabil simetri"cno diferenco. 

Zgornja ena"cba seveda velja le tam, kjer so vsi indeksi smiselni, torej povsod razen na za"cetku in koncu vrvi. Na robovih moramo seveda upo"stevati robne pogoje. Kon"cni robni pogoj je enostaven; tam je sila kar predpisana in je enaka ni"c. Zato lahko kon"cno ena"cbo kar izpustimo, prav tako pa izpustimo "clen z $F_{i+1} = 0$ v predzadnji ena"cbi. S tem smo zmanj"sali dimenzijo sistema in prihranili drobec ra"cunske zmogljivosti. 

Robni pogoj na za"cetku vrvi je bolj zapleten, saj ne poznamo vrednosti sile. Poznamo pa njen prvi odvod, ki ga lahko izra"cunamo iz ena"cbe 

\begin{align}
\label{eq:sila-zacetni-pogoj}
 \parcialno{F}{s} + \sin\phi = 0
\end{align}

Prvi odvod lahko zapi"semo s kon"cno diferenco. Ker smo na za"cetku vrvi, ta diferenca ne more biti simetri"cna, zato vzamemo kar najenostavnej"so

\begin{align}
\label{eq:sila-zacetni-pogoj-diskretno}
 F_1 - F_0 = -h\sin\phi_0
\end{align}

Pri tem sem upo"steval, da odseke vrvi "stevil"cimo z $i=0,1,\ldots,N-1$. S $\phi_0$ je tako ozna"cen naklon prvega odseka vrvi, ne pa naklon ob za"cetnem "casu. Sedaj imamo sistem $N-1$ ena"cb za $N-1$ neznank, medtem ko je zadnja neznanka (vrednost sile v zadnjem odseku vrvi) znana. Na ta na"cin lahko izra"cunamo silo v vrvi ob vsakem "casu. 

\subsection{Izra"cun naklona vrvi}

Drugi korak pa je izra"cun naklona vrvi, "ce poznamo naklon ob prej"snjem "casu in napetost. Za to uporabimo ena"cbo

\begin{align}
 \label{eq:kot}
 \parcdva{\phi}{t} = 2\parcialno{F}{s}\parcialno{\phi}{s} + F\parcdva{\phi}{s}
\end{align}

Odvode po $s$ diskretiziramo podobno kot v prej"snjem poglavju, tako da spremenljivki $F$ in $\phi$ obravnavamo kot vektorja. "Casovni odvod pa ima tu druga"cno vlogo, saj "zelimo simulirati gibanje vrvice z velikim "stevilom "casovnih korakov. Poleg tega poznamo le za"cetni pogoj, o kon"cnem stanju pa ne vemo ni"c. Ker pa poznamo tako naklon kot njegov odvod ob za"cetnem "casu, je "casovna odvisnost v resnici za"cetni problem, ki ga lahko re"simo z enostavno integracijo. Tudi "cas diskretiziramo, tako da v vsakem koraku napredujemo za $\Delta t = k$. 

S poznavanjem trenutnega in prej"snjega stanja lahko izra"cunamo naklon od naslednjem "casu. Za jasnej"so pisavo nisem ozna"cil implicitne krajevne odvisnosti. 

\begin{align}
 \label{eq:kot-razvoj}
 \phi(t+k) &= 2\phi(t) - \phi(t-k) + k^2 \left[ 2\parcialno{F}{s}\parcialno{\phi(t)}{s} + F\parcdva{\phi(t)}{s} \right]
\end{align}

Krajevne odvode, ki nastopajo na desni strani ena"cbe spet nadomestimo s kon"cnimi diferencami. 

\begin{align}
 \phi_i(t+k) = 2\phi_i(t) - \phi_i(t-k) + k^2 &\left[2\frac{(F_{i+1}-F_{i-1})(\phi_{i+1}(t)-\phi_{i-1}(t))}{(2h)^2}\right. + \nonumber \\
+ & \left. F_i\frac{\phi_{i+1}(t) - 2\phi_i(t) + \phi_{i-1}(t)}{h^2}\right] \label{eq:kot-diskretno}
\end{align}

Tudi ena"cba (\ref{eq:kot-diskretno}) velja le za odseke v notranjosti vrvi. Na kraji"s"cih jo nadomestimo z robnimi pogoji. Tokrat nobeden izmed obeh robnih pogojev ni tako enostaven kot je bil pri ra"cunu sile, ampak dobimo dve netrivialni ena"cbi. Pogoj v prijemali"s"cu vrvi sledi iz Newtonovega zakona in se glasi

\begin{align}
 \label{eq:kot-zacetni-pogoj}
 F \parcialno{\phi}{s} + \cos \phi = 0
\end{align}

Enako in kot za ena"cbo (\ref{eq:sila-zacetni-pogoj}) lahko naredimo enostransko diferenco

\begin{align}
 \label{eq:kot-zacetni-pogoj-diskretno}
 \phi_0(t+k) = \phi_1(t+k) - \frac{h}{F_0} \cos \phi_0(t+k)
\end{align}

Vrednosti $\phi_1$ lahko izra"cunamo po ena"cbi (\ref{eq:kot-diskretno}), $\phi_0$ pa je neznanka. Ena"cba (\ref{eq:kot-zacetni-pogoj-diskretno}) implicitna v $\phi_0$ in transcendentna. Lahko bi jo re"sevali s standardnimi orodji za re"sevanje nelinearnih ena"cb (npr. Newtonovo metodo). Re"sitev pa dobimo hitreje in enostavneje, "ce robni pogoj zapi"semo s simetri"cno diferenco za drugi odsek vrvi 

\begin{align}
 \label{eq:kot-zacetni-pogoj-diskretno-enostavno}
 \phi_0(t+k) = \phi_2(t+k) - 2\frac{h}{F_0} \cos \phi_1(t+k)
\end{align}

Zgornja ena"cba ni ve"c implicitna, zato lahko $\phi_0$ izra"cunamo direktno. 

Robni pogoj na prostem koncu vrvi nima jasne fizikalne podlage in si ga lahko izberemo. Pogoj dolo"ca le obliko konca vrvi in ima pri ve"canju "stevila odsekov vedno manj"si vpliv, zato sem si izbral enega izmed najenostavnej"sih, ki pa "se vedno da dovolj realisti"cno gibanje vrvi. To je pogoj $\parcdva{\phi}{s}=0$, ki je predlagan tudi v navodilih. Ker nastopa drugi odvod, ga moramo v diskretni obliki zapisati na predzadnjem odseku vrvi (zadnji odsek je $N-1$)

\begin{align}
 \phi_{N-3} -2\phi_{N-2} + \phi_{N-1} = 0
\end{align}

Pogoj uporabimo ob "casu, za katerega ra"cunamo naklon vrvi, torej $\phi_i = \phi_i(t+k)$. Vrednosti v to"ckah $N-3$ in $N-2$ izra"cunamo po ena"cbi (\ref{eq:kot-diskretno}), edina neznanka je $\phi_{N-1}$, ki ga enostavno izrazimo iz zgorjnje ena"cbe. 

\section{Rezultati}

Nihanje vrvice je najla"zje prikazat z animacijo. Re"sitvi prilagam tri animacije, ki se razlikujejo po za"cetnemu kotu vrvice. V vseh treh primerih sem uporabil simuliral nihanje vrvice s 100 odseki in "casovnim korakom $10^{-5}$ v brezdimenzijskih enotah. V eni sekundi simulacije je ravno $10^5$ korakov, kar ustreza nihanju vrvice s karakteristi"cnim "casom $\tau = 1\mathrm{s}$. 

Poleg polo"zaja vrvi sem posku"sal prikazati tudi silo v vrvi. Ta je v animacija ozna"cena z barvo. Rde"ca barva ustreza sili $F=0$, medtem ko rumena pomeni silo $F=1$. Zaradi centripetalne sile lahko sila na za"cetku vrvi tudi prese"ze vrednost 1, zato sem prispevek ustrezno skaliral. Izkazalo se je, da pri izbranih za"cetnih pogojih sila nikoli ne prese"ze vrednosti 5/4, tako da sem za barvne komponente RGB izbral (255, 200*F, 0). 

\section{Energija vrvi}

Ker shema integracije sama po sebi ne ohranja skupne energije vrvi, je spreminjanje energije s "casom primeren kriterij za dolo"canje natan"cnosti. V nihanjo"ci vrvi imamo kineti"cno in potencialno energijo, medtem ko se pri neraztegljivi vrvice pro"znostna ne spreminja. Kineti"cno energijo vsakega odseka lahko razdelimo na translacijsko in rotacijsko. 

Skupno energijo enega odseka vrvi lahko zapi"semo kot

\begin{align}
 E_i = E_{trans} + E_{rot} + E_{pot} = \frac{m_i}{2} v_i^2 + \frac{m_i l_i^2}{24} \dot{\phi}^2- m_i g y_i
\end{align}

Hitrost $v_i$ in polo"zaj $y_i$ sta odvisni le od $\phi$ in $\dot\phi$, zato za izra"cun ne bomo potrebovali sile. Navpi"cno komponentno polo"zaja lahko izrazimo enostavno, s se"stevanjem navpi"cnih komponent smeri vseh prej"snjih odsekov. Ker v izrazu za potencialno energijo nastopa polo"zaj te"zi"s"ca odsekov, $i$-ti "clen upo"stevamo le polovi"cno. Podobno lahko izra"cunamo tudi vodoravno komponento $x$, ki jo bomo potrebovali pri ra"cunu hitrosti. 

\begin{align}
 y_i &= \sum_{m=0}^{i-1} h \sin \phi_m + \frac{1}{2}h\sin\phi_i \\
 x_i &= \sum_{m=0}^{i-1} h \cos \phi_m + \frac{1}{2}h\cos\phi_i
\end{align}

Hitrost lahko izra"cunamo kot odvod polo"zaja s kon"cno diferenco. La"zje kot samo hitrost je zapisati njen kvadrat

\begin{align}
 v_i^2 = \frac{\left(y_i(t) - y_i(t-k)\right)^2 + \left(x_i(t) - x_i(t-k)\right)^2}{k^2}
\end{align}

"Casovna odvisnost $y_i$ je izra"zena le prek "casovne odvisnosti $\phi$. Ker program vedno shranjuje $\phi$ ob dveh zaporednih korakih, lahko hitrost izra"cunamo po vsakem koraku. 

Izraza za polo"zaj $i$-tega odseka vrvi lahko prepi"semo v rekurzivno formulo

\begin{align}
 (x_0, y_0) &= \frac{h}{2}(0, 0) \\
 (x_i, y_i) &= (x_{i-1}, y_{i-1}) + \frac{h}{2} (\cos\phi_{i-1} + \cos\phi_i, \sin\phi_{i+1} + \sin\phi_i)
\end{align}

po kateri lahko celotno energijo izra"cunamo z enim prehodom po podatki in v linearnem "casu. 

\subsection{Za"cetno stanje}

"Ce ra"cunamo energijo nihanja na za"cetku, po enem ali dveh "casovnih korakih, naletimo na te"zavo. Stanje vrvi, ki ga opisuje za"cetni pogoj, ni fizikalno. Ravna vrv nikakor ne zadostuje pogoju (\ref{eq:kot-zacetni-pogoj}), saj je $\parcialno{\phi}{s}=0$ sila kon"cna, kosinus kota pa neni"celen. 

Za izvajanje simulacije to ni velika ovira. Podobne primere smo re"sevali "ze ve"ckrat, med drugim tudi pri prej"snji nalogi, ko je imelo dno valja druga"cno temperaturo kot pla"s"c, torej rob valja ni mogel ustrezati obema pogojema. Vseeno pa smo ena"cbo lahko re"sili brez velikih ovir. Situacija je tu malo druga"cna, saj je eden izmed pogojev robni, drugi pa za"cetni. Za re"sevanje je to morda celo prednost; "Casovni razvoj re"sitve "ze sam poskrbi, da vrvica hitro preide v fizikalno stanje, kjer velja robni pogoj (\ref{eq:kot-zacetni-pogoj}). V tem prehodnem obdobju se zgodi hitra sprememba kota prvega odseka vrvi, kar posledi"cno premakne celotno vrv in prinese zelo visoko kineti"cno energijo. 

\begin{figure}[H]
 \input{g_energija_zacetek}
 \caption{Spreminjanje skupne kineti"cne energije takoj po za"cetku simulacije. Viden je skok na za"cetku, vrnitev k pri"cakovani vrednosti, nato pa po"casno nara"s"canje, ki je posledica pretvorbe potencialne energije v kineti"cno. }
 \label{fig:energija-zacetek}
\end{figure}

Na sre"co pa je to prehodno obdobje kratko, pri dovolj fini diskretizaciji se fizikalno stanje vzpostavi "ze po treh "casovnih korakih. Takrat v nobeni ena"cbe ne nastopa ve"c eksplicitno za"cetni pogoj. Za bolj grobe diskretizacije ($J < 100$) vzpostavitev robnega pogoja traja dlje, ampak tudi pri $J=10$ traja le 6 korakov. 

Ohranitev energije sem preverjal tako, da sem kot za"cetno stanje vzel energijo vrvi po 10 korakih. 

\subsection{"Stevilo odsekov}

Najprej sem opazoval, kaj se dogaja z ohranitvijo energije pri razli"cnih "stevilih odsekov. Rezultat je na sliki ~\ref{fig:energija-odseki}. 

\begin{figure}[H]
 \input{g_energija}
 \caption{Odstopanje energije pri razli"cnih "stevilih odsekov vrvi in "casovnim korakom $k = 10^{-5}$}
 \label{fig:energija-odseki}
\end{figure}

Po pri"cakovanju finej"sa delitev vrvi povzro"ci manj"se odstopanje energije. U"cinek ve"canja "stevila odsekov je bolje viden na logaritemskem grafu na sliki~\ref{fig:energija-odseki-log}. Tu vidimo, da je povpre"cna napaka pribli"zno sorazmerna z dol"zino enega odseka. 

\begin{figure}[H]
 \input{g_energija_log}
 \caption{Odstopanje energije pri razli"cnih "stevilih odsekov vrvi in "casovnim korakom $k = 10^{-5}$}
 \label{fig:energija-odseki-log}
\end{figure}

\subsection{"Casovni korak}
\label{sec:casovni-korak}

Logi"cno se zdi, da na odstopanje energije vpliva tudi velikost "casovnega koraka. 

\begin{figure}[H]
 \input{g_energija_korak}
 \caption{Odstopanje energije pri razli"cnih "casovnih korakih in delitvijo na $J=100$ odsekov}
 \label{fig:energija-korak}
\end{figure}

Vidimo, da dol"zina koraka nima velikega vpliva na spremembe energije. Izjema so le koraki, dalj"si od $10^{-3}$. V tistem primeru se integracijska shema podre, saj v za"cetnih korakih prekora"cimo meje strojne natan"cnosti. Dodatno zmanj"sevanje koraka pod to mejo pa ima veliko manj"si vpliv na napako energije kot pove"cevanje "stevila odsekov vrvi. 

\end{document}
