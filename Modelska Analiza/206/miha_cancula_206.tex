\documentclass[a4paper,10pt]{article}

\usepackage[utf8]{inputenc}
\usepackage[slovene]{babel}
\usepackage{amsmath}
\usepackage{amsfonts}
\usepackage{relsize}
\usepackage[smaller]{acronym}
\usepackage{graphicx}
\usepackage{subfigure}
\usepackage{cite}
\usepackage{url}
\usepackage[unicode=true]{hyperref}
\usepackage{color}
\usepackage[version=3]{mhchem}
\usepackage{wrapfig}
\usepackage{comment}
\usepackage{float}
\usepackage[top=3cm,bottom=3cm,left=3cm,right=3cm]{geometry}

\renewcommand{\vec}{\mathbf}
\newcommand{\eps}{\varepsilon}
\renewcommand{\phi}{\varphi}
\renewcommand{\theta}{\vartheta}
\newcommand{\dd}{\mathrm{d}}

\newcommand{\parcialno}[2]{
  \frac{\partial #1}{\partial #2}
}
\newcommand{\parcdva}[2]{
  \frac{\partial^2 #1}{\partial #2 ^2}
}
\newcommand{\lag}{\mathcal{L}}

\title{Parcialne diferencialne ena\v cbe \\ Robni problemi, relaksacija}
\author{Miha \v Can\v cula}
\begin{document}

\maketitle

\section{Postopek re"sevanja}

Geometrije nalogo sem pribli"zal z diskretno mre"zo z $N$ to"ckami na vsakem robu. Poljubno stanje opne je tedaj vektor dimenzije $N^2$, operator $\nabla^2$ pa je predstavljen z matriko $A$ dimenzije $N^2 \times N^2$. Diskretizacijo sem izbral tako, da so bile skrajne to"cke za polovico koraka oddaljene od fiksnega roba opne. Na na na"cin je bila matrika $A$ simetri"cna. 

Kvadrati lastnih frekvenc so lastne vrednosti matrike $A$, nihajni na"cini pa ustrezni lastni vektorji. V prvem delu, ko sem obravnaval opno z neenakomerno maso, sem re"seval posplo"sen problem lastnih vrednosti $\vec A\vec u = \lambda \vec B \vec u$. Vse izra"cune sem izvedel v programu Octave s pomo"cjo funkcije \texttt{eigs}, ki uporablja Fortranovo knji"znico \texttt{ARPACK}. Ta knji"znica omogo"ca u"cinkovito re"sevanje problemov lastnih vrednosti velikih strukturiranih matrik. Matriki $A$ in $B$ sta zelo prazni, zato sem zanju uporabil ustrezno redko (\textit{sparse}) predstavitev. Kljub temu mi je uspelo re"sitiv problem v doglednem "casu le do $N = 64$. 

\section{Neenakomerna kvadratna opna}

\begin{figure}[H]
  \subfigure[$k_1 = 0.058841$]{\includegraphics[width=.5\textwidth]{g_opna_2_1}}
  \subfigure[$k_2 = 0.088345$]{\includegraphics[width=.5\textwidth]{g_opna_2_2}}
  \subfigure[$k_3 = 0.092194$]{\includegraphics[width=.5\textwidth]{g_opna_2_3}}
  \subfigure[$k_4 = 0.11387$]{\includegraphics[width=.5\textwidth]{g_opna_2_4}}
  \subfigure[$k_5 = 0.12178$]{\includegraphics[width=.5\textwidth]{g_opna_2_5}}
  \subfigure[$k_6 = 0.12275$]{\includegraphics[width=.5\textwidth]{g_opna_2_6}}
\caption{Prvih 6 nihajnih na"cinov kvadratne opne z $\rho_+ = 2\rho$. }
\end{figure}

\begin{figure}[H]
  \subfigure[$k_1 = 0.058841$]{\includegraphics[width=.5\textwidth]{g_opna_05_1}}
  \subfigure[$k_2 = 0.088345$]{\includegraphics[width=.5\textwidth]{g_opna_05_2}}
  \subfigure[$k_3 = 0.092194$]{\includegraphics[width=.5\textwidth]{g_opna_05_3}}
  \subfigure[$k_4 = 0.11387$]{\includegraphics[width=.5\textwidth]{g_opna_05_4}}
  \subfigure[$k_5 = 0.12178$]{\includegraphics[width=.5\textwidth]{g_opna_05_5}}
  \subfigure[$k_6 = 0.12275$]{\includegraphics[width=.5\textwidth]{g_opna_05_6}}
\caption{Prvih 6 nihajnih na"cinov kvadratne opne z $\rho_+ = \rho/2$. }
\end{figure}

\section{Polkro"zna opna}

\begin{figure}[H]
  \subfigure[$k_1 = 0.058841$]{\includegraphics[width=.5\textwidth]{g_valj_1}}
  \subfigure[$k_2 = 0.088345$]{\includegraphics[width=.5\textwidth]{g_valj_2}}
  \subfigure[$k_3 = 0.092194$]{\includegraphics[width=.5\textwidth]{g_valj_3}}
  \subfigure[$k_4 = 0.11387$]{\includegraphics[width=.5\textwidth]{g_valj_4}}
  \subfigure[$k_5 = 0.12178$]{\includegraphics[width=.5\textwidth]{g_valj_5}}
  \subfigure[$k_6 = 0.12275$]{\includegraphics[width=.5\textwidth]{g_valj_6}}
\caption{Prvih 6 nihajnih na"cinov polkro"zne opne.}
\end{figure}

\end{document}
