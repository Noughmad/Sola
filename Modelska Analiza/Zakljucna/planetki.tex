\documentclass[a4paper,10pt]{article}

\usepackage[utf8]{inputenc}
\usepackage[slovene]{babel}
\usepackage{amsmath}
\usepackage{amsfonts}
\usepackage{relsize}
\usepackage[smaller]{acronym}
\usepackage{graphicx}
\usepackage{subfigure}
\usepackage{cite}
\usepackage{url}
\usepackage[unicode=true]{hyperref}
\usepackage{color}
\usepackage[version=3]{mhchem}
\usepackage{wrapfig}
\usepackage{comment}
\usepackage{float}
\usepackage[top=3cm,bottom=3cm,left=3cm,right=3cm]{geometry}

\renewcommand{\vec}{\mathbf}
\newcommand{\eps}{\varepsilon}
\renewcommand{\phi}{\varphi}
\renewcommand{\theta}{\vartheta}
\newcommand{\dd}{\,\mathrm{d}}

\newcommand{\parcialno}[2]{
  \frac{\partial #1}{\partial #2}
}
\newcommand{\parcdva}[2]{
  \frac{\partial^2 #1}{\partial #2 ^2}
}
\newcommand{\lag}{\mathcal{L}}

\newcommand{\drug}[1]{
  \vec{\ddot{#1}}
}

\title{Potovanje sonde med planeti}
\author{Miha \v Can\v cula}
\begin{document}

\maketitle

\section{Zapis problema}

\subsection{Newtonov zakon}
Sonda se po son"cnem sistemu giblje, kot to dolo"ca Newtonov zakon

\begin{align}
  \vec{\ddot r}= \frac{\vec F}{m_s}
\end{align}

Nanjo delujejo tri sile, to so gravitacijski privlaki sonca in obeh planetov. 

$$ \vec F = Gm_s \left( \frac{-M\vec r}{r^3} + \frac{m(\vec r_1 - \vec r)}{|\vec r_1 - \vec r|^3} + \frac{m(\vec r_2 - \vec r)}{|\vec r_2 - \vec r|^3} \right) $$

Masa sonde se po pri"cakovanju kraj"sa, ostanejo pa nam "se gravitacijska konstanta in mase vseh treh nebesnih teles. Poznamo pa tudi gibanje planetov, saj kro"zita okrog zvezde. Njun radialni pospe"sek je enak

\begin{align}
\label{eq:gibanje-planetov}
\vec{\ddot r_i} = -\omega^2\vec r_i = GM \frac{-\vec r_i}{r_i^3}
\end{align}

Imamo torej zvezo med maso sonca $M$, gravitacijsko konstanto $G$, polmerom orbite planeta $r_i$ ($i$ je 1 ali 2) in frekvenco kro"zenja $\omega$. 

\subsection{"Stevilo parametrov}
Konstanta $G$ nastopa povsod kot multiplikativna konstanta k drugemu "casovnemu odvodu. Izbira njene vrednosti je zato enakovredna izbiri "casovne skale $t \to t/\sqrt{G}$. Enak u"cinek ima pove"canje ali zmanj"sanje mase sonca in planetov za enak faktor. Vrednosti za $G$ in $M$ lahko torej postavimo na 1, namesto mase planeta $m$ pa ra"cunamo z razmerjem $\mu = m/M$. Ena"cbo gibanja planetov re"simo enostavno, tako da zanemarimo medsebojni vpliv in privzamemo kro"zno gibanje. Iz zveze (\ref{eq:gibanje-planetov}) izrazimo kro"zno frekvenco kot $\omega^2 = 1/r^3$. 

\begin{align}
\drug r &= -\frac{\vec r}{r^3} + \mu\frac{\vec{r_1-r}}{|\vec{r_1-r}|^3} + \mu\frac{\vec{r_2-r}}{|\vec{r_2-r}|^3} \label{eq:sonda} \\
\vec r_i(t) &= \omega_i^{-2/3} \left[ \cos(\omega_i t + \phi_i), \sin(\omega_i t + \phi_i) \right]^T
\end{align}

V nalogi je podano razmerje polmerov orbit obeh planetov $r_2 = 2r_i$. Po drugem Keplerjevem zakonu lahko to pretvorimo v razmerje kro"znih frekvenc

\[ \omega_1 = \sqrt{8}\omega_2 \approx 2.828 \omega_2 \]

Nazadnje preverimo "se, ali lahko dolo"cimo tudi vrednosti za $r_1$ in $r_2$. "Ce vse razdalje pove"camo za faktor $k$, se kro"zni frekvenci pomno"zita s $\sqrt{k^{-3}}$, sile na sondo s $k^{-2}$. "Ce spet reskaliramo "casovno skalo kot $t \to t\sqrt{k^3}$, se vsi pospe"ski mno"zijo s $k \cdot \sqrt{k^{-3}} \cdot \sqrt{k^{-3}} = k^{-2}$, kar se ujema z izrazom za silo. Torej oblika re"sitve ni odvisna od prostorske skale, in lahko postavimo $r_1 = 1$ in $\omega_1 = 1$. O"citno je, da problem ni odvisen od orientacije koordinatnega sistema, zato lahko enega izmed kotov $\phi_i$, na primer $\phi_1$, postavimo na 0. 

\begin{align}
 \vec r_1(t) &= \left[ \cos t, \sin t \right]^T \\
 \vec r_2(t) &= 1/\sqrt{8} \left[ \cos(\sqrt{8} t + \delta), \sin(\sqrt{8} t + \delta) \right]^T
\end{align}

V sistemu ena"cb sta ostala le "se dva neodvisna parametra: relativna masa obeh planetov $\mu$ in fazni zamik med orbitama $\delta$. Ostale koli"cine so dolo"cene s Keplerjevim zakonov in z izbiro "casovne skale. 

\subsection{Robni pogoji}

Koordinatni sistem postavimo tako, da je sonce v sredi"s"cu. Ob "casu $t=0$ lahko brez izgube splo"snosti privzamemo, da se prvi (notranji) planet nahaja na osi $x$, torej je njegov kot v polarnih koordinatah enak 0. Za drugi planet tega ne moremo privzeti, saj ima lahko fazni zamik $\delta$. Oba planeta se gibljeta po kro"znih orbitah, torej lahko njun polo"zaj po poljubnem "casu v polarnih koordinatah zapi"semo kot

\begin{align}
 \vec r_1(t) &= (r_1, t) \\
 \vec r_2(t) &= (r_2, \sqrt{8}t + \delta)
\end{align}

"Ce sonda za potovanje med planetoma potrebuje "cas $T$, se njuna robna pogoja glasita

\begin{align}
 \vec r(0) &= (r_1, 0) \\
 \vec r(T) &= (r_2, \sqrt{8} T + \delta)
\end{align}

Poznamo silo na sondo ob vsakem "casu, torej je ena"cba drugega reda, imamo pa tudi dva robna pogoja. Prost pa je "se parameter $T$, torej bomo za pot med dvema planetoma verjetno na"sli ve"c re"sitev. Zato bomo lahko omejili "cas potovanja $T$, ali pa za"cetno in kon"cno hitrost sonde, da bo re"sitev "se vedno obstajala. 

\subsection{Vrednotenje re"sitve}

Najbolj ``ugodna'' orbita za vesoljska plovila je tak"sna, kjer potovanje traja "cim manj, hrati pa sta za"cetna in kon"cna hitrost (relativno na planet) dovolj majhni, da plovilo lahko varno pristane. Potrebi po kratkotrajnem poletu je vsaj teoreti"cno enostavno ugoditi, saj lahko po"sljemo sondo z neskon"cno hitrostjo. "Zal pa poleg tehni"cnih te"zav na ta na"cin tudi uni"cimo sondo ob pristanku. 
Minizimacija za"cetne in kon"cne hitrosti je bolj zahtevna, tudi "ce si dovolimo dolgotrajno pot. Sonda se ne sme gibati enako hitro kot planet, saj bi v tem primeru kro"zila z njim. Vsaj v primeru $\mu \ll 1$ pa lahko skonstruiramo elipso z minimalno ekscentri"cnostjo, ki je tangentna na orbiti obeh planetov, tako da sta za"cetna in kon"cna relativna hitrost vedno v tangentni smeri glede na sonce. 
Njena perioda je v iracionalnem razmerju s periodo kro"zenja drugega planeta, zato se bosta slej ko prej sre"cala. Tak"sno orbito imenujemo Hohmannova prenosna orbita, uporabna pa je predvsem za spremebme orbit satelitov. Po tak"sni orbiti lahko telo po"sljemo z enega planeta na drugega le pri dolo"cenem faznem zamiku med orbitama planetov, torej le v ozkih izstrelitvenih oknih (launch windows). 

Izku"sje z medplanetarnimi poleti s "clove"sko posadko "zal "se nimamo, zato sem predpostavil da je sonda robotska in lahko zdr"zi tudi ve"cleten polet po son"cnem sistemu. Po drugi strani pa ima omejeno zalogo goriva, torej sta za"cetno pospe"sevanje in kon"cno zaviranje omejena. To lahko pri simulaciji upo"stevamo tako, da omejimo za"cetno in kon"cno hitrost, mogo"ce celo 
vsoto velikosti hitrosti glede na za"cetni in kon"cni planet, $|\delta v_1| + |\delta v_2| \leq C$. S tak"sno omejitvimo trdimo, da je sonda zmo"znega hitrega pospe"sevanja (hitrega v primerjavi s trajanjem poleta), omejujo"c faktor je le zaloga goriva. 

Za trajanje potovanja $T$ nimamo tako stroge omejitve, koristno je le, da najdemo re"sitev s "cim manj"sim $T$. 

\section{Re"sevanje}

\subsection{Pribli"zek z $\mu \ll 1$}
Analiti"cno re"sevenaje celotnega sistem je prezahtevno. Lahko pa to"cno re"simo primer, ko sta masi planetov zanemarljivi v primerjavi z maso zvezde. To je upravi"cen pribli"zek, v primeru Sonca, Zemlje in Marsa je vrednost reda velikosti $\mu \approx 10^{-6}$, z upo"stevanjem Jupitra in Saturna pa zraste na $\mu \approx 10^{-3}$. 

"Ce zanemarimo gravitacijska vpliva obeh planetov, dobimo Keplerjev problem, za katerega vemo, da je re"sitev gibanje sonde po elipsi s Soncem z gori"s"cu. Keplerjev problem s podanimi robnimi pogoji $\vec r(t_1) = \vec r_1$ in $\vec r(t_2) = \vec r_2$ se imenuje Lambertov problem\cite{wiki:lambert}. Geometrijsko je problem ekvivalenten iskanju elipse z gori"s"cem v Soncu, ki gre skozi oba robna polo"zaja planetov. Z enostavno geometrijo se da poka"zati, da mora drugo gori"s"ce elipse le"zati na enem kraku hiperbole, torej obstaja enoparametri"cna dru"zina re"sitev. 

Tako "cas potovanja kot za"cetna in kon"cna hitrost sonde so odvisni od izbrite drugega gori"s"ca. Ker je od "casa potovanja odvisen tudi kon"cni polo"zaj drugega planeta, sem najprej izbral vrednost za "cas potovanja $T$, nato na"sel elipti"cno orbito sonde s tak"snim "casom potovanja in zapisal potrebne spremebme hitrosti za to orbito. Za iskanje parametrov elipse sem uporabil orodje \texttt{kepler\_toolbox}, ki so ga razvili v Evropski vesoljski agenciji ESA. Orodje vrne za"cetno hitrost sonde, s pomo"cjo katere sem rekonstruiral celotno orbite z navadno integracijo po metodi RK4. 

Lambertov problem s fiksnim "casom potovanja ima vsaj eno re"sitev, ki pa ni vedno enoli"cna. Pri dovolj dolgih "casih potovanja so mo"zno tudi orbite, kjer sonda naredi enega ali ve"c obratov okrog sonca preden pristane na drugem planetu. 

\subsection{Poljuben $\mu$}

Ker v znanih son"cnih sistemih masa sonca mo"cno presega mase planetov, si lahko pri iskanju splo"sne re"sitve pomagamo s prej"snjim pribli"zkom. Za iskanje orbit zato nisem uporabil strelske metode z ugibanjem za"cetnih pogojev in numeri"cno integriranjem. Namesto tega sem za"cel z elipti"cno orbito in jo relaksiral, tako da je v vsaki to"cki ustrezala Newtonovem zakonu. Tak"sna relaksacija seveda ni fizikalna, se pa je izkazala za u"cinkovit ra"cunski pripomo"cek. 

Trajektorija sonde mora v vsaki to"cki zado"s"cati pogoju (\ref{eq:sonda}). Drugi odvod na levi strani ena"cbe sem pribli"zal s kon"cno diferenco, nato pa z metodo pospe"sene relaksacije SOR popravljal $r(t_i)$, dokler pogoj ni bil izpolnjen v vsaki to"cki.

Velika prednost pristopa z relaksacijo je ta, da lahko za"cetno in kon"cno to"cko postavimo to"cno na polo"zaj ustreznega planeta. Na ta na"cin lahko zadenemo tudi planete, ki so dosti manj"si od polmera svojega tira okrog sonca, kar definitivno dr"zi za planetev v na"sem oson"cju. Na ta na"cin re"sitev sploh ni odvisna od polmera planeta, kar nam zmanj"sa "stevilo pogojev in s tem poenostavi re"sevanje. 

\subsection{Relaksacija}

Tak"sna relaksacija zahteva "casovno diskretizacijo tira. Ni treba, da je ta diskretizacija enakomerna, moramo pa v naprej dolo"citi "stevilo in dol"zino "casovnih korakov, torej moramo izbrati tudi skupen "cas $T$. S to izbiro tedaj najdemo ustrezno orbito in izra"cunamo za"cetno in kon"cno relativno hitrost. 

\begin{figure}[H]
\centering
 \input{g_lambert_mu}
 \caption{Skupna sprememba hitrosti, potrebna za doseg drugega planeta in varno zaustavitev, pri razli"cnih masah planetov ($\delta = 0$)}
 \label{fig:hitrost}
\end{figure}

S slike lahko razberemo, da ve"cja masa planetov terja mo"cnejse pospe"sevanje in zaviranje sonde. To sledi iz dejstva, da moramo tako pri vzletu kot pri pristanku premagovati ve"cjo silo te"znosti. 

Drugo zanimivost je periodi"cnost grafa. Zunanji planet kro"zi s periodo $t_2 = 2\pi\sqrt{8} \approx 17,8$, kar se ujema z razmikom med vrhovi na grafu. Slika \ref{fig:hitrost} je narejena za fazni zamik $\delta=0$, torej so ob "casu izstrelitve sonce in oba planeta na isti premici. Pri tak"sni konfiguraciji je energijsko najugodnej"sa pot, ki traja malo manj kot celi ve"ckratnik periode zunanjega planeta. 

"Ce spremenimo fazni zamik med planetoma, kar je enakovredno izstrelitvi ob drugem "casu, se celoten graf ustrezno premakne. Primer z $\delta = 2\pi/3$ je na sliki \ref{fig:hitrost-2}. 

\begin{figure}[H]
\centering
 \input{g_lambert_mu_23}
 \caption{Skupna sprememba hitrosti, potrebna za doseg drugega planeta in varno zaustavitev, pri razli"cnih masah planetov ($\delta = 2\pi/3$)}
 \label{fig:hitrost}
\end{figure}

Vidimo podobno periodi"cnost kot prej, le da so vsi vrhovi premaknjeni v levo. 

\begin{figure}[H]
\centering
 \input{g_lambert_mu_43}
 \caption{Skupna sprememba hitrosti, potrebna za doseg drugega planeta in varno zaustavitev, pri razli"cnih masah planetov ($\delta = 4\pi/3$)}
 \label{fig:hitrost-2}
\end{figure}

Zanimiv vzorec pa nastane, ko skupaj nari"semo odvisnost $\Delta v$ od $T$ za razli"cne zamike $\delta$ pri istem $\mu$. Ta primer je bolj uporaben, saj mas planetov "se ne znamo poljubno spreminjati, medtem ko datum vzleta in s tem fazni zamik planetov lahko izberemo. Ra"cun sem napravil le za tri mo"zne zamike, ki so med seboj razmaknjeni za tretjino polnega kota. Minimalen $\Delta v$ pri poljubnem "casu poleta $T$ pa lahko interpoliramo z grafa s pomo"cjo ogrinja"ce, ki povezuje najni"zje to"cke na grafu.  

\begin{figure}[H]
\centering
 \input{g_lambert_delta}
 \caption{Skupna sprememba hitrosti, potrebna za doseg drugega planeta in varno zaustavitev, pri faznih zamikih $\delta$ in pri relativni masi planetov $\mu=0,\!01$}
 \label{fig:hitrost-delta}
\end{figure}

Ena"cba ogrinja"ce na zgornji sliki je $\Delta v = 1,\!3 + 13/(T-3,\!1)$. Ta zveza je monotono padajo"ca, torej si s podalj"sevanjem "casa potovanja lahko privo"s"cimo manj pospe"sevanja. Poleg tega pa obstaja spodnja meja za potrebno hitrost, ki je odvisna le od mase planetov. 

\begin{figure}[H]
\centering
 \input{g_lambert_delta_2}
 \caption{Skupna sprememba hitrosti, potrebna za doseg drugega planeta in varno zaustavitev, pri faznih zamikih $\delta$ in pri relativni masi planetov $\mu=0,\!02$}
 \label{fig:hitrost-delta}
\end{figure}

\begin{figure}[H]
\centering
 \input{g_lambert_delta_10}
 \caption{Skupna sprememba hitrosti, potrebna za doseg drugega planeta in varno zaustavitev, pri faznih zamikih $\delta$ in pri relativni masi planetov $\mu=0,\!1$}
 \label{fig:hitrost-delta}
\end{figure}

\begin{thebibliography}{3}
 \bibitem{wiki:lambert} Wikipedia: Lambert's problem (\url{http://en.wikipedia.org/wiki/Lambert's_problem})
 \bibitem{wiki:kepler} Wikipedia: Kepler's laws of planetary motion (\url{http://en.wikipedia.org/wiki/Kepler's_laws_of_planetary_motion})
\end{thebibliography}


\end{document}
