\documentclass[a4paper,10pt]{article}

\usepackage[utf8x]{inputenc}
\usepackage[slovene]{babel}
\usepackage{graphicx}
\usepackage{amsmath}
\usepackage{subfigure}

\title{Nelinearna minimizacija}
\author{Miha \v Can\v cula}

\renewcommand{\phi}{\varphi}
\renewcommand{\epsilon}{\varepsilon}
\renewcommand{\theta}{\vartheta}

\newcommand{\naboji}[1]{
  
\begin{figure}
  \subfigure{\input{naboji-#1-tloris}}
  \subfigure{\input{naboji-#1-stran}}\\
  \subfigure{\input{naboji-#1-pers1}}
  \subfigure{\input{naboji-#1-pers2}}
\end{figure}

}

\begin{document}

\maketitle

\section{Orodja}
Nalogo sem re"seval s prostim programom \texttt{GNU Octave}, ki za nelinearno optimizacijo ponuja funkcijo \texttt{sqp}. 

\section{Thompsonov problem}
Polo"zaj diskretnih to"ckastih nabojev na krogli lahko opi"semo z dvema koordinatama, $\theta\in[0,\pi]$ in $\phi\in[0,2\pi]$. Potencialna energija takih nabojev je odvisna le od medsebojne razdalje, zato moramo najprej izraziti razdaljo med dvema nabojema z njunima koordinatama. Kot med dvema to"ckama izra"cunamo kot skalarni produkt med njunima krajevnima vektorjema in znasa
\begin{align}
  \cos \alpha_{ij} &= \cos (\phi_i-\phi_j) \sin \theta_i \sin \theta_j + \cos \theta_i \cos\theta_j
\end{align}
dejansko razdaljo med to"ckama pa po kosinusnem izreku
\begin{align}
  d_{ij} &= \sqrt{2R^2(1 - \cos\alpha_{ij})}
\end{align}

Skupna potencialna energija sistema je vsota potencialnih energij vseh parov nabitih delcev na krogli
\begin{align}
  E = E_0 \sum_{i<j} \frac{\sqrt{2R^2}}{d_{ij}} = E_0 \sum_{i<j} \left[ 1 - \cos (\phi_i-\phi_j) \sin \theta_i \sin \theta_j - \cos \theta_i \cos\theta_j \right]^{-1/2} 
\end{align}

Ker multiplikativna konstanta v energiji ne spremeni optimalne razporeditve nabojev po krogli, lahko izraz pretvorim v brezdimenzijske enote. Energija $E_0 = \frac{e^2}{4\pi\varepsilon_0 R}$ nas bo zanimala le, "ce bomo "zeleli izra"cunati vrednosti energije v minimumu, ne pa samega polo"zaja minimuma. 
\begin{align}
  y = \sum_{i<j} \left[ 1 - \cos (\phi_i-\phi_j) \sin \theta_i \sin \theta_j - \cos \theta_i \cos\theta_j\right]^{-1/2} 
\end{align}

Brez izgube splo"snosti lahko fiksiramo tri koordinate: $\theta_1 = 0$, $\phi_1 = 0$, $\phi_2 = 0$. Na ta na"cim prepre"cimo vrtenje okrog krogle med re"sevanje, hkrati pa pospe"simo re"sevanje, saj zmanj"samo "stevilo prostostnih stopenj. Za problem z $N$ naboji nam ostane $2N-3$ prostih koordinat. 

\naboji{10}
\naboji{5}

\section{Voznja do semaforja}
Na enak na"cin sem re"seval tudi problem iz prej"snje naloge. "Casovni interval sem razdelil na $N$ podintervalov, hitrost vo"znje sem omejil med 0 in $y_{max}$ in dodal omejitev, da je skupna pot enaka 1. Upo"steval sem tudi asimetri"cno omejitev pospe"ska $a_{min} < a < a_{max}$, kjer je $a_{min}<0$ najve"cji pojemek pri zaviranju in $a_{max} > 0$ najve"cji pospe"sek pri pospe"sevanju. 

Izbral sem metodo, ki je sama dolo"cila smer najugodnej"sega premikanja, brez da bi ji moral podati gradient akcije. Zato ni bilo potrebe po analiti"cnosti funkcije in sem lahko uporabil ostro omejitev tako da hitrost kot za pospe"sek. 

V primerjavi s prej"snjo nalogo dodatne omejitve prinesejo kar nekaj novih parametrov, zato sem narisal ve"c grafov. 

\end{document}
