\documentclass[a4paper,10pt]{article}

\usepackage[utf8]{inputenc}
\usepackage[slovene]{babel}
\usepackage{amsmath}
\usepackage{amsfonts}
\usepackage{relsize}
\usepackage[smaller]{acronym}
\usepackage{graphicx}
\usepackage{subfigure}
\usepackage{cite}
\usepackage{url}
\usepackage[unicode=true]{hyperref}
\usepackage{color}
\usepackage[version=3]{mhchem}
\usepackage{wrapfig}
\usepackage{comment}
\usepackage{float}
\usepackage[top=3cm,bottom=3cm,left=3cm,right=3cm]{geometry}

\renewcommand{\vec}{\mathbf}
\newcommand{\eps}{\varepsilon}
\renewcommand{\phi}{\varphi}
\renewcommand{\theta}{\vartheta}
\newcommand{\dd}{\mathrm{d}}

\newcommand{\parcialno}[2]{
  \frac{\partial #1}{\partial #2}
}
\newcommand{\parcdva}[2]{
  \frac{\partial^2 #1}{\partial #2 ^2}
}
\newcommand{\lag}{\mathcal{L}}

\title{Parcialne diferencialne ena\v cbe \\ Robni problemi, relaksacija}
\author{Miha \v Can\v cula}
\begin{document}

\maketitle

\section{Metoda}

Problem sem re"seval z diskretno mre"zo. Ker imata obe nalogi zrcalno simetrijo, sem diskretizacijo dolo"cil tako, so bile to"cke na vseh robovih in tudi na simetrijski ravnini. Celotno stanje, vklju"cno z robnimi to"ckami, sem predstavil z $(N+1) \times (N/2+1)$ to"ckami. Na ta na"cin sem lahko zagotovil, da je bila razdalja med to"ckami enaka v obeh dimenzijah. 

Iteracijo sem izvajal tako, da sem v spominu dr"zal le eno kopijo celotnega polja, torej po Seidlovem postopku. Za hitrej"so konvergenco sem vsak korak izvedel na lihih in sodih to"ckah lo"ceno. 

Postopek sem prekinil, ko je povpre"cni kvadrat popravkov dosegel dovolj majhno vrednost. 

\section{"Cebi"sev pospe"sek}

Hitrost konvergence sem meril na primeru cevi iz prve naloge. Najprej sem izra"cunal najugodnej"so vrednost $\omega$. Privzel sem, da je ta vrednost med 1 in 2, nato pa optimalno vrednost na"sel z minimizacijo "stevila korakov. Rezultati so na Sliki~\ref{fig:omega-min}. 

\begin{figure}[H]
  \input{g_alpha}
  \caption{Optimalen parameter $\alpha$ za cev z danim prerezom}
  \label{fig:omega-min}
\end{figure}

Dobljena vrednost za $\alpha$ je malo manj"sa od enice, ki bi jo pri"cakovali za kvadratno cev. 

Pri razli"cnem "stevilu to"ck sem tudi opazoval odvisnost "stevila potrebnih korakov od izbrane $\omega$. Pri tem en korak pomeni popravek vrednosti funkcije $u$ v vseh to"ckah. Odvisnost je prikazana na Sliki~\ref{fig:omega}. 

\begin{figure}[H]
  \input{g_omega}
  \caption{Odvisnost "stevila potrebnih korakov od parametra $\omega$}
  \label{fig:omega}
\end{figure}

V obeh primerih vidimo najprej po"casno padanje, nato oster minimum in hitro nara"s"canje, ko se $\omega$ bli"za 2. "Ce dose"ze 2, postopek ne konvergira ve"c. Pri majhnih $\omega$ simulacija z ve"cjim "stevilo to"ck potrebuje tudi ve"cje "stevilo korakov, medtem ko za velike $\omega$ razlike ni ve"c. 

\section{Pretok po cevi}

Tu re"sujemo Poissonovo ena"cbo $\nabla^2 u = \rho$, kjer $\rho$ predstavlja tla"cno razliko med koncema cevi. Ker je ena"cba linearna, lahko postavimo $\rho=1$, tako da imamo brezparametri"cen problem. 

\begin{figure}[H]
\centering
  \includegraphics[width=.8\textwidth]{g_cev}
  \caption{Pretok teko"cine po cevi}
  \label{fig:cev}
\end{figure}

Za nestisljivo teko"cino je pretok $\vec j$ sorazmeren s hitrostjo $\vec u$. Ker smo re"sevali ena"cbo za $u$, moramo za dolo"citev skupnega pretoka le integrirati hitrost po celotnem preseku cevi. 

\begin{align}
  \Phi = \int \! u\; \dd S \approx h^2 \sum_{i,j=0}^N u_{i,j} 
\end{align}


V enotah, dolo"cenih z brezdimenzijsko ena"cbo $\nabla u = 1$, je skupni pretok (izra"cunan pri $N=900$) enak $\Phi \approx 0.0163137$. Za primerjavo, pretok po kvadratni cevi je ve"c kot "se enkrat ve"cji in zna"sa $\Phi_\Box \approx 0.0351342$. "Ce upo"stevamo razli"cna preseka obeh cevi lahko izra"cunamo razmerje preto"cnosti

\begin{align}
  \frac{\Phi}{\Phi_\Box} \cdot \frac{S_\Box}{S} &= \frac{\Phi}{\Phi_\Box} \cdot \frac{9}{7} \approx 0,59698975
\end{align}

\section{Prevajanje toplote po valju}

Prevajanje toplote opisuje difuzijska ena"cba $\parcialno{T}{t} = D\nabla^2 T$. Z relaksacijo dobimo stacionarno stanje, torej $\parcialno{T}{t} = 0$. Difuzijska ena"cba tedaj preide v Laplaceovo

\begin{align}
  \nabla^2 T = 0
\end{align}

"Ce spodnjo ploskev grejemo, bo temperatura povsod po valju ve"cja ali enaka $T_1$, zato lahko robni pogoj postavimo na $T_1=0$. Na greti ploskvi velja zakon o prevajanju toplote $P = -\lambda \parcialno{T}{z}$. Ker je celoten problem "se vedno linearen, lahko konstanto $P/\lambda$ dolo"cimo poljubno in robni pogoj zapi"semo kot

\begin{align}
  \left.\parcialno{T}{z}\right|_{z=0} &= -1
\end{align}

Za samo re"sevanje ena"cbe predznak gretja nima pomena, z minusom ga izberemo zaradi la"zje predstave, saj bo v tem primeru temperatura po valju pozitivna. 

Da lahko ena"cbo re"simo numeri"cno moramo operator $\nabla^2$ zapisati v cilindri"cnih koordinatah in ga diskretizirati. 

\begin{align}
  \nabla^2u = \frac{1}{r} \parcialno{}{r} \left(r\parcialno{u}{r}\right) + \parcdva{u}{z} = \frac{1}{r} \parcialno{u}{r} +  \parcdva{u}{r} + \parcdva{u}{z}
\end{align}

Pri tem smo privzeli simetrijo okrog navpi"cne osi, tako da je $\parcialno{u}{\phi} = 0$. Odvod po $z$ lahko diskretiziramo enako kot pri prvi nalogi, nov je le prvi "clen. Ker re"sujemo ena"cbo drugega reda, lahko za prvi odvod vzamemo simetri"cno diskretizacijo. 

\begin{align}
  \left.\frac{1}{r}\parcialno{u}{r}\right|_{r=r_i} &\approx \frac{1}{r_i} \frac{u_{i+1,j}-u_{i-1,j}}{2h} = \frac{u_{i+1,j}-u_{i-1,j}}{2ih^2} \\
  h^2\nabla^2 u &\approx \frac{u_{i+1,j}-u_{i-1,j}}{2i} + u_{i+1,j} + u_{i-1,j} + u_{i,j+1} + u_{i,j-1} - 4u_{i,j} \\
&= (1+\frac{1}{2i})u_{i+1,j} + (1-\frac{1}{2i})u_{i-1,j} + u_{i,j+1} + u_{i,j-1} - 4u_{i,j} 
\end{align}

Pri tem zapisu sem privzel, da je $r_i = i\cdot h$. V mojem primeru sem vzel $i=0$ na robu, tako da sem v ra"cun vstavil $N/2-i$ namesto $i$. Zaradi simetrije sem lahko tudi tokrat ra"cun izvajal le za polovico valja. Ena"cbo sem re"seval enako kot v primeru cevi. Rezultat je na Sliki \ref{fig:valj}

\begin{figure}[H]
\centering
  \includegraphics[width=.8\textwidth]{g_valj}
  \caption{Temperaturni profil v kovinskem valju}
  \label{fig:valj}
\end{figure}

\end{document}
