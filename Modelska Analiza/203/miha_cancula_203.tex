\documentclass[a4paper,10pt]{article}

\usepackage[utf8]{inputenc}
\usepackage[slovene]{babel}
\usepackage{amsmath}
\usepackage{amsfonts}
\usepackage{relsize}
\usepackage[smaller]{acronym}
\usepackage{graphicx}
\usepackage{subfigure}
\usepackage{cite}
\usepackage{url}
\usepackage[unicode=true]{hyperref}
\usepackage{color}
\usepackage[version=3]{mhchem}
\usepackage{wrapfig}
\usepackage{comment}
\usepackage{float}
\usepackage[top=3cm,bottom=3cm,left=3cm,right=3cm]{geometry}

\renewcommand{\vec}{\mathbf}
\newcommand{\eps}{\varepsilon}
\renewcommand{\phi}{\varphi}
\renewcommand{\theta}{\vartheta}
\newcommand{\dd}{\mathrm{d}}

\newcommand{\parcialno}[2]{
  \frac{\partial #1}{\partial #2}
}
\newcommand{\parcdva}[2]{
  \frac{\partial^2 #1}{\partial #2 ^2}
}
\newcommand{\lag}{\mathcal{L}}

\title{Navadne diferencialne ena\v cbe: \\ Robni problem}
\author{Miha \v Can\v cula}
\begin{document}

\maketitle

\section{Napeta vrv}

\section{Kristal}

\subsection{Gibalna ena"cba}

Ker imam podan potencial $U(x,y)$, lahko zapi"semo Lagran"zijan problema kot

\begin{align}
 \lag(x,y,u,v) &= \frac{1}{2} \left( u^2 + v^2 \right) - \frac{1}{2}\ln \left( \sin^2 \pi x + \sin^2 \pi y\right)
\end{align}

Spremenljivki $u$ in $v$ sta brezdimenzijska impulza v smereh $x$ in $y$, ki sta "ze reskalirana z maso in nabojem delca. "Casovna odvoda koordinat $x$ in $x$ sta kar impulza, "casovna odvoda impulzov pa izrazimo iz Euler-Lagrangevih ena"cb za $\lag$. 
\begin{align}
 \dot{x} &= u \\
 \dot{y} &= v \\
 \dot{u} &= \pi \frac{\sin \pi x \cos \pi x}{\sin^2 \pi x + \sin^2 \pi y} \\
 \dot{v} &= \pi \frac{\sin \pi y \cos \pi y}{\sin^2 \pi x + \sin^2 \pi y} 
\end{align}


 
\end{document}