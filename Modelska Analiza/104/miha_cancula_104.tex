\documentclass[a4paper,10pt]{article}

\usepackage[utf8x]{inputenc}
\usepackage[slovene]{babel}
\usepackage{graphicx}
\usepackage{amsmath}
\usepackage{subfigure}
\usepackage{verse}
\usepackage{comment}

\title{Modeli kemijskih reakcij}
\author{Miha \v Can\v cula}

\renewcommand{\phi}{\varphi}
\renewcommand{\epsilon}{\varepsilon}
\renewcommand{\theta}{\vartheta}

\begin{document}
 \section{Binarna reakcija}

\begin{align}
 A + A &\rightleftharpoons A + A^* \\
 A^* &\rightarrow B + C
\end{align}

\begin{verse}
\textit{Na"si reakciji vrli diferencialk par gre zraven.}  \\
\textit{Tol'ko neznank v njem nastopa, kot 'mamo ena"cb mi na voljo.} \\
\end{verse}

\begin{comment}
Reakciji ustreza sistem linearnih ena"cb prvega reda.
\end{comment}

Za dolo"citev hitrosti reakcije potrebujemo le koncentraciji $[A] = x_1$ in $[A^*] = x_2$, zato bo to sistem dveh ena"cb z dvema neznankama. 

\begin{align}
  \dot x_1 &= -p x_1^2 + q x_1 x_2 \label{eq:n1-sistem-1} \\
  \dot x_2 &= p x_1^2 - q x_1 x_2 - r x_2
\end{align}

\subsection{Eksakten sistem}
Vemo, da je za to reakcijo $p >> q$, zato bo $x_2 << x_1$. Dejanske koncentracije reagentov ne poznamo, za samo dinamiko reakcije pa niti ni pomembna, lahko postavimo za"cetni pogoj $x_1(0) = 1$ in $x_2(0) = 0$. Tak"sen sistem ena"cb z za"cetnim pogojem sedaj lahko numeri"cno integriramo. Zaradi velike razlike v absolutnih vrednosti $p$ in $q$ je ta problem tog, kar integratorju povzro"ca te"zave, saj mora ra"cunati z zelo majhnim korakov, "ceprav nas zanima le rezultat v dalj"sem "casovnem obdobju. 

\subsection{Pribli"zek stacionarnega stanja}
Koncentracija vzbujenih molekul $x_2$ se sicer hitro spreminja, vendar je vseskozi zelo majhna v primerjavi z $x_1$. Zato lahko te hitre in majhne spremembe zanemarimo in dodamo pogoj $\dot x_2 = 0$. To nam da pogoj

\begin{align}
 \dot x_2 &= p x_1^2 - q x_1 x_2 - r x_2 = 0\\
 x_2 &= \frac{px_1^2}{qx_1 + r}
\end{align}

\begin{verse}
\textit{Vstavi sedaj ta izraz v ena"cbo, tretjo povrsti. \\% enacba je \ref{eq:n1-sistem-1}
Vse kar ostane je ena diferencialna ena"cba, \\
eno ima le neznanko, in to integrator obvlada. }
\end{verse}

\begin{align}
\dot x_1 &= -p x_1^2 \left( 1 - \frac{qx_1}{qx_1 + r} \right) = -px_1^2 \frac{r}{qx_1 + r}
\end{align}

\section{Reakcija z ve"c stopnjami}

\begin{verse}
\textit{Glej, "ze nas "caka naloga, kjer treba bo ve"c izra"cunat'.  \\
Stopnje reakcija tri 'ma, in pet je neznank v njej neznanih. \\
Pet pa je tudi ena"cb, re"sitev gotovo je mo"zna. \\
A kje se v njej skriva hitrost z vodikom reakcije broma?  }
\end{verse}

\begin{align}
 \dot{[\mathrm{Br_2}]} &= - k_1 [\mathrm{Br_2}] + k_2 [\mathrm{Br}]^2 - k_5 [\mathrm{H}] [\mathrm{Br_2}] \\
 \dot{[\mathrm{Br}]} &= k_1 [\mathrm{Br_2}] - k_2 [\mathrm{Br}]^2 - k_3 [\mathrm{Br}] [\mathrm{H_2}] + k_4 [\mathrm{HBr}] [\mathrm{H}] + k_5 [\mathrm{H}] [\mathrm{Br_2}]\\
 \dot{[\mathrm{H_2}]} &= - k_3 [\mathrm{Br}] [\mathrm{H_2}] + k_4 [\mathrm{HBr}] [\mathrm{H}] \\
 \dot{[\mathrm{H}]} &= k_3 [\mathrm{Br}] [\mathrm{H_2}] - k_4 [\mathrm{HBr}] [\mathrm{H}] - k_5 [\mathrm{H}] [\mathrm{Br_2}]\\
\dot{[\mathrm{HBr}]} &= k_3 [\mathrm{Br}] [\mathrm{H_2}] - k_4 [\mathrm{HBr}] [\mathrm{H}] + k_5 [\mathrm{H}] [\mathrm{Br_2}]
\end{align}

\begin{verse}
\textit{A fizik "ze ve kaj storiti, kako naj ena"cb vseh se re"si, \\
vraga stevil"cnega loti se najprej s pribli"zkom mirovnim. \\
Hitro gredo radikali, al' nikdar jih v zmesi ni dosti. \\
La"zji problem na"s postane, ak' njih fluktuacij ne vid'mo. 
}
\end{verse}

\begin{align}
 \dot{[\mathrm{Br}]} &= \dot{[\mathrm{H}]} = 0
\end{align}


\begin{align}
 \dot{[\mathrm{Br}]} &= k_1 [\mathrm{Br_2}] - k_2 [\mathrm{Br}]^2 - k_3 [\mathrm{Br}] [\mathrm{H_2}] + k_4 [\mathrm{HBr}] [\mathrm{H}] + k_5 [\mathrm{H}] [\mathrm{Br_2}] &= 0 
\end{align}


\end{document}
