\documentclass[a4paper,10pt]{article}

\usepackage[utf8]{inputenc}
\usepackage[slovene]{babel}
\usepackage{amsmath}
\usepackage{amsfonts}
\usepackage{relsize}
\usepackage[smaller]{acronym}
\usepackage{graphicx}
\usepackage{subfigure}
\usepackage{cite}
\usepackage{url}
\usepackage[unicode=true]{hyperref}
\usepackage{color}
\usepackage[version=3]{mhchem}
\usepackage{wrapfig}
\usepackage{comment}

\renewcommand{\vec}{\mathbf}
\newcommand{\eps}{\varepsilon}
\renewcommand{\phi}{\varphi}
\renewcommand{\theta}{\vartheta}

\title{Metoda maksimalne entropije}
\author{Miha \v Can\v cula}
\begin{document}

\maketitle

\section{Uvod}

Spekter mo"ci vhodnega signala smo pribli"zali z izrazom, ki ima v imenovalcu polinom stopnje $m$ spremenljivke $z = exp(-2
\pi i \frac{1}{N})$. 

\begin{align}
 |S(f)|^2 = \frac{C}{1 - b_1 z - b_2 z^2 - \cdots}
\end{align}

"Ce koeficiente izberemo tako, da vsi razen konstante nastopajo z minusom, so $b_i$ enaki avtoregresijskim (AR) koeficientom izvirnega signala.

\begin{align}
 y_n = b_1 y_{n-1} + b_2 y_{n-2} + \cdots = \sum_{i=1}^m y_{n-1} b_i
\end{align}

Avtoregresija ima "sirso uporabo v obdelavi signalov, zato obstaja kar nekaj orodij, ki znajo izra"cunati te koeficiente. Uporabil sem paket za analizo "casovnih zaporedji (Time-Series analysis - TSA) programa Octave. Koeficinte $b_i$ izra"cuna z uporabo avtokovariance in Durbin-Levinsove rekurzije. 

\section{Frekven"cni spekter}

Najprej sem za vse na"stete datoteke s podatki pribli"zal spekter signalov. V vseh primerih sem uporabil najve"c $m=25$ koeficientov. 

\begin{figure}[h]
 \input{g_val2_psd}
 \caption{Spekter signala \texttt{val2.dat}}
 \label{fig:psd-val2}
\end{figure}

\begin{figure}[h]
 \input{g_val3_psd}
 \caption{Spekter signala \texttt{val3.dat}}
 \label{fig:psd-val3}
\end{figure}

\begin{figure}[h]
 \input{g_co2_psd}
 \caption{Spekter signala \texttt{co2.dat}}
 \label{fig:psd-co2}
\end{figure}


\subsection{Polo"zaj polov}

\begin{figure}[h]
 \input{g_co2_roots}
 \caption{Poli preslikave za signal \texttt{co2.dat}}
 \label{fig:roots-co2}
\end{figure}


\section{Lo"cljivost}

\section{Napovedi}

 
\end{document}