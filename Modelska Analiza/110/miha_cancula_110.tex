\documentclass[a4paper,10pt]{article}
\usepackage[utf8x]{inputenc}
\usepackage[slovene]{babel}
\usepackage{amsmath}
\usepackage{amsfonts}
\usepackage{relsize}
\usepackage[smaller]{acronym}
\usepackage{graphicx}
\usepackage{subfigure}
\usepackage{cite}
\usepackage{url}
\usepackage{hyperref}

\usepackage{multirow}

\renewcommand{\theta}{\vartheta}
\renewcommand{\phi}{\varphi}
\newcommand{\eps}{\varepsilon}

\newcommand{\dd}{\,\mathrm{d}}

%opening
\title{Stohasti\v cni populacijski modeli}
\author{Miha \v Can\v cula}

\begin{document}

\maketitle

\newcommand{\poisson}[2]{
  \frac{#2^{#1}}{#1!} e^{-#2}  
}

\newcommand{\poi}[1]{\poisson{#1}{\overline{#1}}}

\section{Statistika "casov izumrtja}

Simuliral sem stohasti"cno izumiranje populacije z dvema razli"cnema modeloma
\begin{enumerate}
 \item Eksponentno-padajo"ci model
 \item Model z rojstvi in umiranjem
\end{enumerate}

"Casovne konstante sem izbral tako, da je v povpre"cju sprememba populacije enaka pri obeh modelih. Pri generiranju "stevila smrti in rojstev sem uporabil funkcijo \texttt{gsl\_ran\_poisson()}, ki vrne naklju"cno "stevilo s Poissonovo verjetnostno porazdelitvijo. 

Za vsak model sem izra"cunal verjetnostno porazdelitev "casov izumrtja z razli"cnimi "casovnimi koraki, pri eksponentnem modelu pa sem za primerjavo dodal se rezultat ra"cuna s prehodno matriko. Vsaki"c sem naredil $10^6$ ponovitev simulacije in iz dobljenih podatkov napravil statistiko. 

\subsection{Eksponentno umiranje}

Ta model najbolje opisuje radioaktivne razpade jeder ali pa relaksacijo iz vzbujenega stanja atomov. 

\begin{figure}
\input{g_iz_exp_25}
\caption{Eksponentni model, $N=25$}
\end{figure}

S slike razberemo, da velikost koraka vidno vpliva porazdelitev pri $\Delta t \geq 0,1$. Ko je korak dovolj majhen, da le zaneraljiv dele"z osebkov umre v vsakem koraku, z dodatnim manj"sanjem ne pridobimo ni"c ve"c na natan"cnosti. Podobno lahko zaklju"cimo tudi, "ce za"cnemo v ve"cjo populacijo.  Vrh populacije se po pri"cakovanju premakne k dalj"sim "casom, odvisnost od "casovnega koraka pa je podobna kot prej. 

\begin{figure}
\input{g_iz_exp_250}
\caption{Eksponentni model, $N=250$}
\end{figure}

Grafa sta si zelo podobna, spodnji je le premaknjena slika zgornjega. To je za pri"cakovati, saj zaradi eksponentnega upadanja pri"cakujemo logaritemsko odvisnost "zivljenjskega "casa od $N$. 

Vidimo tudi, da pri zelo velikih korakih ($\Delta t \approx 1$) velikost populacije skoraj ni"c ne vpliva na "cas izumrtja. V obeh primerih s pribli"zno polovi"cno verjetnostjo pride do izumrtja "ze po prvem koraku, naprej pa verjetnost pada eksponentno. Pri kraj"sih korakih, "ze pri $\Delta t = 0,5$ pa opazimo vidno odvisnost od za"cetne velikosti populacije. 

\subsection{Rojstva in smrti}

\begin{figure}[h!]
\input{g_iz_rs_25}
\caption{Rojstva in smrti, $N=25$}
\label{fig:rs-25}
\end{figure}

\begin{figure}[h!]
\input{g_iz_rs_250}
\caption{Rojstva in smrti, $N=250$}
\label{fig:rs-250}
\end{figure}

Podobno kot prej lahko tudi tu pridemo do opa"zanja, da pri $\Delta t = 1$ velikost populacije ne vpliva mo"cno na "cas izumrtja, pri manj"sih korakih pa je statistika za ve"cjo populacijo le premaknjena v desno. 

"Ce pa primerjamo zgorja grafa s tistimi iz prej"sjnega poglavja pa vidimo, da proces z umiranjem in rojevanjem v povpre"cju vodi do hitrej"sega izumtrja. 

\subsection{Statistika}

S podatkov na zgornjih slikah sem izra"cunal povpre"cen "cas izumrtja in njegovo standardno deviacijo. Dobljene vrednosti so v tabeli~\ref{tab:stat-iz}

\begin{table}[!h]
\centering
 \begin{tabular}{|c|c|c|c|c|}
  \hline 
Model & $N$ & $\Delta t$ & $\langle \tau \rangle$ & $\sigma_\tau$ \\
\hline
\multirow{8}{*}{Eksponentni} & \multirow{4}{*}{25} & 1 & 1.7977 & 1.09922 \\
& & 0.5 & 3.01617 & 1.26343 \\
& & 0.1 & 3.67288 & 1.26503 \\
& & 0.01 & 3.80141 & 1.26712 \\
\cline{2-5}
& \multirow{4}{*}{250} & 1 & 1.85825 & 1.1505 \\
& & 0.5 & 4.65363 & 1.27941\\
& & 0.1 & 5.83871 & 1.27863 \\
& & 0.01 & 6.07588 & 1.28129 \\
\hline
\multirow{8}{*}{Rojstva in smrti} & \multirow{4}{*}{25} & 1 & 1.92274 & 1.30541 \\
& & 0.5 & 2.02552 & 1.32143 \\
& & 0.1 & 2.23531 & 1.1838 \\
& & 0.01 & 2.32887 & 1.17102 \\
\cline{2-5}
& \multirow{4}{*}{250} & 1 & 1.95215 & 1.32993 \\
& & 0.5 & 3.46507 & 1.4197 \\
& & 0.1 & 4.27771 & 1.2814 \\
& & 0.01 & 4.48119 & 1.26931 \\
\hline
 \end{tabular}
\caption{Povre"cna vrednosti in standardna deviacija "zivljenjskega "casa sistema z razli"cnimi modeli}
\label{tab:stat-iz}
\end{table}


\section{Matrika prehodov}

Verjetnost, da v dolo"cenem koraku umre $n$ osebkov izra"cunamo po Poissonovi porazdelitvi:

\begin{align}
\mathcal P(-n, \overline n) &= \poi{n}
\end{align}

"Ce imamo model, kjer se osebki rojevajo in umirajo, moramo upo"stevati oba "clena lo"ceno in se"steti po vseh kombinacijah, ki nam dajo enako spremembo populacije. 

\begin{align}
\mathcal P(n, \overline n) &= \sum_{n_r - n_s = n}\mathcal{P}(n_r, \overline{n_r})\mathcal{P}(n_s, \overline{n_s}) = \sum_{n_r - n_s = n}\poi{n_r}\poi{n_s}  
\end{align}

Povpre"cno "stevilo umrlih oz. rojenih osebkov $\overline{n_i}$ lahko izrazimo s parametrom $\beta$, velikostjo populacije in dol"zino "casovnega koraka

\begin{align}
  \overline{n_i} &= \beta_i N \Delta t
\end{align}

"Ce vzamemo dovolj majhnen korak $\Delta t$, bo $\overline{n}$ mnogo manj"si od 1 in bodo verjetnosti za velike spremembe "stevila populacije majhne. V tem primeru lahko upo"stevamo le "clene z $n=0$ in $n = \pm 1$. 

\subsection{Paremetrizacija stanj}

Stanje z velikostjo populacije $N$ lahko zapi"semo kot vektor $\vec v$ z $M+1$ komponentami, $M \geq N$, ki ima vse $N$-to komponento enako 1, ostale pa 0. Pri tem smo morali postaviti zgornjo mejo za velikost populacije $M$, vektor pa ima "se dodatno komponento za stanje s populacijo 0. Stanja, ki imajo le eno neni"celno komponento, so "cista stanja, zaradi stohasti"cnosti procesa pa po vsakem koraku dobimo me"sano stanje, ki pomeni, da je velikost populacije porazdeljena po neki verjetnostni porazdelitvi. 

Vsak "casovni korak lahko predstavimo z opracijo, ki je linearna v $v$, torej jo lahko zapi"semo kot matriko dimenzije $M+1\times M+1$. Elementi te matrike so verjetnosti za prehod iz $j$-tega v $i$-to stanje

\begin{align}
W_{i,j} &= P(j\to i) = \mathcal P(j-i, \overline{n}(j))
\end{align}

\subsection{Eksponentni model}

Pri tem modelu se osebki ne rojevajo, zato je velikost populacije monotono padajo"ca, torej bo matrika prehodov spodnje trikotna. Predpostavili smo, da je $\Delta t$ tako majhen, da je $\beta N \Delta t = \eps << 1$

\begin{align}
  W_{i,j} &= \mathcal{P}(j-i, \eps) = \poisson{(j-i)}{\eps}
\end{align}

Ker je $\eps$ majhen, lahko "clene z $\eps^2$ zanemarimo, tako da ostanejo le "se elementi na diagonali in tik pod njo

\begin{equation}
  W_{i,j} = \left\{ \begin{matrix} 
		      1 - \beta j \Delta t, & i=j \\
                      \beta j \Delta t, & i=j-1 \\
		      0, & \mathrm{sicer}
                    \end{matrix}\right.
\end{equation}

\subsection{Rojstva in smrti}

Podobno kot prej zanemarimo verjetnosti, da se rodi ali umre ve"c kot en osebek v vsakem "casovnem koraku. Tudi v verjetnosti, da se en rodi in en umre, nastopa $\eps$ z drugo potenco, torej ga lahko zanemarimo. Ostanejo nam le tri mo"znosti: eno rojstvo, ena smrt, ali pa ohranitev istega stanja. V primerjavi s prej"snjim izrazom moramo matriki dodati le "se diagonalo nad glavno, ki dopu"s"ca rast populacije. 

\begin{equation}
  W_{i,j} = \left\{ \begin{matrix} 
                      \beta_r j \Delta t, & i=j+1 \\
		      1 - (\beta_r+\beta_s) j \Delta t, & i=j \\
                      \beta_s j \Delta t, & i=j-1 \\
		      0, & \mathrm{sicer}
                    \end{matrix}\right.
\end{equation}

\subsection{Diferencialne ena"cbe}

Ker matrika prehodov predstavlja en "casovni korak simulacije, lahko z limitiranjem "casovnega intervala pridemo do sistema linearnih diferencialnih ena"cb

\begin{align}
  \dot {\vec v} &= W \vec v
\end{align}

Tak"sen sistem znamo re"siti, tako da poi"s"cemo lastne vrednosti in lastne vektorje matrike $W$. 

\subsection{"Cas izumrtja}

Za preprost ekspronentni model lahko matriko zapi"semo brez te"zav, saj je vsaka sprememba velikosti populacije mo"zna le na en na"cin, torej da neko "stevilo osebkov umre. Za statistiko "casov izumrtja sem najprej izra"cunal matriko za nek majhen "cas $\Delta t$, nato pa z njo mno"zil trenutni vektor stanja iz ob vsakem koraku izra"cunal, za koliko se je pove"cala komponenta stanja s populacijo 0. Ta postopek je mnogo hitrej"si od simulacije, saj lahko naprovimo statistiko le z eno ponovitvijo. Ker je treba matriko izra"cunati le enkrat, sem najprej izra"cunal matriko za majhen korak $\Delta t$, nato pa nadaljnje ra"cune opravljal z neko potenco te matrike. Na ta na"cin sem se lahko izognil prehitremu izumrtju, ki ga opazimo pri simulaciji s kratkim "casovnim korakok, brez da bi moral "zrtvovati hitrost ra"cunanja. 

Rezultati matri"cnega ra"cuna za populaciji s 25 in 250 osebki so na sliki~\ref{fig:matrike-exp}. 

\begin{figure}
 \input{g_matrike_exp}
  \caption{Statistika "casov izumrtja, dobljena z matriko prehodov. Primerjava z rezultati simulacije je na slikah v prvem poglavju. }
  \label{fig:matrike-exp}
\end{figure}

Pri modeliranju procesa z rojstvi in smrtmi pa se stvar zaplete zaradi dveh dejavnikov:

\begin{enumerate}
 \item Mo"zna so stanja, kjer populacija prese"ze za"cetno velikost
 \item Dolo"ceno spremembo velikosti populacije $\Delta N$ lahko dose"zemo na ve"c na"cinov
\end{enumerate}

\section{Povpre"cje in odmik porazdelitve}

Z matri"cnim pristopom je enostavno izra"cunati tudi spreminjanje povpre"cno vrednost in standardno deviacijo velikosti populacije v odvisnosti od "casa. Slika~\ref{fig:m-povprecje} potrdi na"se pri"cakovanje, da populacija v povpre"cju eksponentno pada. 

\begin{figure}[!h]
 \input{g_m_povprecje}
\caption{Spreminjanje povpre"cja porazdelitve s "casom}
\label{fig:m-povprecje}
\end{figure}

Na grafu~\ref{fig:m-dev} pa vidimo, da po za"cetnem nara"scanju tudi "sirina porazdelitve eksponentno pada. 

\begin{figure}[!h]
 \input{g_m_stdev}
\caption{Spreminjanje standardne deviacije porazdelitve s "casom}
\label{fig:m-dev}
\end{figure}

Z obeh grafov je jasno razvidno, da velikost populacije nima nobenega vpliva na "casovni potek njenega spreminjanja, le z ustreznimi konstantami moramo normirati grafa. 

\section{Zajci in lisice}

Z enakim pristopom kot eksponentnega izumiranja sem se lotil tudi problema zajcev in lisic. V teoreti"cnem zveznem primeru bi moral biti ta sistem obstojen, tako da do izumrtja ne bi pri"slo. "Ce pa upo"stevamo diskretno "stevilo osebkov in stohasti"cnost procesa, pa do izumrtja vsaj ene izmed vrst vedno pride. 

"Ce najprej izumrejo zajci, bodo za njimi tudi lisice, saj jim primanjkuje hrane. "Ce pa najprej izumrejo lisice, pa so bodo zajci ekspronento namno"zili, saj njihove hrane nismo upo"stevali v ena"cbah. Da bi se izognil temu primeru sem simulacijo ustavil takoj, ko je izumrla vsaj ena izmed obeh vrst. 

Tudi tu sem izra"cunal verjetnostno porazdelitev ca"sov izumrtja. 

\begin{figure}[!h]
\input{g_zl_50}
\caption{Zajlci in lisice, $Z=200$, $L=50$}
\end{figure}

Ogledal sem si "se primer, ko za"cnemo v legi, ki tudi v zveznem primeru ni ravnovesna. V tem primeru pri"cakujemo hitrej"se izumrtje. 

\begin{figure}[!h]
\input{g_zl_40}
\caption{Zajlci in lisice, $Z=200$, $L=40$}
\end{figure}

\begin{figure}[!h]
\input{g_zl_40}
\caption{Zajlci in lisice, $Z=200$, $L=60$}
\end{figure}

\end{document}
