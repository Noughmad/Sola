\documentclass[a4paper,10pt]{article}
\usepackage[utf8x]{inputenc}

\usepackage{verse}
\usepackage[slovene]{babel}
\usepackage{graphicx}
\usepackage{hyperref}
\usepackage{amsmath}
\usepackage{amsfonts}

%opening
\title{Lu\v s\v cenje modelskih parametrov}
\author{Miha \v Can\v cula}

\begin{document}

\maketitle

\section{Splo"sno}
\begin{verse}
Nalog sem lotil se z Levenberg-Marquardtovim algoritmom. \\
\texttt{Gnuplot} ima ga vgrajen'ga, prav tako tudi \texttt{Octave}. \\
Za vsak primer sem oba uporabil, nato pa primerjal. \\
Malo med njima razlike, drugi ima eno prednost: \\
ve"c argumentov mu damo, ve"c informacij nam vrne. \\ 
Kdaj konvergenco ustavi, lahko sem natan"cno nastavil, \\
malo izbolj"sal na ta na"cin prilagoditve dobroto. 
\end{verse}

\begin{verse}
 ``Goodness of fit'', sem prebral, da s $\chi$ na kvadrat je izra"zen, \\
treba ga le "se deliti s "stevilom prostostnih je stopenj. \\
Re"ce tedaj se tem\`u, da je $\chi^2$ reduciran. \\
Fit je najbolj"si takrat, ko blizu je vrednosti ena. \\
\end{verse}

\begin{align}
 \chi^2 &= \sum_{i=1}^N \left(\frac{f(x_i, \mathbf{p}) - y_i}{\sigma_i}\right)^2 \\
 \chi^2_{red} &= \frac{\chi^2}{N-n}
\end{align}

\begin{verse}
 Veliki $N$ v tem primeru "stevilo je na"sih meritev, \\
 mali pa $n$ je "stevilo prostih parametrov fit-a. \\
 $\sigma$ od $i$ je napaka, ki bla je z nalogo podana, \\
 $f$ pa je funkcija na"sa, odvisna od $x$ in parametrov. \\
\end{verse}

\section{Farmakolo"ski model}
\subsection{Parameter $p$}

\begin{verse}
V model kar iz naloge prej"snje parameter $p$ "se dodamo. \\ 
Legla bo bolje k podatkom, izbolj"sana na"sa napoved. \\
"Zal pa zgubili smo s tem linearnost, gr"si problem je, \\
Levenberg-Marquardt edino je upanje k hitri re"sitvi.
\end{verse}

\subsection{Statisti"cna upravi"cenost}
\begin{verse}
 Smisla kaj dosti pa nima, da si problem bi zapletli, \\
 ve"c spremenljivk pridodali, "ce n'bene od njih ni koristi. \\
 Smiselnost nov'ga modela zato $\chi$-kvadrat bo ocenil. \\
 Ker spremeni se "stevilo neznank, ga "se reduciram. 
\end{verse}

\subsection{Rezultati}
\begin{verse}
 V prvem primeru je fajn, saj prav vse so napake enake, \\
 formula za izra"cun dobrote je bolj enostavna. \\
 Vse sem ra"cune po dvakrat naredil, saj dva 'mam modela, \\
 eden je s fiksnim $p$, ena, a v drugem je $p$ spremenljivka. \\
 Vsaki"c je na"s algoritem proste parametre vrnil, \\
 vrednosti njih optimalne podane so v prvi tabeli. 
\end{verse}

\begin{table}[h]
 \centering
\begin{tabular}{|c|c|c|c|c|c|}
 \hline
  Model & $y_0$ & $a$ & $p$ & $\chi^2_{red}$ \\
\hline
  Dva parametra & 106 $\pm$ 5 & 25 $\pm$ 4 & 1 & 3,67 \\
  Trije parametri & 100 $\pm$ 4 & 20 $\pm$ 2 & 1,4 $\pm$ 0,2 & 1,87 \\
\hline
\end{tabular}
\caption{Primerjava med modeloma z dvema oz. tremi spremenljivkami. }
\label{tab:farmacija}
\end{table}

\begin{verse}
Vidi se z zgornje tabele, da nov parameter pomaga, \\
fit-a dobroto izbolj"sa, blizu enici jo pahne. \\
Ampak pozoren pogled "se to pridobitev razkrije: \\
bli"zje je $y_0$ zdaj pri"cakovani stotici. 
\end{verse}

\begin{figure}[h]
 \input{farma}
  \caption{Primerjava med modeloma z dvema oz. tremi spremenljivkami. }
  \label{fig:farmacija}
\end{figure}

\begin{verse}
 La"zjo predstavo da graf, pod katerim enica se skriva. \\
 Modra se "crta na videz "ze bolje ujema s podatki. \\
 Prednost le-te pred zeleno je vidna predvsem pri robovih. 
\end{verse}

\newpage
\subsection{Odvisnost od $p$}
\begin{verse}
 Kaj pa "ce $p$-ja ne b\'i prilagajali v na"sem ra"cunu, \\
 vseeno pa ne bi prav vedno natan"cno enici enak bil? \\
 Dal sem mu vrednosti razne, od "cetrtine do "stiri, \\
 gledal sem kaj se godilo je z mero za fita dobroto. 
\end{verse}

\begin{figure}[h]
 \input{farma_hi}
\caption{Odvisnost $\chi^2_{red}$ od fiksnega parametra $p$ }
\label{fig:farmacija-hi}
\end{figure}

\newpage

\section{Ledvice}
\begin{verse}
 Isti postopek kot prej, "se tu bomo uporabili, \\
 vse kar se res spremeni je funkcija testna modela. \\
 Druge so tudi napake, razpade zdaj "stejemo jeder, \\
 $\sigma$ je spet enostavna, koren je "stevila razpadov. \\
 Vse je ostalo kot prej, kriterij dobrote obdr"zimo. 
\end{verse}

\subsection{Modeli}
\begin{verse}
 Trije so tak"sni modeli, ki sem jih v nalogi preverjal, \\
 Dva sta razdel"cna, kjer kri v posode zapremo, \\
 eden pa tak, da je ca"s v eksponentu celo pod korenom. 
\end{verse}

\begin{align}
 f_1(t) &= A \exp(-\lambda t) \\
 f_2(t) &= A \exp(-\lambda_1 t) + B \exp(-\lambda_2 t) \\
 f_3(t) &= A \exp(-\lambda \sqrt{t-t_0})
\end{align}

\begin{verse}
 Zadnji model je poseben, saj va"zen je "cas na za"cetku. \\
 Kdaj smo meritve za"celi, poskusil sem s fitom dognati, \\
 dal spremenljivko sem zraven, $t_0$, da ta "cas sem izlu"s"cil. 
\end{verse}

\subsection{Rezultati}

\begin{verse}
 Slika slede"ca poka"ze, da najbolj"s' model je tadrugi. \\
 Tisti s krvjo v dveh posodah, ki 'mata razli"cne volumne. \\
\end{verse}

\begin{figure}[h]
 \input{ledvice}
  \caption{Modeli "cistilnosti ledvic. }
  \label{fig:ledvice}
\end{figure}

\begin{verse}
 Koliko res pridobimo, ko drugo posodo dodamo? \\
 ``Goodness'' se vidno izbolj"sa, kar petdesetkrat je zdaj manj"sa. \\
 Dale"c najbolj"si model je tisti s prekatoma dvema, \\
 tisti model, ki korene ima, po dobroti je v sredi. 
\end{verse}

\begin{table}[h]
 \centering
\begin{tabular}{|c|c|c|c|c|c|}
 \hline
  Model & $\chi^2_{red}$ \\
\hline
  En razdelek & 156,0 \\
  Dva razdelka & 3,0 \\
  Korenska odvisnost & 14,4 \\
\hline
\end{tabular}
\caption{Statisti"cna upravi"cenost modelov "cistilnosti ledvic. }
\label{tab:ledvice}
\end{table}

\begin{verse}
 V tretjem modelu nastopa "se "cas ob za"cetku meritve, \\
 prva meritev v tabeli ob "casu $t_0$ je nar'jena. \\
 Najbolj"si fit sem dobil, "ce blizu ni"cle ta "cas bil. \\
 Ve"cja od njega bila je napaka, pribli"zno stotina. 
\end{verse}

\begin{align}
 t_0 &= 1,5 \cdot 10^{-5} \pm 9 \cdot 10^{-3}
\end{align}


\section{Korozija}
\begin{verse}
 Tretji"c postopek spet isti na novih podatkih nar'dimo. \\
 Trik pa je nov v tem primeru, da za"cetni pribli"zki so te"zji. \\
 Ve"c sem poskusil jih, gledal sem kdaj $\chi^2$ je najmanj"si. \\
\end{verse}

\begin{verse}
  Merske napake sedaj eksplicitno niso podane. \\
  Vemo da so le v toku, in vse so med sabo enake. \\
  Tok je izmerjen na pet decimalk, to oceni napako. \\
  Dobra verjetno bo vrednosti, deset z minus peto potenco. \\
  Same dobrote ne mor'mo podati, a z njun'ga razmerja, \\
  vidimo ko'lko koristi popravek $U_0$ nam prinese. 
\end{verse}

\begin{table}[h]
 \centering
\begin{tabular}{|c|c|c|c|c|c|}
\hline
Model & $I_0$ [nA] & $U_a$ [mV] & $U_c$ [mV] & $U_0$ [mV] & $\chi^2_{red}$ \\
\hline
Brez popravka & 2,6 $\pm$ 0,4 & 140 $\pm$ 25 & 74 $\pm$ 7 & 0 & 408 \\
S popravkom & 3,2 $\pm$ 0,2 & 200 $\pm$ 24 & 77 $\pm$ 3 & -4,6 $\pm$ 0,4 & 68 \\
\hline
\end{tabular}
\caption{Optimalni parametri in dobrota prilagoditve za oba modela prevodnosti korozivnega stika}
\end{table}

\begin{verse}
 Drugi model 'ma dobroto, ki dosti je bolj"sa od prve, \\
 ampak "se vedno velika, dale"c od ciljne enice. \\
 Merske napake ocena, je zgornja o"citno premajhna. \\
 "Ce bi napako z deset pomno"zili, morda bi b'lo bolje. \\
 s sto bi dobroto delil, kvadrat v njej napake nastopa. \\
 Na"s pa model te meritve opisal bi skor' idealno. \\
 Tako napako sem tudi narisal, na sliki poslednji. 
\end{verse}

\begin{figure}[h]
 \input{korozija}
 \caption{$U-I$ diagram med kovino in korozivnim elektrolitom}
 \label{fig:korozija}
\end{figure}

\newpage

\begin{verse}
 Tudi na sliki se vidi, da to"cke ne grejo skoz' ni"clo. \\
 "Crta zelena pokriva meritve kar dobro na koncih,\\
 blizu sredine pa njeno odstopanje res je mote"ce. \\
 Ostri pogled na ta graf potrjuje koristnost popravka. 
\end{verse}


\end{document}
