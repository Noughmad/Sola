\documentclass[a4paper,10pt]{article}
\usepackage[utf8x]{inputenc}

\usepackage{verse}
\usepackage[slovene]{babel}
\usepackage{graphicx}
\usepackage{hyperref}
\usepackage{amsmath}
\usepackage{amsfonts}

%opening
\title{Lu\v s\v cenje modelskih parametrov}
\author{Miha \v Can\v cula}

\begin{document}

\maketitle

\section{Farmakolo"ski model}
\subsection{Parameter $p$}

\begin{verse}
V model iz naloge prej"snje parameter $p$ "se dodamo. \\ 
Legla bo bolje k podatkom, izbolj"sana na"sa napoved. \\
"Zal pa zgubili smo s tem linearnost, gr"si problem je. \\
Levenberg-Marquardt edino je upanje k hitri re"sitvi.
\end{verse}

\subsection{Statisti"cna upravi"cenost}
\begin{verse}
 Smisla kaj dosti pa nima, da si problem bi zapletli, \\
 ve"c spremenljivk pridodali, "ce n'bene od njih ni koristi. \\
 Smiselnost nov'ga modela zato $\chi$-kvadrat bo ocenil. 
\end{verse}

\begin{align}
 \chi^2 &= \sum_{i=1}^n \left(\frac{f(x_i, \mathbf{p}) - y_i}{\sigma_i}\right)^2 \\
\end{align}

\begin{verse}
 $\sigma$ od $i$ je napaka, ki bla je z nalogo podana. \\
 V na"sem primeru je fajn, saj prav vse so trojki enake. \\
 $f$ pa je funkcija na"sa, odvisna od $x$ in parametrov. \\
 Vrednosti njih optimalne podane so v prvi tabeli. \\ 
\end{verse}

\begin{verse}
 ``Goodness of fit'', sem prebral, da s $\chi$ na kvadrat je izra"zen, \\
Treba ga le "se deliti s "stevilom prostostnih je stopenj. \\
Re"ce tedaj se tem\`u, da je $\chi^2$ reduciran. \\
Fit je najbolj"si, ko blizu je vrednosti ena. \\
\end{verse}

\begin{align}
 \chi^2_{red} &= \frac{\chi^2}{N-n}
\end{align}

\begin{verse}
 Veliki $N$ v tem primeru "stevilo je na"sih meritev, \\
 mali pa $n$ je "stevilo prostih parametrov fit-a. 
\end{verse}

\subsection{Rezultati}

\begin{table}[h]
 \centering
\begin{tabular}{|c|c|c|c|c|c|}
 \hline
  Model & $y_0$ & $a$ & $p$ & $\chi^2_{red}$ \\
\hline
  Dva parametra & 106 $\pm$ 5 & 25 $\pm$ 4 & 1 & 3,67 \\
\hline
  Trije parametri & 100 $\pm$ 4 & 20 $\pm$ 2 & 1,4 $\pm$ 0,2 & 1,87 \\
\hline
\end{tabular}
\caption{Primerjava med modeloma z dvema oz. tremi spremenljivkami. }
\label{tab:farmacija}
\end{table}

\begin{verse}
Vidi se z zgornje tabele, da nov parameter pomaga. \\
Fit-a dobroto izbolj"sa, blizu enici jo pahne. \\
Ampak pozoren pogled "se to pridobitev razkrije: \\
bli"zje je $y_0$ zdaj pri"cakovani stotici. 
\end{verse}

\begin{figure}
 \input{farma}
  \caption{Primerjava med modeloma z dvema oz. tremi spremenljivkami. }
  \label{fig:farmacija}
\end{figure}

\begin{verse}
 La"zjo predstavo da graf, pod katerim enica se skriva. \\
 Modra se "crta na videz "ze bolje ujema s podatki. \\
 Prednost le-te pred zeleno je vidna predvsem pri robovih. 
\end{verse}

\subsection{Odvisnost od $p$}
\begin{verse}
 "Ce $p$-ja ne bi prilagajali, vrednost mu raje kar dali, \\
 Gledal sem kaj bi godilo se, z mero za fita dobroto. 
\end{verse}

\begin{figure}
 \input{farma_hi}
\caption{Odvisnost $\chi^2_{red}$ od fiksnega parametra $p$ }
\label{fig:farmacija-hi}
\end{figure}

\begin{figure}
 \input{farma_hi_log}
\caption{Odvisnost $\chi^2_{red}$ od fiksnega parametra $p$ -- logaritemski graf }
\label{fig:farmacija-hi}
\end{figure}

\section{Ledvice}
\begin{verse}
 Isti postopek kot prej, "se tu bomo uporabili, \\
 vse kar se res spremeni je funkcija testna modela. \\
 Druge so tudi napake, razpade zdaj "stejemo jeder, \\
 $\sigma$ je kar enostavna, koren je "stevila razpadov. \\
 Vse je ostalo kot prej, kriterij dobrote obdr"zimo. 
\end{verse}

\subsection{Modeli}
\begin{verse}
 Trije so tak"sni modeli, ki sem jih v nalogi preverjal, \\
 Dva sta razdel"cna, kjer kri v posode zapremo, \\
 eden pa tak, da je ca"s v eksponentu celo pod korenom. 
\end{verse}

\begin{align}
 f_1(t) &= A \exp(-\lambda t) \\
 f_2(t) &= A \exp(-\lambda_1 t) + B \exp(-\lambda_2 t) \\
 f_3(t) &= A \exp(-\lambda \sqrt{t-t_0})
\end{align}

\begin{verse}
 Zadnji model je poseben, saj va"zen je "cas na za"cetku. \\
 Kdaj smo meritve za"celi, poskusil sem s fitom dognati, \\
 dal spremenljivko sem zraven, $t_0$, da ta "cas sem izlu"s"cil. 
\end{verse}

\subsection{Rezultati}

\begin{verse}
 Slika tadruga poka"ze, da najbolj"s model je tadrugi. \\
 Tisti s krvjo v dveh posodah, ki 'mata razli"cne volumne. \\
\end{verse}

\begin{figure}
 \input{ledvice}
  \caption{Modeli "cistilnosti ledvic. }
  \label{fig:ledvice}
\end{figure}

\begin{verse}
 Koliko res pridobimo, ko drugo posodo dodamo? \\
 ``Goodness'' se vidno izbolj"sa, kar petdesetkrat je zdaj manj"sa. \\
 Tisti model, ki korene ima, po dobroti je v sredi. 
\end{verse}

\begin{table}[h]
 \centering
\begin{tabular}{|c|c|c|c|c|c|}
 \hline
  Model & $\chi^2_{red}$ \\
\hline
  En razdelek & 156,0 \\
  Dva razdelka & 3,0 \\
  Korenska odvisnost & 14,4 \\
\hline
\end{tabular}
\caption{Statisti"cna upravi"cenost modelov "cistilnosti ledvic. }
\label{tab:ledvice}
\end{table}

\begin{verse}
 Dale"c najbolj"si model je tisti s prekatoma dvema. \\
 Lambdi pri njem sta v razmerju, tako kot razdelkov volumna. \\
 Vemo da "clovek v sebi pribli"zno pet litrov krvi 'ma. \\
 Sklepamo torej lahko o volumnu ob"zilnih prostorov. 
\end{verse}

\section{Korozija}


\end{document}
